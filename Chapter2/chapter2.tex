%!TEX root = ../thesis.tex
%*******************************************************************************
%****************************** Second Chapter *********************************
%*******************************************************************************

\chapter{Literature Review}

\ifpdf
    \graphicspath{{Chapter2/Figs/Raster/}{Chapter2/Figs/PDF/}{Chapter2/Figs/}}
\else
    \graphicspath{{Chapter2/Figs/Vector/}{Chapter2/Figs/}}
\fi


%**************************** %Zero Section  ***********************************
\section{Source of Variability in Human Movement}
\section{Sensors}
\section{Variability within and between persons}
\section{Variability for simple and complex activities}
\section{Techniques to measure human movement variability}




\section[Short title]{Reasonably long section title}

I'm going to randomly include a picture Figure~\ref{fig:minion}.

\begin{figure}[htbp!]
\centering
\includegraphics[width=0.5\textwidth]{minion}
\caption[Minion]{This is just a long figure caption for the minion in Despicable Me from Pixar}
\label{fig:minion}
\end{figure}


\section*{Enumeration}
Lorem ipsum dolor sit amet, consectetur adipiscing elit. Sed vitae laoreet lectus.

Nunc et dolor diam. Phasellus eu justo vitae diam vehicula tristique.
\begin{enumerate}
\item The first topic is dull
\item The second topic is duller
\begin{enumerate}
\item The first subtopic is silly
\item The second subtopic is stupid
\end{enumerate}
\item The third topic is the dullest
\end{enumerate}


\section*{Itemize}
\begin{itemize}
\item The first topic is dull
\item The second topic is duller
\begin{itemize}
\item The first subtopic is silly
\item The second subtopic is stupid
\end{itemize}
\item The third topic is the dullest
\end{itemize}

\section*{Description}
\begin{description}
\item[The first topic] is dull
\item[The second topic] is duller
\begin{description}
\item[The first subtopic] is silly
\item[The second subtopic] is stupid
\end{description}
\item[The third topic] is the dullest
\end{description}


\clearpage

\tochide\section{Hidden section}
\textbf{Lorem ipsum dolor sit amet}, \textit{consectetur adipiscing elit}.
In magna nisi, aliquam id blandit id, congue ac est. Fusce porta consequat leo.

Etiam elementum tristique lacus, sit amet eleifend nibh eleifend sed
\footnote{My footnote goes blah blah blah! \dots}. Maecenas dapibu augue ut urna
malesuada, non tempor nibh mollis.


\begin{landscape}

\section*{Subplots}
I can cite Wall-E (see Fig.~\ref{fig:WallE}) and Minions in despicable me (Fig.~\ref{fig:Minnion}) or I can cite the whole figure as Fig.~\ref{fig:animations}


\begin{figure}
  \centering
  \begin{subfigure}[b]{0.3\textwidth}
    \includegraphics[width=\textwidth]{TomandJerry}
    \caption{Tom and Jerry}
    \label{fig:TomJerry}
  \end{subfigure}
  \begin{subfigure}[b]{0.3\textwidth}
    \includegraphics[width=\textwidth]{WallE}
    \caption{Wall-E}
    \label{fig:WallE}
  \end{subfigure}
  \begin{subfigure}[b]{0.3\textwidth}
    \includegraphics[width=\textwidth]{minion}
    \caption{Minions}
    \label{fig:Minnion}
  \end{subfigure}
  \caption{Best Animations}
  \label{fig:animations}
\end{figure}


\end{landscape}
