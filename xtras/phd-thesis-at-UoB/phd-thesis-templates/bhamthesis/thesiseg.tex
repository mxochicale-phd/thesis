\documentclass{bhamthesis}
\title{Annotated quotations:\\
       an example using \clsname}
\author{Tin Lok Wong}
\date{December~2009}  %% Version 2009/12/26

\usepackage{amsthm}
\usepackage{amssymb}

\newtheorem*{thm}{Theorem}

\theoremstyle{definition}
\newtheorem*{defn}{Definition}

\newcommand{\mar}[1]{\marginpar{\raggedright#1}}
\newcommand{\clsname}{\textsf{bhamthesis}}
\newcommand{\bktitle}[1]{\textit{#1}}
\newcommand{\ZF}{\mathrm{ZF}}
\newcommand{\IN}{\mathbb{N}}

\prefixappendix

\begin{document}
\frontmatter
\maketitle

\begin{abstract}
 This document only aims to demonstrate how one can use \clsname,
 a \LaTeX\ class written by me.  Please see the source file for
 more comments.  Instead of making something up myself, I quote
 several pieces that may be of interest to mathematicians, and
 include some of my comments. \emph{The contents of this document
 should not be treated seriously.}  In particular, this document
 is not meant to be a thesis intended for submission.
\end{abstract}

\tableofcontents


\mainmatter
\chapter{Thesis writing}
\section{Pictures}
It is not surprising that the English proverb `a picture is worth
a thousand words' has analogues in many languages, including
Chinese and Japanese.  Pictures are a very useful tool in
explaining mathematics.  The following is from Hardy's \bktitle{A
Mathematician's Apology}~\cite[\S23]{book:math-apol}.
\begin{quotation}
 Let us suppose that I am giving a lecture on some system of
 geometry, such as ordinary Euclidean geometry, and that I draw
 figures on the blackboard to stimulate the imagination of my
 audience, rough drawings of straight lines or circles or
 ellipses.  It is plain, first, that the truth of the theorems
 which I prove is in no way affected by the quality of my
 drawings.  Their function is merely to bring home my meaning
 to my hearers, and, if I can do that, there would be no gain in
 having them redrawn by the most skillful draughtsman.  They are
 pedagogical illustrations, not part of the real subject-matter
 of the lecture.
\end{quotation}
\TeX\ and \LaTeX\ are not very good at drawing pictures.  However,
it may still worth the effort to include some more important
diagrams.

\section{The mathematical experience}
In the following excerpt from \bktitle{The Mathematical
Experience}~\cite[pp.~34--37]{book:math-exp}, Davis and Hersh
described how the ideal mathematician writes.
\begin{quotation}
 To talk about the ideal mathematician at all, we must have a name
 for his ``field,'' his subject.  Let's call it, for instance,
 ``non-Riemannian hypersquares.''  [\ldots]

 To his fellow experts, [the ideal mathematician] communicates
 [his] results in a casual shorthand.  ``If you apply a tangential
 mollifier to the left quasi-martingale, you can get an estimate
 better than quadratic, so the convergence in the Bergstein
 theorem turns out to be of the same order as the degree of
 approximation in the Steinberg theorem.''

 This breezy style is not to be found in his published writings.
 There he piles up formalism on top of formalism.  Three pages of
 definitions are followed by seven lemmas and, finally, a theorem
 whose hypotheses take half a page to state, while its proof
 reduces essentially to ``Apply  Lemmas 1--7 to definitions
 A--H.''

 His writing follows an unbreakable convention: to conceal any
 sign that the author or the intended reader is a human being.  It
 gives the impression that, from the stated definitions, the
 desired results follow infallibly by a purely mechanical
 procedure.  In fact, [to] read his proofs, one must be privy to a
 whole subculture of motivations, standard arguments and examples,
 habits of thought and agreed-upon modes of reasoning.  The
 intended readers (all twelve of them) can decode the formal
 presentation, detect the new idea hidden in lemma~4, ignore the
 routine and uninteresting calculations of lemmas~1, 2, 3, 5, 6,
 7, and see what the author is doing and why he does it.  But for
 the uninitiate, this is a cipher that will never yield its
 secret.  If (heaven forbid) the fraternity of non-Riemannian
 hypersquarers should ever die out, our hero's writings would
 become less translatable than those of the Maya.
\end{quotation}
Perhaps mathematical writing, and especially thesis writing,
should not be this cold?  Why should one take away \emph{all} the
human elements from his/her three years of mathematical
experience?

\section{Respectable mathematics}
How much is your thesis supervisor involved in your thesis
writing?  Crilly and Johnson tell the story of Brouwer in their
chapter in \bktitle{History of
Topology}~\cite[Section~7]{incoll:emerg-topodim}.
\begin{quotation}
 [Brouwer's] doctoral thesis of 1907, \textit{On the Foundations
 of Mathematics} (\textit{Over de Grondslagen der Wiskunde}),
 marked the real beginning of his mathematical career.  The work
 revealed the twin interests in mathematics that dominated his
 entire career: his fundamental concern with critically assessing
 the foundations of mathematics, which led to his creation of
 Intuitionism, and his deep interest in geometry, which led to his
 seminal work in topology [\ldots].  Brouwer quickly found that
 his philosophical ideas sparked controversy.  D.J.~Korteweg
 (1848--1941), his thesis supervisor, had not been pleased with
 the more philosophical aspects of the thesis and had even
 demanded that several parts of the original draft be cut from the
 final presentation [\ldots].  Korteweg urged Brouwer to
 concentrate on more ``respectable'' mathematics, so that the
 young man might enhance his mathematical reputation and thus
 secure an academic career.
\end{quotation}
You may get the same `advice' from your supervisor, 90~years after
Brouwer's days, if you try to put some philosophy into your thesis
(and if your supervisor reads it).  I do not think one should be
discouraged for putting his/her ideas into the thesis, as long as
the majority of the work is still mathematics.  (This is only my
personal opinion.)  Given Brouwer's strong personality, one may
expect him to have insisted on what he wanted to do.  For some
reason, he did not.  As Crilly and
Johnson~\cite{incoll:emerg-topodim} write:
\begin{quotation}
 Brouwer was fiercely independent and did not follow in anybody's
 footsteps, but he apparently took his teacher's advice and set
 out to solve some really hard problems of mathematics.  Brouwer
 put in a prodigious effort in these early years and rapidly
 produced a flood of papers on continuous group theory and
 topology --- more than forty major papers in less than five
 years [\ldots].
\end{quotation}
Perhaps I would only have a choice if I were as good as Brouwer.

By the way, the `rejected parts' of Brouwer's thesis is now
published~\cite{art:brouwer-rej}.  You can even see the big
crosses his supervisor drew on his drafts.


\chapter{The nature of mathematics}
\section{Arithmetical splitting}
I was recently in a conversation with Andrey Bovykin, a former
graduate student of the University of Birmingham.  He told me
about the idea of \emph{arithmetic splitting}, which essentially
says that there is no `absolute truth' about the natural numbers.
Pettigrew~\cite[pp.~19--20]{unpub:uniqueIN} explains it in a
better way.
\begin{quotation}
 This list of facts [\ldots] gives a glimpse of the varied zoology
 of natural number structures that it supports.  This, I propose,
 should be our foundation for arithmetic.  [\ldots]

 Of course, at first sight this proposal will seem extremely
 radical. It will seem that it entails changing much
 number-theoretic practice.  Most importantly, if this foundation
 were adopted, each of our number-theoretic statements would have
 to be relativized to a particular collection of natural number
 structures.  For instance, if I were to say that there are
 infinitely many primes, I would have to say in which natural
 number structures I take this to holds.  Does it hold only in
 those closed under exponentiation?  Or also in weaker structures?
 Thus, on my proposal, arithmetic would come to resemble branches
 of algebra such as group theory, in which we prove theorems that
 hold of all Abelian groups, for instance, or all cyclic groups;
 or field theory, in which sometimes we prove theorems that hold
 of all finite fields.  [\ldots]

 But perhaps [this] feature is [not] as revolutionary as it seems
 at first.  [\ldots]  Much current research in mathematical logic
 concerns the strength of the number-theoretic assumptions
 required to prove certain propositions.  For instance, in the
 case of Euclid's theorem that there are infinitely many primes,
 it has recently been shown that this holds in more natural number
 structures than previously thought.[\ldots]  This result, along
 with many, many similar to it, belongs to the research project
 known as \emph{Weak} or \emph{Bounded Arithmetic}, which studies
 the deductive power of a certain sort of fragment of first-order
 arithmetic.[\ldots]  In a similar vein, the research project of
 \emph{Reverse Mathematics}, inaugurated by Harvey Friedman and
 carried on by Stephen~G. Simpson, aims to identify, for a given
 proposition, the weakest fragment of second-order arithmetic in
 which that proposition may be proved.[\ldots]  Both research
 projects occupy central positions in contemporary research in
 arithmetic (more usually called \emph{number theory} by
 mathematicians).  So, it seems that a foundation for arithmetic
 in which \textsc{Uniqueness} fails might be more appropriate to
 the concerns of contemporary arithmetic than the traditional
 foundation, which guarantees \textsc{Uniqueness}.

 Furthermore, arithmetic belongs to that part of mathematics that
 is used to model phenomena in other disciplines.  And, in some of
 these disciplines, it may well help to be able to choose between
 natural number structures with different properties.  For
 instance, consider the notion of feasibility in computer science.
 We might well wish to say that the class of natural numbers that
 measure the feasible computations is closed under successor, but
 not under exponentiation [\ldots].  And, if so, it will be
 difficult to model the notion of feasibility in arithmetic with
 the traditional foundation.  But it is straightforward on the
 foundation proposed in this paper since there are natural number
 structures closed under successor but not under exponent[i]ation.
 Such a proposal requires more work than I have space to carry out
 here, but it suggests that there may be advantages to the
 foundation described and advocated in the previous section.
 [\ldots]

 I conclude that mathematics took a wrong turning when it accepted
 the uniqueness of natural number structures and built into its
 foundations presuppositions that guarantee that uniqueness.  Had
 it not taken that wrong turning, the orthodox foundation for
 arithmetic might have been $\mathsf{BST}^{\sf Bnd}_2$, which
 permits many different sorts of natural number structure:
 structures closed under addition and structures closed under
 multiplication, structures shorter than a structure that is
 closed under exponentiation, and so on. Had this been the
 foundation for arithmetic, we might have begun the study of
 so-called Weak Arithmetic and Reverse Mathematics much earlier
 than we did.  I propose we rectify our error now.
\end{quotation}

From my education and experience, I am convinced that the set of
natural numbers exists objectively.  Therefore, in Pettigrew
terminology, there is only one true `natural number structure' for
me.  Set theory is different.  We thought we knew a lot about
sets, and we thought we could `visualize' the universe of sets.
Then it came Russell's Paradox, and everyone, including me, gave
up (some of) our intuitions and work with formal axioms instead.
Therefore, I do not believe there is an objectively existing
universe of sets, just as I do not believe set theory is
consistent.\footnote{I suppose it is now common consensus amongst
set theorists that there is a variety of set theoretic universes,
none of which is superior to the others.  There are (only) one or
two exceptions.  For example, Woodin thinks
$2^{\aleph_0}=\aleph_2$ with strong reasons.}\mar{This is a widow,
probably caused by the footnote.}  %% This is a marginal note.

I tried to think as Bovykin and Pettigrew suggest, but I still
cannot convince myself.  Here is one of the arguments Bovykin told
me: you were convinced that natural numbers exist in exactly the
same way as you were convinced `the' universe of sets exists; why
should you believe in the first argument but not the second? There
is also some objective evidence supporting arithmetical splitting,
e.g., G\"odel's Incompleteness Theorems (depending on how you view
it), but \emph{there has to be one and only one $\IN$!}  Perhaps I
just have a psychological barrier?

On the other hand, I would never call what is commonly known as
arithmetic `number theory', nor would I call number theory
`arithmetic'.  They are different.

Logicians may be biased.  What do other mathematicians think?

\section{Topological spaces}
\begin{defn}
 A \emph{topological space} is a pair $(X,\mathcal{T})$ where $X$
 is a nonempty set and $\mathcal{T}$ is a collection of subsets of
 that contains the empty set and $X$, and is closed under finite
 intersections and arbitrary unions.
\end{defn}
As mathematicians, we are familiar with the idea of a topological
space.  The notion of topology abstracts the idea of open sets in
a metric space, and the axioms for a topological space come from
the properties of open sets.  I remember having some difficulties
accepting the definition of a topological space in my second year.
Why are the open sets a good candidate for abstracting the notion
of continuity?  What do they model?  If they model open sets in a
metric space, then why should we allow topological spaces that are
not Hausdorff?  As Vickers said~\cite{unpub:vickers/dptsem},
mathematicians sometimes choose to work with a notion just because
it works.  Perhaps mathematicians should look for better
motivations.  In the particular case of topological spaces, you
may find the answers to my questions in Vicker's
book~\cite{book:topologic}.\mar{This is another awkward looking
widow.}

\section{Set theory as \emph{the} foundation}
I recently participated in a conversation between several
mathematicians, most of which are, strictly speaking, not
logicians.  One of them, who happens to know quite a lot of logic,
insisted that \emph{all} mathematicians should know enough set
theory to distinguish whether they are working with sets, (proper)
classes, or families of class, etc.\ to avoid running into set
theoretic paradoxes.

I disagree with him.  As a logician, I like set theory being
popularized too, but mathematics \emph{is} not set theory. Conway
devoted a whole appendix in his \bktitle{On Numbers and
Games}~\cite{book:numgames} to this topic.  Here is an excerpt on
pages~65--67.

\begin{quotation}
 In this simpler formalisation, a number is still a pretty
 complicated thing, namely a certain function in $\ZF$, which is
 of course a certain set of Kuratowskian ordered pairs.  The first
 members of these ordered pairs will be ordinals in the sense of
 von Neumann, and the second members chosen from the particular
 two-element set we take to represent $\{{+},{-}\}$.

 The curiously complicated nature of these constructions tells us
 more about the nature of formalisations within $\ZF$ than about
 our system of numbers, and it is partly for this reason that we
 did not present any such formalised theory in this book. But the
 main reason was that we regard it as almost self-evident that our
 theory is as consistent as $\ZF$, and that formalisation in $\ZF$
 destroys a lot of its symmetry.  [\ldots]

 It seems to us, however, that mathematics has now reached the
 stage where formalisation within some particular axiomatic set
 theory is irrelevant, even for foundational studies. It should be
 possible to specify conditions on a mathematical theory which
 would suffice for embeddability within $\ZF$ (supplemented by
 additional axioms of infinity if necessary), but which do not
 otherwise restrict the possible constructions in that theory. Of
 course the conditions would apply to $\ZF$ itself, and to other
 possible theories that have been proposed as suitable foundations
 for mathematics (certain theories of categories, etc.), but would
 not restrict us to any particular theory.  This appendix is in
 fact a cry for a Mathematicians' Liberation Movement!

 Among the permissible kinds of construction we should have:
 \begin{enumerate}
 \renewcommand{\theenumi}{\roman{enumi}}  %% Changing the way things look.
 \renewcommand{\labelenumi}{(\theenumi)}
 \item Objects may be created from earlier objects in any
   reasonably constructive fashion.
 \item Equality among the created objects can be any desired
   equivalence relation.
 \end{enumerate}

 In particular, set theory would be such a theory, sets being
 constructed from earlier ones by processes corresponding to the
 usual axioms, and the equality relation being that of having the
 same members.  But we could also, for instance, freely create a
 new object $(x,y)$ and call it the ordered pair of $x$ and $y$.
 We could also create an ordered pair $[x,y]$ different from
 $(x,y)$ but co-existing with it, and neither of these need have
 any relation to the set $\{\{x\},\{x,y\}\}$.  If instead we
 wanted to make $(x,y)$ into an unordered pair, we could define
 equality by means of the equivalence relation $(x,y) = (z,t)$ if
 and only if $x = z$, $y = t$ \textit{or} $x = t$, $y = z$.  I
 hope it is clear that this proposal is not of any particular
 theory as an alternative to $\ZF$ (such as a theory of
 categories, or of the numbers or games considered in this book).
 What is proposed is instead that we give ourselves the freedom to
 create arbitrary mathematical theories of these kinds, but prove
 a metatheorem which ensures once and for all that any such theory
 could be formalised in terms of any of the standard foundational
 theories. The situation is analogous to the theory of vector
 spaces.  Once upon a time these were collections of $n$-tuples of
 numbers, and the interesting theorems were those that remained
 invariant under linear transformations of these numbers.  Now
 even the initial definitions are invariant, and vector spaces are
 defined by axioms rather than as particular objects.  However, it
 is proved that every vector space has a base, so that the new
 theory is much the same as the old.  But now no particular base
 is distinguished, and usually arguments which use particular
 bases are cumbrous and inelegant compared to arguments directly
 in terms of the axioms.

 We believe that mathematics itself can be founded in an invariant
 way, which would be equivalent to, but would not involve,
 formalisation within some theory like $\ZF$.  No particular
 axiomatic theory like $\ZF$ would be needed, and indeed attempts
 to force arbitrary theories into a single formal strait-jacket
 will probably continue to produce unnecessarily cumbrous and
 inelegant contortions.

 For those who doubt the possibility of such a programme, it might
 be worthwhile to note that certainly principles (i) and (ii) of
 our Mathematicians' Lib movement can be expressed directly in
 terms of the predicate calculus without any mention of sets (for
 instance), and it can be shown that any theory satisfying the
 corresponding restrictions can be formalised in $\ZF$ together
 with sufficiently many axioms of infinity.

 Finally, we note that we have adopted the modern habit of
 identifying $\ZF$ (which properly has only sets) with the
 equiconsistent theory $\rm NBG$ (which has proper Classes as
 well) in this appendix and elsewhere.  The classification of
 objects as Big and small is not peculiar to this theory, but
 appears in many foundational theories, and also in our formalised
 versions of principles (i) and (ii).
\end{quotation}

Formalization is probably still quite important to logicians
though.


\appendix
\chapter*{Two}
The following is a famous quote from
Knuth~\cite[Section~3.1]{book:ArtProgram}.
\begin{quotation}
 In a sense there is no such thing as a random number; for
 example, is 2 a random number?
\end{quotation}
I strongly believe that \emph{two is not a random number}.  There
are many ways to see this.  For example, two is the first number
that allows \emph{branching}.  So while unary trees are not so
interesting and ternary trees are too complicated, binary trees
are just good.

More importantly, if one is interested in the relationships
between objects, then one naturally considers \emph{binary}
relations to start with.  We can draw the diagram
\[ {\circ} \stackrel{R}{\longrightarrow} {\circ} \]
to mean the left object is in relation $R$ with the right object.
This leads us to the study of \emph{arrows}, or in technical
terms, \emph{categories}.  Arrows have two ends.  So they have a
special kind of symmetry: one can flip an arrow by changing its
head to its tail, and its tail to its head.  When you do this
\emph{twice}, you get back to where you started.  Such kinds of
symmetries are called \emph{involutions}, and this phenomenon is
known as \emph{duality}.  In conclusion, if mathematics is about
relationships between objects, then \emph{two} must be a very
important number.

If the previous paragraph does not look convincing to you, then
you may find the following well-known theorem easier to take in.

\begin{thm}[Folklore]
 All natural numbers are interesting.
\end{thm}

\begin{proof}
 Suppose not.  Then there is a natural number that is not
 interesting.  Since the natural numbers are well-ordered, we can
 find a smallest such number $n$.  Nothing can be more interesting
 than being the smallest uninteresting natural number.  Therefore,
 $n$ is both interesting and uninteresting, which is a
 contradiction.
\end{proof}


\backmatter
\bibliographystyle{plain}
\bibliography{bibeg}
\end{document}
