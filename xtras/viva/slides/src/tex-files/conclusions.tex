\section{Conclusions}

\subsection{}
%%%%%%%%%%%%%%%%%%%%%%%%%%%%%%%%%%%%%%%%%%%%%%%%%%%%%%%%
{
\begin{frame}{Conclusions}

Modest contributions to knowledge
\begin{itemize}
	\item Measurements of Entropy using RQA appear
	to be robust to real-word data (i.e. different time series
	structures, window length size and levels of smoothness )
	\item 3D surfaces of RQA are independent of either
	the type series or the selection of parameters.
	\item First open access thesis with data and code 
	for its replication. 
\end{itemize}

%\badge{/badge/badge_v00}
\end{frame}
}

\subsection{}
%%%%%%%%%%%%%%%%%%%%%%%%%%%%%%%%%%%%%%%%%%%%%%%%%%%%%%%%
{
%\paper{Xochicale et al. 2018 in Progress}

\begin{frame}{Future Work}

Investigate:
\begin{itemize}
	\item other derivatives of acceleration data
	to have better understanding of the nature of human movement,
	\item other methodologies for state space reconstruction,
	\item the robustness of Entropy measurements with RQA, and 
	\item variability in perception of velocity.
\end{itemize}

Apply the proposed method in the context of human-humanoid interaction to:
\begin{itemize}
	\item evaluate improvement of movement performance,
	\item provide feedback of level of skillfulness, and 
	\item quantify motor control problems and pathologies.
\end{itemize}


\end{frame}
}



