\section{Movement Variability}




\subsection{Why is important to study Human Movement Variability}
%%%%%%%%%%%%%%%%%%%%%%%%%%%%%%%%%%%%%%%%%%%%%%%%%%%%%%%%
{
%\paper{Xochicale 2018 in {\bf PhD thesis}}

\begin{frame}{Why is challenging to investigate Human Movement Variably?}

\Large
Human movement variability involves not only multiple joints and limbs for
a specific task in a determined environment  but also 
external information processed through all of our
available senses and our prior experiences (Xochicale, 2018).

\end{frame}
}



\subsection{}
%%%%%%%%%%%%%%%%%%%%%%%%%%%%%%%%%%%%%%%%%%%%%%%%%%%%%%%%
{
\paper{
Stergiou et al. 2006 in {\bf Neurologic Physical Therapy} 
Stergiou and Decker 2011 in {\bf Human Movement Science}}
\begin{frame}{Modelling Movement Variability}
    \begin{figure}
        \includegraphicscopyright[width=0.9\linewidth]{stergiou2006/drawing}{}
	\caption{Theoretical Model of Optimal Movement Variability} 
   \end{figure}
\end{frame}
}





\subsection{}
%%%%%%%%%%%%%%%%%%%%%%%%%%%%%%%%%%%%%%%%%%%%%%%%%%%%%%%%
{
\begin{frame}{Nonlinear Analyses}

%Entropy measures to quantify regularity and complexity of time series.
%"Is there a best tool to measure variability?" (Caballero et al. 2014, p. 67)

There is no best tool to measure MV and unification of tools is still
an open question (Caballero et al. 2014; Wijnants et al. 2009)
which led me (i) to explore different nonlinear analyses 
to measure MV and (ii) to understand its strengths and weaknesses. 

\begin{itemize}
	\item Approximate Entropy (Pincus 1991, 1995)
	\item Sample Entropy (Richman and Moorman, 2000)
	\item Multiscale Entropy (Costa et al., 2002)
	\item Detrended Fluctation Analysis (Peng et al., 1995)
	\item Largest Lyapunov exponent (Stergiou, 2016)
	\item Recurrence Quantification Analysis (Zbilut and Webber et al., 1992)
\end{itemize}


\end{frame}
}





%
%\subsection{Quantifying Movement Variability}
%%%%%%%%%%%%%%%%%%%%%%%%%%%%%%%%%%%%%%%%%%%%%%%%%%%%%%%%%
%{
%
%\paper{Stergiou 2006, Stergiou and Decker 2011 in {\bf Human Movement Science}}
%
%\begin{frame}{What is Movement Variably?}
%
%%\LARGE
%* MOVEMENT VARIABILITY (MV) is defined as the variations that occur in motor
%performance across multiple repetitions of a task.
%
%%\LARGE
%* Movement Variability is considered as an inherent feature 
%within and between each person's movement.   
%
%\end{frame}
%}
%
%
%

%\subsection{}
%%%%%%%%%%%%%%%%%%%%%%%%%%%%%%%%%%%%%%%%%%%%%%%%%%%%%%%%%
%{
%%\paper{}
%\begin{frame}{Modelling Movement Variability}
%    \begin{figure}
%        \includegraphicscopyright[width=0.8\linewidth]{stergiou2006/fig2A}{}
%%{Work in progress (Xochicale M. et al. 2018)}
%	\caption{MV within and between each person's movement.} 
%   \end{figure}
%\end{frame}
%}
%



%\subsection{}
%%%%%%%%%%%%%%%%%%%%%%%%%%%%%%%%%%%%%%%%%%%%%%%%%%%%%%%%%
%{
%%\paper{
%%Stergiou et al. 2006 in {\bf Neurologic Physical Therapy} 
%%Stergiou and Decker 2011 in {\bf Human Movement Science}}
%\begin{frame}{Modelling Movement Variability}
%    \begin{figure}
%        \includegraphicscopyright[width=0.9\linewidth]{hatze1986/drawing}{}
%	%\caption{Theoretical Model of Optimal Movement Variability} 
%   \end{figure}
%\end{frame}
%}
%
%
%\subsection{}
%%%%%%%%%%%%%%%%%%%%%%%%%%%%%%%%%%%%%%%%%%%%%%%%%%%%%%%%%
%{
%%\paper{
%%Stergiou et al. 2006 in {\bf Neurologic Physical Therapy} 
%%Stergiou and Decker 2011 in {\bf Human Movement Science}}
%\begin{frame}{Modelling Movement Variability}
%    \begin{figure}
%        \includegraphicscopyright[width=0.9\linewidth]{muller-sternad2004/drawing}{}
%	%\caption{Theoretical Model of Optimal Movement Variability} 
%   \end{figure}
%\end{frame}
%}
%
%
%\subsection{}
%%%%%%%%%%%%%%%%%%%%%%%%%%%%%%%%%%%%%%%%%%%%%%%%%%%%%%%%%
%{
%%\paper{
%%Stergiou et al. 2006 in {\bf Neurologic Physical Therapy} 
%%Stergiou and Decker 2011 in {\bf Human Movement Science}}
%\begin{frame}{Modelling Movement Variability}
%    \begin{figure}
%        \includegraphicscopyright[width=0.9\linewidth]{seifert2011/drawing}{}
%	%\caption{Theoretical Model of Optimal Movement Variability} 
%   \end{figure}
%\end{frame}
%}
%
%
%\subsection{}
%%%%%%%%%%%%%%%%%%%%%%%%%%%%%%%%%%%%%%%%%%%%%%%%%%%%%%%%%
%{
%%\paper{
%%Stergiou et al. 2006 in {\bf Neurologic Physical Therapy} 
%%Stergiou and Decker 2011 in {\bf Human Movement Science}}
%\begin{frame}{Modelling Movement Variability}
%    \begin{figure}
%        \includegraphicscopyright[width=0.9\linewidth]{preatoni2010/drawing}{}
%	%\caption{Theoretical Model of Optimal Movement Variability} 
%   \end{figure}
%\end{frame}
%}
%
%




%\subsection{}
%%%%%%%%%%%%%%%%%%%%%%%%%%%%%%%%%%%%%%%%%%%%%%%%%%%%%%%%%
%{
%%\paper{
%%Stergiou et al. 2006 in {\bf Neurologic Physical Therapy} 
%%Stergiou and Decker 2011 in {\bf Human Movement Science}}
%\begin{frame}{MV in Human-Humanoid Interaction}
%    \begin{figure}
%        \includegraphicscopyright[width=0.9\linewidth]{gorer2013/drawing}{}
%	%\caption{Theoretical Model of Optimal Movement Variability} 
%   \end{figure}
%\end{frame}
%}
%


%\subsection{}
%%%%%%%%%%%%%%%%%%%%%%%%%%%%%%%%%%%%%%%%%%%%%%%%%%%%%%%%%
%{
%%\paper{
%%Stergiou et al. 2006 in {\bf Neurologic Physical Therapy} 
%%Stergiou and Decker 2011 in {\bf Human Movement Science}}
%\begin{frame}{MV in Human-Humanoid Interaction}
%    \begin{figure}
%        \includegraphicscopyright[width=0.9\linewidth]{guneysu2014/drawing}{}
%   \end{figure}
%\end{frame}
%}
%

\subsection{}
%%%%%%%%%%%%%%%%%%%%%%%%%%%%%%%%%%%%%%%%%%%%%%%%%%%%%%%%
{
%\paper{
%Stergiou et al. 2006 in {\bf Neurologic Physical Therapy} 
%Stergiou and Decker 2011 in {\bf Human Movement Science}}
\begin{frame}{Movement Variability in Human-Humanoid Interaction}
    \begin{figure}
        \includegraphicscopyright[width=0.9\linewidth]{guneysu2015/drawing}{}
   \end{figure}
\end{frame}
}



%\subsection{}
%%%%%%%%%%%%%%%%%%%%%%%%%%%%%%%%%%%%%%%%%%%%%%%%%%%%%%%%%
%{
%%\paper{
%%Stergiou et al. 2006 in {\bf Neurologic Physical Therapy} 
%%Stergiou and Decker 2011 in {\bf Human Movement Science}}
%\begin{frame}{MV in Human-Robot Interaction}
%    \begin{figure}
%        \includegraphicscopyright[width=0.9\linewidth]{tsuchida2013/drawing}{}
%   \end{figure}
%\end{frame}
%}
%


%
%\subsection{}
%%%%%%%%%%%%%%%%%%%%%%%%%%%%%%%%%%%%%%%%%%%%%%%%%%%%%%%%%
%{
%%\paper{
%%Stergiou et al. 2006 in {\bf Neurologic Physical Therapy} 
%%Stergiou and Decker 2011 in {\bf Human Movement Science}}
%\begin{frame}{MV in Human-Robot Interaction}
%    \begin{figure}
%        \includegraphicscopyright[width=0.9\linewidth]{peng2015/drawing}{}
%   \end{figure}
%\end{frame}
%}
%


%
%\subsection{}
%%%%%%%%%%%%%%%%%%%%%%%%%%%%%%%%%%%%%%%%%%%%%%%%%%%%%%%%%
%{
%\paper{Vaillancourt and Newell 2002 in {\bf Neurobiology of Aging}}
%\begin{frame}{Modelling Movement Variability}
%    \begin{figure}
%        \includegraphicscopyright[width=0.7\linewidth]{vaillancourt2002/fig2A}{}
%	%{(Vaillancourt DE and Newell KM 2002 in {\bf Neurobiology of Aging})}
%	\caption{Theoretical Model of Optimal Movement Variability} 
%   \end{figure}
%\end{frame}
%}
%


\subsection{}
%%%%%%%%%%%%%%%%%%%%%%%%%%%%%%%%%%%%%%%%%%%%%%%%%%%%%%%%
{
\begin{frame}{Research Questions}

\large
\begin{itemize}
	\item What are the effects on nonlinear analyses
	(i.e. RSSs, RPs, and RQA) for 
	different embedding parameters, 
	different recurrence thresholds and 
	different characteristics of time series 
	(i.e. window length size, smoothness and structure)?
	\item What are the weaknesses and strengths of RQA when
	quantifying MV?
	\item How the smoothing of raw time series affects the nonlinear
	analyses when quantifying MV?
\end{itemize}
%\badge{/badge/badge_v00}
\end{frame}
}




