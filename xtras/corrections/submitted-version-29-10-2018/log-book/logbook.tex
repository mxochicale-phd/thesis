\title{Corrections for submitted thesis version}
\author{Miguel Xochicale}
\date{ \today }



\documentclass[12pt]{article}
\usepackage[margin=0.99in]{geometry}
\usepackage{enumitem}

% for asterik in circle
\usepackage{amsmath,amssymb,graphicx,color}
\newcommand{\invcircledast}{%
  \mathbin{\vphantom{\circledast}\text{%
    \ooalign{\smash{\blackcircle}\cr
             \hidewidth\smash{\textcolor{white}{$*$}}\hidewidth\cr
            }%
  }}%
}


\begin{document}
\maketitle


\begin{abstract}
This document presents a log-book for the corrections
of the submitted version.
Minor corrections are mainly for the improvement of the use of  English language
and typos.
Major corrections include further clarificaiton of 
ambigous statements.
%These comments of draft 02 are located in 
%\emph{.../revisions/draft0O2/comments/v2.0/*.pdf}.
\end{abstract}

\tableofcontents
\newpage



%%%%%%%%%%%%%%%%%%%%%%%%%%%%%%%%%%%%%%%%%%%%%%%%%%%%%%%%%%%%%%
%%%%%%%%%%%%%%%%%%%%%%%%%%%%%%%%%%%%%%%%%%%%%%%%%%%%%%%%%%%%%%

\section{Abstract}

\subsection{Minor corrections}
\begin{enumerate}
\item   \_$\wedge$  
	(pp. iii )
	cross-out: smapling rate changes or noisiness
	% template for solutions
	\begin{verbatim}

	\end{verbatim}
	\textit{
	SORTED:
	}
	\\
	
\end{enumerate}


\subsection{Major corrections}
\begin{enumerate}
\item $\circledast_1$ (pp. iii )
	the use of 'no scientific work has been reported'
	sounds a bit pretentious, i guess it might be better
	to say the contributions of the thesis are: 1,2,3.
	which, in a way, allow readers to state the controraty
	% template for solutions
	\begin{verbatim}

	\end{verbatim}
	\textit{
	SORTED:  
	}
	\\
	
\end{enumerate}



%%%%%%%%%%%%%%%%%%%%%%%%%%%%%%%%%%%%%%%%%%%%%%%%%%%%%%%%%%%%%%
%%%%%%%%%%%%%%%%%%%%%%%%%%%%%%%%%%%%%%%%%%%%%%%%%%%%%%%%%%%%%%


\section{Acknowledgements}

\subsection{Minor corrections}
\begin{enumerate}
\item   \_$\wedge$  
	(pp. vii) confereces TO conferences
	\begin{verbatim}
	
	\end{verbatim}
	\textit{
	SORTED:  
	}
	\\
	
\end{enumerate}




%%%%%%%%%%%%%%%%%%%%%%%%%%%%%%%%%%%%%%%%%%%%%%%%%%%%%%%%%%%%%%
%%%%%%%%%%%%%%%%%%%%%%%%%%%%%%%%%%%%%%%%%%%%%%%%%%%%%%%%%%%%%%

\section{Chapter 1}

\subsection{CH1: Minor corrections}
\begin{enumerate}
\item   (pp. 2)  \_$\wedge$  
	or sensors with different
	\begin{verbatim}
	
	\end{verbatim}
	\textit{
	SORTED:  
	}
	\\
	

\item   (pp. 7)  \_$\wedge$  
	differentiate 
	\begin{verbatim}

	\end{verbatim}
	\textit{
	SORTED:  
	}
	\\

\item  (pp. 12)  \_$\wedge$  
	activities
	\begin{verbatim}
	
	\end{verbatim}
	\textit{
	SORTED:  
	}
	\\

\item  (pp. 12)  \_$\wedge$  
	.For
	\begin{verbatim}
	
	\end{verbatim}
	\textit{
	SORTED:  
	}
	\\


\item  (pp. 13)  \_$\wedge$  
	replace 'e.g.' to 'i.e.'
	\begin{verbatim}
	
	\end{verbatim}
	\textit{
	SORTED:  
	}
	\\




\item  (pp. 14)  \_$\wedge$  
	, and
	\begin{verbatim}
	
	\end{verbatim}
	\textit{
	SORTED:  
	}
	\\

\item  (pp. 15)  \_$\wedge$  
	weaknesses 
	\begin{verbatim}
	
	\end{verbatim}
	\textit{
	SORTED:  
	}
	\\

\item  (pp. 15)  \_$\wedge$  
	(i.e.
	\begin{verbatim}
	
	\end{verbatim}
	\textit{
	SORTED:  
	}
	\\


\item  (pp. 16)  \_$\wedge$  
	use the right reference citation format 
	for the wikipedia link
	\begin{verbatim}
	
	\end{verbatim}
	\textit{
	SORTED:  
	}
	\\

\item  (pp. 17)  \_$\wedge$  
	updloaded in ... future 
	\begin{verbatim}
	
	\end{verbatim}
	\textit{
	SORTED:  
	}
	\\

\item  (pp. 17)  \_$\wedge$  
	Scientific Reports
	\begin{verbatim}
	
	\end{verbatim}
	\textit{
	SORTED:  
	}
	\\



\end{enumerate}


\subsection{CH1: Major corrections}
\begin{enumerate}
\item  (pp. 2)  $\circledast_1$ 
	What is that accuracy and precision 
	that Frank et al., 2010 talk about?
	Apparently, these accuacy and precision is about
	 the classification activities 
	and little about the charactersitics of sensors
 	\begin{verbatim}	
	\end{verbatim}
	\textit{
	SORTED:  
	}
	\\

\item (pp. 9) 	$\circledast_1$
	Stating that work has been
	made in MV in HHI for the last six years
	citing 5 works is somewhat shallow!
	Probably more literrature review would help
	to give better understanding of MVinHHI 
	\begin{verbatim}
	
	\end{verbatim}
	\textit{
	SORTED:  
	}
	\\


\item  (pp. 13) $\circledast_1$ 
	underlined text is out of the scope of the phd 
	thesis. reconsider to rewrite the undeline part
	or delete it.	
	\begin{verbatim}
	
	\end{verbatim}
	\textit{
	SORTED:  
	}
	\\


	
\end{enumerate}









\section{Chapter 2}

\subsection{CH2: Minor corrections}
\begin{enumerate}

\item  (pp. 21)  \_$\wedge$  
	deterministic
	\begin{verbatim}
	
	\end{verbatim}
	\textit{
	SORTED:  
	}
	\\

\item  (pp. 22)  \_$\wedge$  
	replace '.' with '. and '
	\begin{verbatim}
	
	\end{verbatim}
	\textit{
	SORTED:  
	}
	\\

\item  (pp. 23)  \_$\wedge$  
	replace 'us' with 'me'
	\begin{verbatim}
	
	\end{verbatim}
	\textit{
	SORTED:  
	}
	\\

\item  (pp. 25)  \_$\wedge$  
	replace 'analysing'  with  'the analyses of'
	Also, rewrite the sentence to explain better the 
	use of Detrended Fluctuantion Analyses (Peng et al. 1995).
	For instance:
	`
	Therefore, considering the previous weaknesses of 
	ApEn, SampEn and MSE, Peng et al., 1995 propose
	Deterended Fluctuation Analyses which is based on ...
	`
	\begin{verbatim}
	
	\end{verbatim}
	\textit{
	SORTED:  
	}
	\\



\item  (pp. 26)  \_$\wedge$  
	replace 'e.g.' with 'i.e.'
	\begin{verbatim}
	
	\end{verbatim}
	\textit{
	SORTED:  
	}
	\\


\item  (pp. 28)  \_$\wedge$  
	add: `Multi Scale Entropy`
	\begin{verbatim}
	
	\end{verbatim}
	\textit{
	SORTED:  
	}
	\\

\item  (pp. 29)  \_$\wedge$  
	add '. However'
	\begin{verbatim}
	
	\end{verbatim}
	\textit{
	SORTED:  
	}
	\\



\item  (pp. 29)  \_$\wedge$  
	replace 'to' with 'the'
	\begin{verbatim}
	
	\end{verbatim}
	\textit{
	SORTED:  
	}
	\\

\item  (pp. 31)  \_$\wedge$  
	rewrite sentence:
	`showed that PeEn is...`
	\begin{verbatim}
	
	\end{verbatim}
	\textit{
	SORTED:  
	}
	\\

\item  (pp. 32)  \_$\wedge$  
	add: 'fundamentals of'
	\begin{verbatim}
	
	\end{verbatim}
	\textit{
	SORTED:  
	}
	\\



\end{enumerate}


\subsection{CH2: Major corrections}
\begin{enumerate}

\item  (pp. 20) $\circledast_1$ 
	I guess it would be better to cite
	other authors to give further references
	to the interested readers for fundamental
	definitions for signal processing in 
	nonlinear dynamics.
	\begin{verbatim}
	
	\end{verbatim}
	\textit{
	SORTED:  
	}
	\\


\item  (pp. 21) $\circledast_1$ 
	I am wondering if citing two authors is a bit shallow
	to make conclusions about the determistic chaotic 
	characteristics of signals.
	\begin{verbatim}
	
	\end{verbatim}
	\textit{
	SORTED:  
	}
	\\

\item  (pp. 22) $\circledast_1$ 
	replace ',' wiht '.' and start a new stence. Maybe with:
	`The challenge is to find tools to quantify the 
	subtle changes ... 	`
	\begin{verbatim}
	
	\end{verbatim}
	\textit{
	SORTED:  
	}
	\\

\item  (pp. 23) $\circledast_1$ 
	Have a better understanding of villancourt and newell 
	statemetn about the model of optimal varialibtlu. 
	Maybe add the fig of Vaillancourt
	\begin{verbatim}
	
	\end{verbatim}
	\textit{
	SORTED:  
	}
	\\

\item  (pp. 24) $\circledast_1$ 
	why $ln(0)$ is a problem when compuyting ApEn (Richman and Moorman, 2000)?
	it might be worthwhile to revise the use of log in
	probability
	\begin{verbatim}
	
	\end{verbatim}
	\textit{
	SORTED:  
	}
	\\


\item  (pp. 25) $\circledast_1$ 
	What is a `coarse-grainded` time series?
	Maybe read Costa et al. (2002) to understand more!
	\begin{verbatim}
	
	\end{verbatim}
	\textit{
	SORTED:  
	}
	\\


\item  (pp. 26) $\circledast_1$ 
	Have a better understanding of the results
	of Wijnants et al. 2009.
	\begin{verbatim}
	
	\end{verbatim}
	\textit{
	SORTED:  
	}
	\\

\item  (pp. 28) $\circledast_1$ 
	Stating that EMD is still an open problem,
	it is a very shallow statement without doing 
	giving further evidence! 
	Hence, it is suggested to provide further 
	evindence
	\begin{verbatim}
	
	\end{verbatim}
	\textit{
	SORTED:  
	}
	\\

\item  (pp. 29) $\circledast_1$ 
	sententece is required to be completed
	\begin{verbatim}
	
	\end{verbatim}
	\textit{
	SORTED:  
	}
	\\


\item  (pp. 30) $\circledast_1$ 
	Give further explanation of signal-to-noise charactesirtis,
	and what does it mean that the ration is substantially lower?
	\begin{verbatim}
	
	\end{verbatim}
	\textit{
	SORTED:  
	}
	\\

	
\end{enumerate}



\section{Chapter 3}

\subsection{CH3: Minor corrections}

\begin{enumerate}

\item  (pp. 36)  \_$\wedge$  
	erase 'the' 
	\begin{verbatim}
	
	\end{verbatim}
	\textit{
	SORTED:  
	}
	\\

\item  (pp. 36)  \_$\wedge$  
	add 'space'
	\begin{verbatim}
	
	\end{verbatim}
	\textit{
	SORTED:  
	}
	\\

\item  (pp. 37)  \_$\wedge$  
	replace 'we review' with 'i study'
	\begin{verbatim}
	
	\end{verbatim}
	\textit{
	SORTED:  
	}
	\\

\item  (pp. 38)  \_$\wedge$  
	add 'from'
	\begin{verbatim}
	
	\end{verbatim}
	\textit{
	SORTED:  
	}
	\\


\item  (pp. 39)  \_$\wedge$  
	replace 'the' for 'uniform'
	\begin{verbatim}
	
	\end{verbatim}
	\textit{
	SORTED:  
	}
	\\

\item  (pp. 40)  \_$\wedge$  
	add commmas
	\begin{verbatim}
	
	\end{verbatim}
	\textit{
	SORTED:  
	}
	\\

\item  (pp. 41)  \_$\wedge$  
	increase the font size for the threshold
	\begin{verbatim}
	
	\end{verbatim}
	\textit{
	SORTED:  
	}
	\\

\item  (pp. 41)  \_$\wedge$  
	add 'would experience the following'
	add 'uniform'
	Make sure that the sentence is rewritten and
	that make sense
	\begin{verbatim}
	
	\end{verbatim}
	\textit{
	SORTED:  
	}
	\\


\item  (pp. 42)  \_$\wedge$  
	add 'comma'
	\begin{verbatim}
	
	\end{verbatim}
	\textit{
	SORTED:  
	}
	\\


\item  (pp. 42)  \_$\wedge$  
	Fix reference!
	\begin{verbatim}
	
	\end{verbatim}
	\textit{
	SORTED:  
	}
	\\

\item  (pp. 42)  \_$\wedge$  
	add ', for this thesis'
	\begin{verbatim}
	
	\end{verbatim}
	\textit{
	SORTED:  
	}
	\\

\item  (pp. 42)  \_$\wedge$  
	add 'where, p'
	\begin{verbatim}
	
	\end{verbatim}
	\textit{
	SORTED:  
	}
	\\


\item  (pp. 43)  \_$\wedge$  
	modify the subindex in the equation $p_{(i,j)}$
	\begin{verbatim}
	
	\end{verbatim}
	\textit{
	SORTED:  
	}
	\\

\item  (pp. 44)  \_$\wedge$  
	add 'Principal Component Analysis'
	\begin{verbatim}
	
	\end{verbatim}
	\textit{
	SORTED:  
	}
	\\

\item  (pp. 45)  \_$\wedge$  
	replace 'and' with 'where'
	\begin{verbatim}
	
	\end{verbatim}
	\textit{
	SORTED:  
	}
	\\

\item  (pp. 46)  \_$\wedge$  
	delete underline line
	\begin{verbatim}
	
	\end{verbatim}
	\textit{
	SORTED:  
	}
	\\

\item  (pp. 46)  \_$\wedge$  
	add: 'for'
	\begin{verbatim}
	
	\end{verbatim}
	\textit{
	SORTED:  
	}
	\\

\item  (pp. 49)  \_$\wedge$  
	add: 'of'
	\begin{verbatim}
	
	\end{verbatim}
	\textit{
	SORTED:  
	}
	\\

\item  (pp. 49)  \_$\wedge$  
	amend sentence: 
	diagonal lines, for chaotic signals shorter 
	diagonla lines,
	or for stochastic signals absent diagonal lines
	\begin{verbatim}
	
	\end{verbatim}
	\textit{
	SORTED:  
	}
	\\


\item  (pp. 51)  \_$\wedge$  
	delete: 'has different sampling rate'
	\begin{verbatim}
	
	\end{verbatim}
	\textit{
	SORTED:  
	}
	\\

\item  (pp. 51)  \_$\wedge$ 
	delete: 'therefore' 
	add: '(RSS, RP, RQA)'
	\begin{verbatim}
	
	\end{verbatim}
	\textit{
	SORTED:  
	}
	\\

\item  (pp. 52)  \_$\wedge$  
	replace: 'in' with 'with'
	\begin{verbatim}
	
	\end{verbatim}
	\textit{
	SORTED:  
	}
	\\

\item  (pp. 52)  \_$\wedge$  
	replace: 'our' with 'remain'
	\begin{verbatim}
	
	\end{verbatim}
	\textit{
	SORTED:  
	}
	\\



\end{enumerate}




\subsection{CH3: Major corrections}
\begin{enumerate}

\item  (pp. 35) $\circledast_1$ 
	Give further explanation of box-counting
	(it is explained in 11.3.1 kantz and schreiber 2003)
	\begin{verbatim}
	
	\end{verbatim}
	\textit{
	SORTED:  
	}
	\\

\item  (pp. 39) $\circledast_1$ 
	Mathematical notation of FNN
	does not agree with descriptoins 
	in Eqs in pages 34 and 34.
	Check the absolute values.
	Revise CAO1997 references
	Something like this:
	$X_{\tau, i}^{m+1} - X_{\tau, i(i,m)}^{m+1}  $

	\begin{verbatim}
	
	\end{verbatim}
	\textit{
	SORTED:  
	}
	\\

\item  (pp. 40) $\circledast_1$ 
	Check the notation of E2 and the absolute values.
	
	\begin{verbatim}
	
	\end{verbatim}
	\textit{
	SORTED:  
	}
	\\

\item  (pp. 41) $\circledast_1$ 
	Why 0.05 is the right threshold value?
	Add extensive experiments with other chaotic
	time series in the Appendix to provide
	evidence for the selection of 0.05
	\begin{verbatim}
	
	\end{verbatim}
	\textit{
	SORTED:  
	}
	\\


\item  (pp. 42) $\circledast_1$ 
	Review Average Mutual Information equation.
	How is the histogram is computed and what
	is the relation with probablity and log values?
	\begin{verbatim}
	
	\end{verbatim}
	\textit{
	SORTED:  
	}
	\\


\item  (pp. 44) $\circledast_1$ 
	Why is the reason to choose sample mean operation
	for the overall value of minimum embedding values?
	\begin{verbatim}
	
	\end{verbatim}
	\textit{
	SORTED:  
	}
	\\


\item  (pp. 47) $\circledast_1$ 
	Why there is no examples of Recurrence Plots
	for texture of small-scale patters?
	\begin{verbatim}
	
	\end{verbatim}
	\textit{
	SORTED:  
	}
	\\


\item  (pp. 48) $\circledast_1$ 
	how the percentage of recurrence is computed?
	and how the effect of the time series lenght affects 
	such percentage
	\begin{verbatim}
	
	\end{verbatim}
	\textit{
	SORTED:  
	}
	\\



\item  (pp. 49) $\circledast_1$ 
	What is $dmin$?
	Verify formulas with R code functions!
	\begin{verbatim}
	
	\end{verbatim}
	\textit{
	SORTED:  
	}
	\\




\item  (pp. 50-51 ) $\circledast_1$ 
	Strength and weaknesses of RP and RQA
	are not well balance. It might be rewritten!
	\begin{verbatim}
	
	\end{verbatim}
	\textit{
	SORTED:  
	}
	\\







\end{enumerate}





\section{Chapter 4}

\subsection{CH4: Minor corrections}
\begin{enumerate}

\item  (pp. 53)  \_$\wedge_1$  
	replace: 'We design two experiments' with
	'Two experiments were designed for this thesis: one'
	\begin{verbatim}
	
	\end{verbatim}
	\textit{
	SORTED:  
	}
	\\

\item  (pp. 53)  \_$\wedge_2$  
	add: 'the other for '
	\begin{verbatim}
	
	\end{verbatim}
	\textit{
	SORTED:  
	}
	\\


\item  (pp. 53)  \_$\wedge$  
	replace 'aims'  with 'aim'
	\begin{verbatim}
	
	\end{verbatim}
	\textit{
	SORTED:  
	}
	\\

\item  (pp. 53)  \_$\wedge$  
	replace: 'affect' with
	'would be affected by'
	\begin{verbatim}
	
	\end{verbatim}
	\textit{
	SORTED:  
	}
	\\

\item  (pp. 53)  \_$\wedge$  
	add: 'repetitions'
	\begin{verbatim}
	
	\end{verbatim}
	\textit{
	SORTED:  
	}
	\\

\item  (pp. 58)  \_$\wedge$  
	add: 'from'
	\begin{verbatim}
	
	\end{verbatim}
	\textit{
	SORTED:  
	}
	\\

\item  (pp. 59)  \_$\wedge$  
	add: ','
	\begin{verbatim}
	
	\end{verbatim}
	\textit{
	SORTED:  
	}
	\\


\item  (pp. 62)  \_$\wedge$  
	add: 'Time'
	\begin{verbatim}
	
	\end{verbatim}
	\textit{
	SORTED:  
	}
	\\




\end{enumerate}


\subsection{CH4: Major corrections}
\begin{enumerate}

\item  (pp. 55) $\circledast_1$ 
	Find and add code for NAO's arm movements
	\begin{verbatim}
	
	\end{verbatim}
	\textit{
	SORTED:  
	}
	\\


\item  (pp. 60) $\circledast_1$ 
	Rewrite underline section and
	add more evidence for magnetic 
	disturbances for inertial sensors
	\begin{verbatim}
	
	\end{verbatim}
	\textit{
	SORTED:  
	}
	\\


\end{enumerate}







\section{Chapter 5}

\subsection{CH5: Minor corrections}
\begin{enumerate}

\item  (pp. 71)  \_$\wedge$  
	add: 'listeing'
	\begin{verbatim}
	
	\end{verbatim}
	\textit{
	SORTED:  
	}
	\\

\item  (pp. 71)  \_$\wedge$  
	add: 
	however sg2zmuvGyroY appear to be slightly
	affected by the smoothness as it shows
	similarity to the trajectories from
	sg0 and sg1.
	\begin{verbatim}
	
	\end{verbatim}
	\textit{
	SORTED:  
	}
	\\


\item  (pp. 72,73,74,75)  \_$\wedge$  
	in figs 5.5 to 5.8
		add:
	'RSS for horizontal/vertical arm movements (with/no beat)' 
	for participant p04
	\begin{verbatim}
	
	\end{verbatim}
	\textit{
	SORTED:  
	}
	\\


\item  (pp. 76)  \_$\wedge$  
	replace: 'it' '-'
	\begin{verbatim}
	
	\end{verbatim}
	\textit{
	SORTED:  
	}
	\\

\item  (pp. 76)  \_$\wedge$  
	revise that p01 is p04
	\begin{verbatim}
	
	\end{verbatim}
	\textit{
	SORTED:  
	}
	\\

\item  (pp. 76)  \_$\wedge$  
	delete: 'the'	
	\begin{verbatim}
	
	\end{verbatim}
	\textit{
	SORTED:  
	}
	\\


\item  (pp. 77)  \_$\wedge$  
	replace: 'once' with 'one'	
	\begin{verbatim}
	
	\end{verbatim}
	\textit{
	SORTED:  
	}
	\\

\item  (pp. 82)  \_$\wedge$  
	delete an space
	\begin{verbatim}
	
	\end{verbatim}
	\textit{
	SORTED:  
	}
	\\

\item  (pp. 83)  \_$\wedge$  
	replace: 'figs' with 'Figs'
	\begin{verbatim}
	
	\end{verbatim}
	\textit{
	SORTED:  
	}
	\\

\item  (pp. 86)  \_$\wedge_1$  
	add: 'over $z-$axis'
	\begin{verbatim}
	
	\end{verbatim}
	\textit{
	SORTED:  
	}
	\\


\item  (pp. 86)  \_$\wedge_2$  	
	add: 'over $x-$axis'
	\begin{verbatim}
	
	\end{verbatim}
	\textit{
	SORTED:  
	}
	\\

\item  (pp. 86)  \_$\wedge_3$  
	add: 'over $y-$axis'
	\begin{verbatim}
	
	\end{verbatim}
	\textit{
	SORTED:  
	}
	\\

\item  (pp. 86)  \_$\wedge_4$  
	add: '3D'
	\begin{verbatim}
	
	\end{verbatim}
	\textit{
	SORTED:  
	}
	\\


\item  (pp. 87)  \_$\wedge$  
	add in Fig descriptions for differnt axis
	'over $x-$axis'
	\begin{verbatim}
	
	\end{verbatim}
	\textit{
	SORTED:  
	}
	\\

\item  (pp. 88)  \_$\wedge$  
	add: 'a'
	\begin{verbatim}
	
	\end{verbatim}
	\textit{
	SORTED:  
	}
	\\

\item  (pp. 88)  \_$\wedge$  
	add: 'shown'
	\begin{verbatim}
	
	\end{verbatim}
	\textit{
	SORTED:  
	}
	\\

\item  (pp. 89)  \_$\wedge$  
	delete 's'
	\begin{verbatim}
	
	\end{verbatim}
	\textit{
	SORTED:  
	}
	\\

\item  (pp. 90,91,92,93)  \_$\wedge$  
	Ammend the caption of figures 
	adding the over axis $x-axis$	
	that is used in Fig 5.15
	\begin{verbatim}
	
	\end{verbatim}
	\textit{
	SORTED:  
	}
	\\


\item  (pp. 94)  \_$\wedge$  
	add: 'of the time series'
	\begin{verbatim}
	
	\end{verbatim}
	\textit{
	SORTED:  
	}
	\\

\item  (pp. 94)  \_$\wedge$  
	replace: 'C' with 'D'
	\begin{verbatim}
	
	\end{verbatim}
	\textit{
	SORTED:  
	}
	\\

\item  (pp. 97)  \_$\wedge$  
	add: 'can produce'
	\begin{verbatim}
	
	\end{verbatim}
	\textit{
	SORTED:  
	}
	\\

\item  (pp. 97)  \_$\wedge$  
	add: 'appears to be'
	\begin{verbatim}
	
	\end{verbatim}
	\textit{
	SORTED:  
	}
	\\



\item  (pp. 95,96,98)  \_$\wedge$  
	Ammend the caption of figures 
	adding the over axis $x-axis$	
	that is used in Fig 5.15
	\begin{verbatim}
	
	\end{verbatim}
	\textit{
	SORTED:  
	}
	\\


\end{enumerate}


\subsection{CH5: Major corrections}
\begin{enumerate}

\item  (pp. 70 ) $\circledast_1$ 	
	* Is there any scientific reason to choose only
	two parameters that represent all time series?	
	* Does the data is losing the richness of 
	variation of all embedded parameters when only 
	using two parameters.
	\begin{verbatim}
	
	\end{verbatim}
	\textit{
	SORTED:  
	}
	\\

\item  (pp. 70 ) $\circledast_2$ 
	HUMAN MISTAKE:
	$m_0$ should be 6 and $\tau_0$ 9.
	However, of such error, RQA variations
	for different embedding paramters helps
	us also to understand how the surfaces
	change with these variation of embedding 
	parameters!
	\begin{verbatim}
	
	\end{verbatim}
	\textit{
	SORTED:  
	}
	\\

\item  (pp. 70 ) $\circledast_3$ 
	Add description to describe the reason
	of why it is only presented one participant
	and also verify that p04 was the one suggested
	in fig1 used in figs 5.5. to 5.8	
	\begin{verbatim}
	
	\end{verbatim}
	\textit{
	SORTED:  
	}
	\\




\item  (pp. 70 ) $\circledast_4$ 
	* What do I mean by little?
	* Is there any way to quantify the changes in the RSSs 
	trajectories. EXPLAIN MORE!	
	\begin{verbatim}
	
	\end{verbatim}
	\textit{
	SORTED:  
	}
	\\


\item  (pp. 86) $\circledast_1$ 
	* Why do I choose those range of values?
	* What was the criterium to choose those values
	Maybe add an appendix with more experiments for values
	or add an explanation of the method in chapter 3
	\begin{verbatim}
	
	\end{verbatim}
	\textit{
	SORTED:  
	}
	\\


\item  (pp. 87) $\circledast_1$ 
	Add paragraph to describe what is happening in the next
	sections.
	\begin{verbatim}
	
	\end{verbatim}
	\textit{
	SORTED:  
	}
	\\

\item  (pp. 87) $\circledast_2$ 
	increase the size of the font
	\begin{verbatim}
	
	\end{verbatim}
	\textit{
	SORTED:  
	}
	\\


\item  (pp. 88) $\circledast_1$ 
	What is the reason for those
	decreases of DET at those $\epsilon$ values?
	\begin{verbatim}
	
	\end{verbatim}
	\textit{
	SORTED:  
	}
	\\

\item  (pp. 94) $\circledast_1$ 
	What do I mean by the increase of smoothness
	in the 3D surface?
	\begin{verbatim}
	
	\end{verbatim}
	\textit{
	SORTED:  
	}
	\\





\end{enumerate}




\section{Chapter 6}

\subsection{Minor corrections}
\begin{enumerate}

\item  (pp. 100)  \_$\wedge$  
	change D by $E1$
	\begin{verbatim}
	
	\end{verbatim}
	\textit{
	SORTED:  
	}
	\\

\item  (pp. 101)  \_$\wedge$  
	add: Similarly
	\begin{verbatim}
	
	\end{verbatim}
	\textit{
	SORTED:  
	}
	\\

\item  (pp. 101)  \_$\wedge$  
	add: Box plots of
	\begin{verbatim}
	
	\end{verbatim}
	\textit{
	SORTED:  
	}
	\\

\item  (pp. 102)  \_$\wedge$  
	add: minimum
	\begin{verbatim}
	
	\end{verbatim}
	\textit{
	SORTED:  
	}
	\\

\item  (pp. 103)  \_$\wedge$  
	add: overall
	\begin{verbatim}
	
	\end{verbatim}
	\textit{
	SORTED:  
	}
	\\

\item  (pp. 104)  \_$\wedge$  
	add paragraph for more of RSS
	in the appendicx E.3
	\begin{verbatim}
	
	\end{verbatim}
	\textit{
	SORTED:  
	}
	\\

\item  (pp. 107)  \_$\wedge$  
	change: D.4 to E.4
	\begin{verbatim}
	
	\end{verbatim}
	\textit{
	SORTED:  
	}
	\\



\item  (pp. 108,109)  \_$\wedge$  
	change and by for
	\begin{verbatim}
	
	\end{verbatim}
	\textit{
	SORTED:  
	}
	\\

\item  (pp. 113)  \_$\wedge$  
	over the $x-$axis; over the $y-$axis
	\begin{verbatim}
	
	\end{verbatim}
	\textit{
	SORTED:  
	}
	\\

\item  (pp. 113)  \_$\wedge$  
	delete dot
	\begin{verbatim}
	
	\end{verbatim}
	\textit{
	SORTED:  
	}
	\\


\item  (pp. 114)  \_$\wedge$  
	over the $x-$axis; over the $y-$axis
	\begin{verbatim}
	
	\end{verbatim}
	\textit{
	SORTED:  
	}
	\\


\item  (pp. 122)  \_$\wedge$  
	add: add; in;
	more robust thant the other RQA metrics
	\begin{verbatim}
	
	\end{verbatim}
	\textit{
	SORTED:  
	}
	\\





\end{enumerate}


\subsection{Major corrections}
\begin{enumerate}

\item  (pp. 100) $\circledast_1$ 
	Why only two filter lenghts were considered for 
	the Savitzky-Golay filter?
	Why those particular values?	
	\begin{verbatim}
	
	\end{verbatim}
	\textit{
	SORTED:  
	}
	\\

\item  (pp. 104) $\circledast_1$ 
	Explanation of why I use RQA?
	which should be introduced in ch5 pp.71
	\begin{verbatim}
	
	\end{verbatim}
	\textit{
	SORTED:  
	}
	\\

\item  (pp. 107) $\circledast_1$ 
	Statement on the subjectivity of a person
	that observe changes in RPs should be
	also added in p77
	\begin{verbatim}
	
	\end{verbatim}
	\textit{
	SORTED:  
	}
	\\

\item  (pp. 113) $\circledast_1$ 
	what do I mean by cascade effect in 
	the 3D surface plot?
	\begin{verbatim}
	
	\end{verbatim}
	\textit{
	SORTED:  
	}
	\\


\item  (pp. 113) $\circledast_2$ 
	rewrite this part, as I am not really
	satisfied with the way it reads the description
	of 6.9D
	\begin{verbatim}
	
	\end{verbatim}
	\textit{
	SORTED:  
	}
	\\

\item  (pp. 114) $\circledast_1$ 
	increase the font size of RQA legend numbers!
	\begin{verbatim}
	
	\end{verbatim}
	\textit{
	SORTED:  
	}
	\\


\item  (pp. 115) $\circledast_0$ 
	add an explanation to introduce 
	sections 6.7.1 to 6.7.4
	\begin{verbatim}
	
	\end{verbatim}
	\textit{
	SORTED:  
	}
	\\



\item  (pp. 115) $\circledast_1$ 
	add further statements about the 
	patterns seen in other 3D surfaces and then
	point out what one can see in Fig 6.12
	\begin{verbatim}
	
	\end{verbatim}
	\textit{
	SORTED:  
	}
	\\















\end{enumerate}






\section{Chapter 7}

\subsection{Minor corrections}
\begin{enumerate}

\item  (pp. 123)  \_$\wedge$  
	replace: we with I;I; add:d
	\begin{verbatim}
	
	\end{verbatim}
	\textit{
	SORTED:  
	}
	\\

\item  (pp. 124)  \_$\wedge$  
	add: The approach of 3D surface of RQA
	\begin{verbatim}
	
	\end{verbatim}
	\textit{
	SORTED:  
	}
	\\

\item  (pp. 124)  \_$\wedge$  
	add: (Computation of $m$, $\tau$, $\epsilon$)
	\begin{verbatim}
	
	\end{verbatim}
	\textit{
	SORTED:  
	}
	\\




\item  (pp. 124)  \_$\wedge$  
	replace: we with I;I;I
	\begin{verbatim}
	
	\end{verbatim}
	\textit{
	SORTED:  
	}
	\\



\item  (pp. 125)  \_$\wedge$  
	replace: we with I;I;I;I
	\begin{verbatim}
	
	\end{verbatim}
	\textit{
	SORTED:  
	}
	\\

\item  (pp. 125)  \_$\wedge$  
	add: also; additionally
	\begin{verbatim}
	
	\end{verbatim}
	\textit{
	SORTED:  
	}
	\\


\item  (pp. 126)  \_$\wedge$  
	replace: we with I
	\begin{verbatim}
	
	\end{verbatim}
	\textit{
	SORTED:  
	}
	\\

\item  (pp. 126)  \_$\wedge$  
	add: open
	\begin{verbatim}
	
	\end{verbatim}
	\textit{
	SORTED:  
	}
	\\


\item  (pp. 127)  \_$\wedge$  
	replace: out with the
	\begin{verbatim}
	
	\end{verbatim}
	\textit{
	SORTED:  
	}
	\\


\item  (pp. 128)  \_$\wedge$  
	add: ()
	\begin{verbatim}
	
	\end{verbatim}
	\textit{
	SORTED:  
	}
	\\

\item  (pp. 129)  \_$\wedge$  
	replace: we with I;I;I; some with while
	\begin{verbatim}
	
	\end{verbatim}
	\textit{
	SORTED:  
	}
	\\

\item  (pp. 129)  \_$\wedge$  
	add: the work of this thesis can be applied
	\begin{verbatim}
	
	\end{verbatim}
	\textit{
	SORTED:  
	}
	\\

\item  (pp. 129)  \_$\wedge$  
	add: ,
	\begin{verbatim}
	
	\end{verbatim}
	\textit{
	SORTED:  
	}
	\\


\item  (pp. 130)  \_$\wedge$  
	replace: and therefore with:
	Hence the proposed 3D surfaces of RQAEntr can provide
	adecuate to quantify MV and to provide 
	feedback in the HHI activity.

	\begin{verbatim}
	
	\end{verbatim}
	\textit{
	SORTED:  
	}
	\\







\end{enumerate}


\subsection{Major corrections}
\begin{enumerate}

\item  (pp. 125) $\circledast_1$ 
	What do I mean by erratic changes?
	\begin{verbatim}
	
	\end{verbatim}
	\textit{
	SORTED:  
	}
	\\

\item  (pp. ) $\circledast_1$ 
	
	\begin{verbatim}
	
	\end{verbatim}
	\textit{
	SORTED:  
	}
	\\


\end{enumerate}

\section{APPENDIX}

\subsection{Minor corrections}
\begin{enumerate}

\item  (pp. 142)  \_$\wedge$  
	change:	typo to type
	\begin{verbatim}
	
	\end{verbatim}
	\textit{
	SORTED:  
	}
	\\

\item  (pp. 143)  \_$\wedge$  
	add: shows
	\begin{verbatim}
	
	\end{verbatim}
	\textit{
	SORTED:  
	}
	\\



\end{enumerate}


\subsection{Major corrections}
\begin{enumerate}

\item  (pp. ) $\circledast_1$ 
	
	\begin{verbatim}
	
	\end{verbatim}
	\textit{
	SORTED:  
	}
	\\


\end{enumerate}








\end{document}

