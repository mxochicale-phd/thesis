\title{Corrections for thesis from draft v2.0 to draft v2.75 (pre-submission)}
\author{Miguel Xochicale}
\date{ \today }

\documentclass[10pt]{article}
\usepackage[margin=0.99in]{geometry}
\usepackage{enumitem}

\begin{document}
\maketitle



\begin{abstract}
This document presents a log book for the corrections 
of the draft v2.0 handed on 20 Sep 2018 to Chris Baber (CB).
Hand written comments of thesis draft v2.0 of were handed in on 
11 Oct 2018. 
These comments of draft 02 are located in 
\emph{.../revisions/draft0O2/comments/v2.0/*.pdf}.
%\\
%\\
Then all changes for draft v2.75 by Miguel Xochicale (MX)
are located in \emph{~/phd-thesis/draftrevisions/draft02/draft/v2.75/*.pdf}.
\end{abstract}

\tableofcontents
\newpage


%

\section{Major corrections}


\subsection{Cover}


\begin{enumerate}

\item How well are the research questions shown to be:
	(a) necessary and 
	(b) novel?

\item What argument for the application of the approaches to H-H-I?


\end{enumerate}



\subsection{Chapter 1}

\begin{enumerate}

\item  (pp 8) Can you say how these measures compare? \\

	(pp. 8)Are entropy measures as good as their models?


\item (pp. 11) What does this tell us?

\begin{verbatim}

However, such statistical differences cannot capture the structure of the 
time series from each of the participants which performed the movements
at different frequencies and therefore with different data length
(see Fig. 10 \cite{guneysu2015} for further details).

\end{verbatim}
\textit{
SORTED: 
Sat 13 Oct 11:34:10 BST 2018
}
\\




\item (pp. 15) Add descriptions of contributions to the thesis.

\begin{verbatim}

Partial work of this thesis has been presented in the following peer-reviewed 
conferences. Author contributions for the publications of 
Miguel Xochicale (MX), Chris Baber (CB) and Mourad Oussalah (MO) are as follow:
Conceptualisation: MX, CB, MO;
Data Curation: MX;
Formal Analysis: MX;
Funding Acquisition: MX, CB;
Investigation: MX;
Methodology: MX;
Project Administration: MX;
Resources: CB;
Software: MX;
Supervision: CB;
Validation: MX;
Verification: MX;
Writing - Original Draft Preparation: MX;
Writing - Review: CB, MO; and 
Writing - Editing: MX.


\end{verbatim}
\textit{
SORTED: 
Sun 14 Oct 13:06:43 BST 2018
}
\\




\end{enumerate}


\subsection{Chapter 2}

\begin{enumerate}

\item (pp. 21) Expand deterministic chaotic time series!

\item (pp. 24) So ... what is the answer to the 'What to measure' 
	questions?


\begin{verbatim}

Therefore, considering the works 
of \cite{vaillancourt2002, vaillancourt2003} and \cite{stergiou2006},
one can quantify movement variablity 
based on the complexity and predictability of human movement.

\end{verbatim}
\textit{
SORTED: 
Mon 15 Oct 09:00:52 BST 2018
}
\\





\item (pp. 29) how big?

\begin{verbatim}

(up to the scale of $6 \times 10^3$ data points) to ensure 

\end{verbatim}
\textit{
SORTED: 
Mon 15 Oct 10:56:56 BST 2018
}
\\





\item (pp. 31) The chapter just stops... you should add a 'conclusions'
paragraph that states the main ideas of that are to be used in the thesis?
In particular you could add:
why defining embedding parameters is a challenged.



\begin{verbatim}

\section{Conclusions}
Having reviewed works regarding the questions of: (i) what to quantify in 
Movement Variability and (ii) which nonlinear tools are appropriate to quantify 
MV and the strengths and weaknesses of nonlinear anayses with 
real-world data, it can then be concluded that little research has been done 
on the effects with Reconstructed State Spaces (RSSs), Recurrent Plots (RPs), 
and Recurrence Quantification Analysis (RQA) metrics for different 
embedding parameters, different recurrence thresholds and different 
characteristics of time series (window length size, smoothness and structure).
Hence, nonlinear analyses such as estimation of embedding parameters, 
RSSs, RPs, and RQAs are reviewed in the following chapter.

\end{verbatim}
\textit{
SORTED: 
Mon 15 Oct 12:31:17 BST 2018
}
\\








\end{enumerate}



\subsection{Chapter 3}

\begin{enumerate}

\item (pp. 43) This could be explained a little more clearly.
	Are you suggesting to just read off figures 3.3A and 3.4A,
	or there is some additional calculation?


\begin{verbatim}

Little has been changed but the distillation of the paragraphs

\end{verbatim}
\textit{
SORTED: 
Mon 15 Oct 14:39:42 BST 2018
}
\\



\item (pp. 45) perhaps show the final results of this process?


\begin{verbatim}

See Fig. \ref{fig:ssr} that illustrates and describes the method of 
reconstructed state space with UTDE.

\end{verbatim}
\textit{
SORTED: 
Mon 15 Oct 14:34:49 BST 2018
}
\\





\item (pp. 53) Add conclusions for the chapter and connect them to 
	the next one


\begin{verbatim}

\section{Conclusions}
In this chapter, we introduced fundamentals of nonlinear analyses such 
as RSS with UTDE, estimation of embedding parameters with FNN and AMI, RP, 
and RQA. Hence, this thesis is only focused on traditional methods 
(FNN and AMI) to compute embedding parameters 
as this is still an open challenge \citep{uzal2011, gomezgarcia2014}. 
However, from the computation of RSSs and RPs in our experiments
(Chapter \ref{chapter4}), facing the real-world time series data issues,
we realise about the importance of the increment not only embedding 
parameters \citep{iwanski1998} 
but also recurrence thresholds and its effects of 3D surfaces of RQA metrics
(Chapters \ref{chapter5} and \ref{chapter6}).

\end{verbatim}
\textit{
SORTED: 
Mon 15 Oct 16:57:14 BST 2018
}
\\





\end{enumerate}




\subsection{Chapter 4}

\begin{enumerate}

\item (pp. 1) Method and experiments in this chapter should have all
	the detail that someone needs to do exactly the same experiment.
	It should be like a cookery book that gives a recipe for 
	making something. 
	Is everythign in this chapter?



\begin{verbatim}

Generally, sentences of mchapter 4 were distilled!

\end{verbatim}
\textit{
SORTED: 
Mon 15 Oct 18:54:05 BST 2018
}
\\



\item (pp. 62) is this a real statistic?


\begin{verbatim}

Using a low-pass filter is the common way to either capture the low 
frequencies (below 15 Hz) that represent \%99 of the human body 
energy or to get the gravitational and body motion components of 
accelerations (below 0.3 Hz) \citep{anguita2013}.

\end{verbatim}
\textit{
SORTED: 
Mon 15 Oct 17:30:04 BST 2018
}
\\





\end{enumerate}


\subsection{Chapter 6}



\begin{verbatim}

Figures were changed to be more impact and well embedded to
be read better.
\end{verbatim}
\textit{
SORTED: 
}
\\


ALL THE FOLLOWING QUESTIONS WERE TACKLE 
IN CHAPTER 5 and 6 
FOR THE FINAL VERSION OF THE THESIS. 

\begin{enumerate}




\item (pp. 69) What values does this results in ?

\item (pp. 72) need to provide numbers here to support your claim!

\item (pp. 75) We need a more robust terminology to represente these
	differences; you can't assume that your interpretation of
	'smooth' is universal. So, an option is to define smooth as
	variation of $\pm x$ ...

\item (pp. 75) OK -- what you are suggesting is that eye-balling 
	the figures (as you have done ) is not objective, 
	so you need RQA ... but I think this argument could introduce
	this.

\item (pp. 76) so, this is what 'meaningful' means for this thesis?

\item (pp. 78) What is being 'quantified' in RQA --
	what numbers? could you, for example, report slope or
	max spread of a line? or use eigen values for the matrix!
	
\item (pp. 81) 1. The figures need to be better embedded into the 
	text -- so you point a figure next to the paragraph discussing it

 	
\item (pp. 81) 2. I worry that much of the evaluation involves looking 
	at the images and this needs the reader to see what you see.


\item (pp. 81) 3. I aslo worry that you haven't fully defined the 
	qualitative categories you use -- like 'smooth'

\item (pp. 81) 4. So, are there ways of quantifying the results?
	And can these numbers be point into the table?
	How much effect do the variables have on the analysis?

\item (pp. 84) REC,DET, etc embedded into RP
 

\begin{verbatim}

RQA values are embedded into the RPs

\end{verbatim}
\textit{
SORTED: 
Wed 17 Oct 17:33:36 BST 2018
}
\\






\end{enumerate}



\subsection{Chapter 7}

\begin{enumerate}

\item (pp. 97) Are there consistent variations, do some differences
	matter more than others?


\begin{verbatim}

Generally, it is evident that time series from different sources 
of time series (e.g. participants, movements, axis type, window size lengths 
or levels of smoothness) will present differences for not only 
computation time of the embedding parameters but also for the 
patterns in RSSs, RP, RQAs and 3D surfaces of RQA metrics. 
Particularly, we observed that the increase of level of smoothness 
of time series created more complex trajectories in the RSSs and 
added more black dots in Recurrence Plots 
(see RSSs and RPs sections in Chapters \ref{chapter5} and \ref{chapter6}) ).
With regard to RQA metrics, we observed that DET metric varies little 
independently of the time series, and REC and RATIO varied a bit 
more but not as much as ENTR metrics for which ENTR metrics are able 
to capture any change make in the time series.

\end{verbatim}
\textit{
SORTED: 
Tue 23 Oct 11:59:39 BST 2018
}
\\





\item (pp. 98) But by how much?



\begin{verbatim}
Sentence were modified to:

The weaknesses and strengths of RQA metrics are related to the selection
of embedding parameters, recurrence thresholds and the capacity 
of RQA metrics to provide understanding on the dynamics of a time series.

\end{verbatim}
\textit{
SORTED: 
Tue 23 Oct 12:48:52 BST 2018
}
\\



\item (pp. 98) increase of what?

\begin{verbatim}
sentence restated to be less ambigous!

\end{verbatim}
\textit{
SORTED: 
Wed 24 Oct 06:43:17 BST 2018
}
\\




\item (pp. 98) as RATIO is comprensible, what causes this?

\item (pp. 99) Explain

\item (pp. 100) rewrite question

\begin{verbatim}

\subsection{How the smoothing of raw time series affects the 
	nonlinear analyses when quantifying MV?}

\end{verbatim}
\textit{
SORTED: 
Tue 23 Oct 12:17:55 BST 2018
}
\\



\item (pp. 100) What else?
	(
	* benefits in the human-humanoid interaction 
	by having the data
	* why variablity is happening?
	)	

\item (pp. 104) Applications for the approach?



\begin{verbatim}

\subsection*{Applications}
I can foresee many of the published work regarding modeling human movement 
variability applied to the context of human-humanoid interaction.
For instance, implementing nonlinear analyses algorithms in humanoid robots 
to evaluate the improvement of movement performances \citep{muller2004}, 
to quantify and provide feedback of level skillfulness as a function 
of movement variability \citep{seifert2011} or to quantify movement 
adaptations, pathologies and skill learning 
\citep{preatoni2007, preatoni2010, preatoni2013}.
More specifically in the context of human-humanoid rehabilitation 
where little work has been done \citep{gorer2013, guneysu2015}
with regards to the use of nonlinear analyses and therefore 
provide adequate metrics to quantify and provide feedback 
for movement variability. 

\end{verbatim}
\textit{
SORTED: 
Wed 24 Oct 13:16:18 BST 2018
}
\\








\end{enumerate}







\section{Minor corrections}
These correction are mainly with the use of the English language, 
suggestions to improve the thesis by CB and some others that improving 
the visual resutls of the figures.


\subsection*{Chapter 1}

\begin{verbatim}
Fixed -- :spellings, typos and corrections of the use of English
\end{verbatim}
\textit{
SORTED: 
Sat 13 Oct 13:49:16 BST 2018
}
\\



\subsection*{Chapter 2}

\begin{verbatim}
Fixed -- :spellings typos and corrections of the use of English
\end{verbatim}
\textit{
SORTED: 
Mon 15 Oct 11:42:31 BST 2018
}
\\


\subsection*{Chapter 3}

\begin{verbatim}
Fixed -- :spellings typos and corrections of the use of English
\end{verbatim}
\textit{
SORTED: 
Mon 15 Oct 15:31:50 BST 2018
}
\\



\subsection*{Chapter 4}

\begin{verbatim}
Fixed -- :spellings typos and corrections of the use of English
\end{verbatim}
\textit{
SORTED: 
Mon 15 Oct 17:31:42 BST 2018
}
\\


\subsection*{Chapter 6}

\begin{verbatim}
Fixed -- :spellings typos and corrections of the use of English
\end{verbatim}
\textit{
SORTED: 
Tue 16 Oct 09:37:47 BST 2018
}
\\









\end{document}

