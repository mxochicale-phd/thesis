%*******************************************************************************
%****************************** Third Chapter **********************************
%*******************************************************************************

\chapter{Time-series Dataset}
% **************************** Define Graphics Path **************************
\ifpdf
    \graphicspath{{chapter5/figs/raster/}{chapter5/figs/PDF/}{chapter5/figs/}}
\else
    \graphicspath{{chapter5/figs/vector/}{chapter5/figs/}}
\fi

%
%%**************************** %Broad Purpose  **********************************
%\section*{Summary and broad purpose of the chapter}
%* How long (number of words)?
%* Deadline
%* What have you got?

%%\section{Time-series dataset collec}
%For this experiment  23 subjects performed simple arm movements;
%however the instructions for the first participant, who was the only left-handed,
%were mistakenly given in a way that movements were performed different
%than what had been planned, to which the data for this participant were taken out.
%Therefore, data for only 22 participants were analysed.

For this thesis, twenty-three participants, from now on defined as $pN$ where $N$ is the number of participant, 
were invited for an experiment of arm movements which will be explained in detail in Chapter 6 and 7.
Data for twenty participants were analysed in this thesis,
since the instructions for $p01$, who was the only left-handed,
were mistakenly given in a way that movements were performed different
from what had been planned, and for participants $p13$ and $p16$ 
data were corrupted because bluetooth communications problems 
with the sensors. With that in mind.
Of the twenty participants, all of them are right-handed healthy participants 
and four are females and sixteen are males with a mean and standard deviation (SD) 
age of mean=19.8 (SD=1.39).

During the experiment, participants worn two neMEMSi Inertial Measurement Units (IMU) 
sensors in their right hand.
Then, each participant was asked to perform ten repetitions of simple horizontal
and vertical arm movements described in the following sections.
We refer the reader to Appendix~\ref{appendix:b} for technical information 
about the IMU sensors.

\section{Arm movements following an image while not hearing a beat}
Participants received instructions to perform the following upper arm 
movements while only looking an image of the movement.
\begin{itemize}[noitemsep,topsep=0pt]
\item ten repetitions of horizontal arm movement at their comfortable speed of each participant,
\item ten repetitions of vertical arm movement at their comfortable speed of each participant,
\item ten repetitions of horizontal arm movement at a faster speed than the comfortable speed
	but not at their fastest speed and
\item ten repetitions of vertical arm movement at a faster speed than the comfortable speed
 	but not at their fastest speed.
\end{itemize}

\section{Arm movements following an image while hearing a beat}
Participants received instructions to perform the following upper arm 
movements while only listening a beat to constraint their movements. 
\begin{itemize}[noitemsep,topsep=0pt]
\item ten repetitions of horizontal arm movement at normal speed,
\item ten repetitions of vertical arm movement at normal speed,
\item ten repetitions of horizontal arm movement at faster speed and
\item ten repetitions of vertical arm movement at faster speed
\end{itemize}


\section{Arm movements following a humanoid robot while hearing beat rate}
Participants received instructions to perform the following upper arm 
movements while imitation upper arm movements of a humanoid robot.
\begin{itemize}[noitemsep,topsep=0pt]
\item ten repetitions of horizontal arm movement at normal speed, 
\item ten repetitions of vertical arm movement at normal speed, 
\item ten repetitions of horizontal arm movement at faster speed and
\item ten repetitions of vertical arm movement at faster speed.
\end{itemize}



