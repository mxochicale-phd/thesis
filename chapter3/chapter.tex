%!TEX root = ../thesis.tex
%*******************************************************************************
%****************************** Second Chapter *********************************
%*******************************************************************************

%\chapter{Nonlinear Time-series Analysis}
\chapter{Quantifying Movement Variability}

%% **************************** Define Graphics Path **************************
%\ifpdf
%    \graphicspath{{chapter3/figs/raster/}{chapter3/figs/PDF/}{chapter2/figs/}}
%\else
%    \graphicspath{{chapter3/figs/vector/}{chapter3/figs/}}
%\fi
%

%%**************************** %Broad Purpose  **********************************
%\section*{Summary and broad purpose of the chapter}
%* How long (number of words)?
%* Deadline
%* What have you got?


%**************************** %Zero Section  ***********************************
\section{Introduction}


How to measure human movement variability?


%stergio2011

One main hightligt of Stergio2011 is that varialibily 
is not anymore consider as a unsdierired factor which creates errors
but a signature of healthlyness (or association with unhealtly pathological states
or accessesment of skillfullness which is reflected as 
the functionality of movmenet.


stergio2011 pointed out the controversy of methods to quantifiy movmente variablityu
where for examples statistical tools like mean and the range are measure of 
centrality where metrics compared areound a central pooint.
Nonetheless, lomax2007 (READ MORE) reported that 
measyiing variatioibliy from statistical stadpoint is valid 
assumis that such variaions between repetionas 
represend randomless and noise.
However, other authors stated the contrary where
varailbiyte ius not noise 
%(Delignières & Torre,2009; Dingwell & Cusumano, 2000; Dingwell & Kang, 2007; Stergiou, Buzzi, Kurz, & Heidel, 2004a).
and such varialibyt have a deterministic origini
%(Delignières & Torre, 2009; Dingwell & Cusumano, 2000; Dingwell & Kang, 2007; Stergiou, Buzzi, Kurz, & Heidel, 2004a).

Esentially, 
tools from nonlinear dynamics
capture better how motor behaviour emerge which is
another aspect that conventioanal statistic tools only 
quantify a magnitude of variation independet of the 
distribution.
These tools are:

approximate entropy
sample entropy
correlation dimension
largest Lyapunov exponent
detrended fluctuation analysis
%( Bruijn, van Dieën, Meijer, & Beek, 2009; 
%Cavanaugh, Kochi, & Stergiou, 2010; 
%Delignières, Deschamps, Legros, & Caillou, 2003; 
%Donker, Roerdink, Greven, & Beek, 2007; 
%Gates & Dingwell, 2007, 2008; 
%Hausdorff, 2009; 
%Liao, Wang, & He, 2008; 
%Kurz & Hou, 2010; 
%Kurz, Markopoulou, & Stergiou, 2010; 
%Sosnoff, Valantine, & Newell, 2006; 
%Sosnoff & Voudrie, 2009; 
%Stins, Michielsen, Roerdink, & Beek, 2009; 
%Vaillancourt, Sosnoff, & Newell, 2004; 
%Yang & Wu, 2010)



\section{Fundamental concepts of time-series analysis}
\subsection{Stochastic vs deterministic}
\subsection{Nonstationary vs stationary}

%review \cite{klonowski2007} 



\section{Quantifying Movement Variability with traditional approaches}
\subsection{Time-domain}
\subsection{Frequency-domain}


\cite{preatoni2013} raised 
when dealing with periodic data
Fourier based approached is appropriate.
if the time series are smooth and non-periodic, then 
B-splitnes may be approapriate.
if the data is noisy with informative spikes 
then avoiding severe smoothign is necessary for which 
wavelet analysis may be appropriate.
%[FOR BSPLITNES
%Coffey, N., Harrison, A. J., Donoghue, O. A., & Hayes, K. (2011). Common functional principal components
%analysis: A new approach to analyzing human movement data. Human Movement Science, 30 (6), 1144–1166.
%]

\subsection{Shortcommings of linear methods}
%
%* 4.1 Shortcommings of linear methods for biosignal analis.
%klonowski2007



\section{Quantifying Movement Variability with Nonlinear Dynamics}
\subsection{RSS}
\subsection{RP and RQA}

%Caballero2014
Recurrence Plots and RQA for analysis the degree of irregulariy of the time seires
with RQA can analyse
postural fluctuations (Riley, Balasubramaniam and Turvey, 1999)
heart rate varialiby (Javorka et al., 2008; Wilkins et al.,2009).

common mehtods to analysis heart rate variablity,
specifically RQA for heart rate varialibitly (shumacher2004)


\subsection{Others: Entropy, LyE, Poincare Maps}


\subsubsection{Lyapunov exponent}
%Caballero2014
LyE has been used to 
ageing changes(Buzzi et al., 2003),
sitting postural control in children (Cignetti,Kyvelidou, Harbourne and Stergiou, 2011)
gait patterns accross lifespan (Smith, Stergiou and Ulrich, 2011).
for more about Lye Wolf et al. (1985)


\subsubsection{Entropy}
%Caballero2014
Entropy mesuare for irregulairy of time series
of which approximate Entropy has been used in many problems such as
quantification of "impared neuromotor control of movmentes in early life"
(Smith,Teulier, Sansom, Stergiou and Ulrich, 2011),
mental fatigue  (Liu,Zhang and Zheng, 2010),
or changes in intracranial pressure 
(Hornero, Aboy, Abásolo, McNames and Goldstein, 2005).

the problems with ApEn is the dpendency wht tiem series length for which,
in 2000, Richman and Moorman proposed Sample Entropy which has been applied 
to quantify postural control  (Menayo, Encarnación, Gea and Marcos, 2014),
or 
"to find differences between schizophrenia and depression"
(Hauge, Berle, Oedegaard, Holsten and Fasmer, 2011).
Then in 2007, Chen et al. develop Fuzzy Entropy which has less
depency to tdata lenght and offer more robutnsess to noise.
FuzzyEn has been used to qunatify muscle fatique 
(Xie, Guo and Zheng, 2010)
to qunatify the problems in satanding  balance tasks 
(Barbado et al.,2012).


Multiscale Entropy Costa, Goldberger and Peng (2002)
Permutation Entropy Vakharia et al. (2014),
Bandt and Pompe (2002).
Another tools to nalayuse the long range auto-correlation 
of time series is fractal features
(Holden, 2005).
or the Detrended Fluctutaion analaysi which is useful
to quantify long-range power-law correlations in time series.
(Peng, Havlin, Stanley and Goldberger, 1995)


Entropy in generals has been used to quantify movement complexity
(Barbado et al., 2012; Chen et al., 2007; Pincus, 1991;
Richman and Moorman, 2000; Xie et al., 2010; Zanin, Zunino, Rosso and Papo,
2012; Zbilut et al., 2000).
However, other authores (Stergiou 2006) relate a measure of 
complexity with "the level of predictabilty"
with an inverted U-shape relathisonship. 
(which is not explained!)

Given the pros and const of the preivoues Nonlinear dybnamics tools,
to measure MV requieres a combination of many of the previous tools
to analysis either the dynamic complexity or the dgree of regularty.
(Goldberger, Peng, et al., 2002; Harbourne and Stergiou, 2009;
Stergiou and Decker, 2011),



\subsection{Shortcoming of nonlinear dynamics tools}



%%% PROBLEMS WITH NONLINEAR TOOLS USING NONSTATIONARY TIMESERIES
Caballero2014 mention that many biological signals have 
the cahractreisc of being nonstationary, 
"noisy series containing extreme values"
(Wijnants, Bosman, Hasselman, Cox and Van Orden, 2009).
which might create unrealible restuls special when suing 
tools like entrooy or LyE. However, tools like DFT or RQA are not
entrleyt affected witht he nonstaionary of the singals
which make them ideas for analaysius of such signas.
Another possiblity is to apply procedure to treat 
nonstatioary time series with nonlinear dynamics.

With that in mind, costa et al. 2007
proposed to derive the time series to have more statiorary data
or use Empirical Mode Decompositon (EMD) READ>> Wu et al. (2014).
To avoid nonstationarly in the siglas, ther authors suggest 
Carroll and Freedman (1993) or Van Dieën, Koppes and Twisk
(2010) to avoid the analysis for the first seconds of the time seires

%empirical mode decomposition
%\cite{huang1998}
%\cite{bonnet2014}
%




Similarly, nonlinear tools are senstiirtve to data lenght
for LyE Wolf et al. (1985), DFA (Delignières et al., 2006).
or the less sensitive to data lenght like 
SampEn
(Rhea et al., 2011),
or FuzzyEn
(Richman and
Moorman, 2000).
(Chen et al., 2009)


and the less afffected tools from the review of Caballero2014
are the RQA 
(Riley et al., 1999; Wijnants et al., 2009)
and PE
(Zunino et al., 2009).


Also that the signal have the right smapling rate is imporatne
for instnace the SampEn is affected as the samplerate of the time series
is increased (Caballero et al., 2013),
to which such sampling rate requires further investionatoin.


%stergio2011
With regard to noise robutness,
In genreral LyE seems to be affected byt noise singlas
(Rosenstein et al., 1993).
but PE and FuzzEn are more robotus to noise 
(Chen et al.,2009; Bandt and Pompe, 2002). 
Althought  RQA can also be affected by noise, these effects can be 
mitigated with the selection of the right parameters to do RQA
(Webber and Zbilut, 2005).













