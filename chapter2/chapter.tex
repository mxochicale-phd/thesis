%!TEX root = ../thesis.tex
%*******************************************************************************
%****************************** Second Chapter *********************************
%*******************************************************************************

\chapter{Literature Reviews}

%% **************************** Define Graphics Path **************************
%\ifpdf
%    \graphicspath{{chapter2/figs/raster/}{chapter2/figs/PDF/}{chapter2/figs/}}
%\else
%    \graphicspath{{chapter2/figs/vector/}{chapter2/figs/}}
%\fi
%

%%**************************** %Broad Purpose  **********************************
%\section*{Summary and broad purpose of the chapter}
%* How long (number of words)?
%* Deadline
%* What have you got?


\section{Movement Variability}
Variability is inherent within and between all biological systems \cite{newell1993}.
For instance, variability has been studied in the analysis of the electroencephalographic 
signals in human brains \cite{klonowski2007}, in physiological signals like the heart 
rate variability \cite{schumacher2004, acharya2006}, respiratory patterns of rats \cite{dhingra2011}, 
in speech variability where not only the linguistic aspect are investigated but 
factors like gender, age, social, state of health, emotional state are
strongly related to uniqueness of the speaker \cite{benzeghiba2007}
or even variability of responses to odors based on culture and gender \cite{ferdenzi2013}.
Variability has also been well studied in human body movement, where 
for instance, \cite{bernstein1967} stated that no human movement is repeated exactly 
with the same trajectory.
Henceforth, movement variability has been used 
as a model of fatigue to prevent 
chronic musculetical disorders (e.g.,  \cite{mathiassen2006, srinivasan2012},
Also, movement variability is also considered as an indicator of skill performance
in sport science where, for instance, \cite{wagner2012} analysed the decrease of movement 
variability based on statistical analysis for three skills levels of 
throwing techniques (low-skilled, skilled, and high-skilled).
\cite{seifert2011} modelled movement variability using hierarchical clustering 
analysis for competitive and recreational swimmers.
Therefore, movement variability is ubiquitous across sports 
(javelin throwing, basketball shooting or running) \cite{bartlett2007}.
Another interesting example is that movement variability can be considered as a 
identifier for personal trait \cite{sandlund2017} where many factors 
of the human body can be considered, such as:
age \cite{kruger2013, macdonald2006, vaillancourt2003, stergiou2016},
gender \cite{svendsen2010},
pain status \cite{madeleine2008, sandlund2008},
body composition  \cite{chiari2002},
work experience \cite{madeleine2009},
pace, movement direction or cognitive demands  
like perception, memory or capacity of introspection \cite{srinivasan2012, kanai2011}.
Additionally, \cite{bartlett2007} highlighted that movement variability can be 
seen from different angles: for instance, cognitive control theorist considered 
variability as undesirable noise and variability is reduced as the skill performance 
is increasing meaning that "becoming dexterous freezes unwanted degrees of freedom 
in the kinematic chain".
In contrast, ecological motor control specialist consider movement variability 
as a functional role in human movement for "coordination change and flexibility to adapt" 
in different environment or movement variability is considered as exploration
and exploitation of body part of the "perceptual-motor workspace" \cite{wu2014, herzfeld2014}.

With regard to the evaluation of healthiness, \cite{stergiou2011} highlighted that 
an optimal state of movement variability is associated with healthiness. 
Similarly, motor disabilities are associated with either wide range of behaviours 
such as random, unfocussed and unpredictable or narrow range of behaviours e.g, 
rigid, inflexible and predictable. For instance, postural sway variability was larger 
for patients with Parkinson disease or variability in step width in elderly 
individuals where too little or too much steep width variability 
is associated with higher likelihood of falling.
Additionally, \cite{stergiou2011} reviewed works that quantify movement variability 
in medicine, such as: variability of hearth rhythms with hearth attacks, 
heart rate irregularities, cardiac death syndrome, blood pressure control, 
brain ischemia, epileptic seizure,  among many others.






\section{Human Movement Variability}

%\subsection{Introduction} 
%\subsection{MV for assessment of skilfulness, pathologies, motor control} 
The human body movement variability is a complex system where many sensorimotor 
variables such as joints, muscles, nervous system, motor unit and cells are the sources 
for different types of variability \cite{newell1993}.
Hence, variability encompasses different types, sources and views of variability.
For instance, from a biomechanical view, motion variability is modelled
as system of differential equations for the nuero-musculoskeletal 
control system where motion variations can occur because of 
"perturbations of initial states of the skeletal",
%for instance where right knee initial angles  can produce another type %of movement", 
perturbations of "muscular or neural subsystems ",
%where, for instance, initial conditions for certain muscles in the right leg can increase the "initial muscle activity"
or "external torques and forces acting on the skeletal system" 
%where for instance gust of wind might perturb the skeletal system.
\cite{hatze1986}.
%For which any injury, pain sensation  effect of fatigue are considered for 
%"variation of muscular parameters, perturbation of sensory neurons",
%and finally any changes int the motor program will create
%"variation of the motor program."
According to \cite{hatze1986} motion variability can be caused by 
(a) direct consequences of adaptive learning process, 
(b) random fluctuation which are the result of stochastic processes in the
nervous system.

Recently, \cite{preatoni2007, preatoni2010, preatoni2013} reported 
that inter-trial variability is 
defined as combination of error $V_e$ in the neuro-motor-skeletal system with 
the associated nonlinear changes $V_{nl}$, therefore the total variability is 
defined as $V_{tol}=V_e+V_{nl}$ and it  "may reveal the effects of adaptation, 
pathologies and skills learning".
Similarly as \cite{hatze1986} and  \cite{wu2014, herzfeld2014},
\cite{preatoni2013} considered that $V_{nl}$ "may be interpreted
as the flexibility of the system to explore different strategies 
to find the most effective one among the many available".
Additionally, \cite{preatoni2013} stated that part of movement variability 
is due to error $V_{e}=V_{eb}+V_{ee}+V_{em}$ which is composed by an addition of 
$V_{eb}$ "error in the sensory information and in the motor output commands",
$V_{ee}$ "changes in the environmental conditions" and 
$V_{em}$ "changes in measuring and data processing procedures".

Another approach to model variability has been proposed by \cite{muller2004},
whom proposed a model that decomposed variability into exploration of 
task tolerance(T), noise reduction(N), and covariation(C).
\cite{muller2004} considered that the quality of performance in goal-oriented tasks, 
e.g. hitting a target, is defined "by the accuracy and replicability of the 
results (deviations from the target) over repeated attempts of execution 
(configuration of join angles with its velocity, angles and position). 
Hence, \cite{muller2004} concluded that T and N contribute more to improvement of a 
performance of a task than C for initial practice, 
meaning that a new combination of angles and velocities explore a 
large region of solution space (hitting the target).
However, for later practice, T diminished  and N and C started to 
be more relevant.
Also, \cite{muller2004} showed in various experiments of throwing actions
that variability in the movement results (deviations from the target) 
is generally smaller than variability in the execution
( variables or release angles and velocities) for which 
it is concluded that covariation between execution variables
is another component of variability.
%However, muller experiments only considered single point data points
%and not evaluation of trajectories.
With that in mind, \cite{muller2004} concluded that task space exploration 
is an essential  contribution to the improvement of movement performances
which is an explain to the noise increase in early practice phases
as explained by \cite{newell1993} 
where variability is part of the exploitation of the workspace.

Moreover, \cite{seifert2011} conducted experiments with competitive and recreational swimmers, 
to conclude that non-expert(recreational swimmers) whom "seek an individual 
coordination pattern to accommodate the novel constrains of locomotion in water"
and experts whom after a considerable practice movement variability 
have a narrower range of movement solutions. However, it is predicted 
that elite swimmers will explore the environments to optimise 
their technique which create another  secondary blooming of variability.
\cite{seifert2011} also mentioned that inter-individual coordination variability 
could be the result of (i) different state of process learning,
(ii) environmental constraints (different perception of the aquatic resistance), or
(iii) different perception of the task constrains (floating instead of swimming).


However, further research requires to be done to quantify variability 
where \cite{newell1998} stated that movement variability can be considered as 
"an emergent property of determinism, stochastic, and even singular processes
in an evolving nonstationary dynamical system" and \cite{stergiou2011} 
pointed out that nonlinear measurement tools revealed that is not the magnitude 
that is important but the structure of movement variability that helps 
to understand human-perceptual-motor functioning which led us to 
the next section of measurements of variability.


\subsection{Measures of Variability}
Measuring movement variability represent also a challenge where for instance 
traditional approaches in statistics or frequency domain tend to fail when 
measuring different types and sources of variability.

For example, \cite{hatze1986} proposed a measure of dispersion to quantify the 
deviation of motion from a certain reference using the Fourier series, however such 
deviations came from angular coordinates (radians) and linear coordinates (meters) 
which is an unaceptable fusion of variables.
Hence, \cite{hatze1986} proposed the use of transentropy which is global quantifier 
for motion variability and is able to measure dispertion considering that any 
movement deviation on a body join may be the result of deterministic and stochastic causes.
Transentropy, as a mesuremente of motion variabilty, is fundamental to compute other 
metrics such as average transentropy, weighted global transentropy or time transentropy.
%\cite{hatze1986} It is also important to note that experimental work of measuring
%running cycles, Hatzel1986 can quantify the variability between four
%cycles of running where the initial cycle has the largest (60m) 
%then it decreased and stay stable until (1600m) and then again
%increased at the final phase. 

\cite{vaillancourt2001} pointed out that no difference of frequency and amplitude 
is presented in postural tremor of patients with Parkinson's disease
but differences in time-dependent structures.
Such changes of time-dependent structures with no differences in amplitude and 
frequency of tremor are associated  with a change of regularity of postural tremor.
Therefore, \cite{vaillancourt2001} considered appreciate entropy (ApEn) to quantify
such regularity in time-dependent structures.
%Entropy metrics (Approximate Entropy ApEn, Sample Entropy SamEn) quantify the 
%regularity of time series either for kinematic or kinetic measure and therefore 
%the increase of regularly means that there is a decrease in the complexity of 
%the system that produce the time series therefore such decrease in complexity 
%is associated with pathological conditions \cite{vaillancourt2001}.


\cite{preatoni2010} pointed out that subtle changes in the neuromuscular system are caused 
by influences of environmental changes, training procedures or latent pathologies.
Measuring such variables with conventional statistics (e.g. standard deviation, 
coefficient of variation, intra-class correlation coefficient) is only for overall
variability. Therefore, using nonlinear dynamics tools such as sample entropy (SampEn)
and approximate entropy (ApEn) can help to analyse the deterministic and stochastic
origin of movement variability.
%using SampEn with original sources of ... and surrogate data where time series maintain 
%large-scale structures like periodicity, mean, variance and spectrum and 
%eliminate small-scale structures like chaotic, linear/nonlinear-determinism.
%It is hence confirmed experimentally that movement variability is not noise
%but the information concerning with regards to the  nuero-musculo-skeletal system. 
%That is because SamEn after surrogation had an increase from 16\% to 59\%,
%suggesting that the time series is not the outcome of a random process \cite{preatoni2010}.
Recently, \cite{preatoni2013} investigated that movement variability is considered
as a compensation of noise in the neuro-musculo-skeletal system 
and the exploration of different strategies of movements to find the most
appropriate pattern for the actual task. 
Such compensation of noise and adaptation of movements cannot be
quantified entirely with the use of conventional approaches for which 
non only the use of entropy measures (SampEn and ApEn) but 
Lyapunov exponent \cite{abarbanel1993, smith2010}.
%reviewed methods of nonlinear dynamics such as
%entropy measures as one of the alternative tools compared to the traditional
%ones to investigate the nature of movement variability in elite athletes.
%Research on quantifying pathologies with nonlinear dynamics has been done,
%however, very little work were reported concerning movement variability in sport 
%science due to limited availability of data.




\section{Human-Robot Movement Variability (HRMV)}
%\section{Movement Variability in Human-Robot Interaction}


\subsection{HRMV in rehabilitation} 
Movement variability in the context of human-humanoid interaction has also been reported
but not investigated in detail.
For example, \cite{guneysu2014} studied automatic evaluation of upper body actions
with eight healthy children mimicked NAO, an humanoid robot and
with the use of a Kinect sensor to get data of join angles of the participants` skeleton.
To evaluate the motion imitation, \cite{guneysu2014} evaluated similarity error,
which is low for hight similarly, using Dynamic Time Warping(DTW) which penalising 
large amount of angle errors and tolerate angles which are ten percent in the area 
range of the motion type.
Also recall measure, which is hight for hight similarity, represents 
"how much of angular area covered by baseline motion is also covered
by the child's motion in the selected frame window."
Then, \cite{guneysu2014} reported physiotherapist's evaluation using Intraclass correlation 
coefficient (ICC) which a metric for reliability of ratings.
However, it it interesting to note that 
the proposed metrics of similarity error and recall measure with the ICC metric are not 
totally reliable since they did not model complex movements (involvement of multiple joins).
%\cite{guneysu2014} used a moving average filter with a window size of 
%50 frames to smooth the data and eliminate unnecessary fluctuations 
%of local minima and maxima.
%Then peak detection of smoothened and unsmoothed data were performed to find 
%alignment of motion directions.
%Evaluation of five therapist using Intraclass correlation coefficient.
%Following the work of Ranatunga et al. which use DTW for the evaluation 
%of similarly of movements  performance between robot and person,
%\cite{guneysu2014} implement DTW with a window size of three frames.
%\cite{guneysu2014} evaluated physiotherapist's evaluation using intraclass
%correlation coefficient based on pooled variance within subjects and 
%variance of the trait between subjects, however the original formula
%proposed in [http://www.john-uebersax.com/stat/icc.htm]
%were modified as true values were not known. 
Recently, \cite{guneysu2015} presented variation of four physiotherapists movements performing 
five actions repeated ten times each which were quantified using basic statistical features 
(sample mean and sample variance). However, \cite{guneysu2015} founded that initial 
positions of arms changed from person to person, specially for the key turning action. 
Additionally, different the structures of time series were presented from each of the 
physiotherapist, where for instance, Therapist 2 performed the activity at half amplitude 
compared to the others.
%reported quaternions data collected with two inertial sensors 


\cite{gorer2013} conducted an experiment for a robotic fitness coach where 
eight participants performed five gestures 
(three for arm related exercises and two for leg strength exercises).
However, it is not clear how the evaluation of 
synchronisation of human's and robot's movements 
with a gold standard movement performed by the robot
was performed since only graphical visualisation
is presented. With that, it was stated that only one subject 
out of eight fail to imitate the gestures correctly.
%\cite{gorer2013} pointed out that three out of the four DOF for NAO arm are for the shoulder
%and one for the elbow with its a limitation for the production of direct mapping
%from human arm join angles.
%Limitations of DOF from robots are compensated with auditory feedback 
%to deal with complex movements.
%Skeleton join angles collected using an a Kinect (RGB-D camera) mounted on Nao's head.
%It was reported that 
%from the eight participants of which only one did not imitated the gesture correctly.


\subsection{HRMV in robotic dance}


Human-robot movement variability can also be seen in robotic dance activities.
For example, \cite{tsuchida2013} explored four dance formations which were performed 
three times by nine participants whom had three years of experience: 
dancing with a robot, dancing alone, dancing with a self-propelled robot and dancing with a projected video.
Two participant's movement positions were presented using twelve trajectories 
per participant of z and x directions for four dance activities 
in three trials.
Although, the dance experiment was very rich in terms of movement variability
for participants, only distance between each of the conditions in the dance formation 
was considered to conclude that the sense of dancing with a projected video of a person 
were the closest to dance with a real person and the trajectory of dance with a 
self-propelled robot were the closest to the trajectory of a dancer.
It is not very clear why the distribution of trajectories for subject 1 were 
more uniform than the trajectories of subject 2.
%\cite{tsuchida2013} considered a choreography of three elements 
%of interseciton (front to back), approaching,
%and parallel translations.


%%\subsection{Movement imitation}
%\cite{ijspeert2002} presented various scenarios for applications of humanoid robotics in
%rehabilitation where, for example, a robot can supervise rehabilitation exercises 
%in stroke patients from demonstrations of professional therapist, 
%demonstrating the motion with a robot, evaluating the performance of the patient and suggesting 
%and demonstrating corrections.
%Similarly, \cite{ijspeert2002} proposed the system of nonlinear
%differential equations which form a control policy for imitation of reaching 
%movements, comparing trajectories of two-dimensional single-stroke patterns,
%and learning tennis swings with a humanoid robot.
%
%

\section{Gaps in Movement Variability in the context of Human-Humanoid Interaction}

Even though with the previous investigation in movement variability little research 
has been done with regard to the conditions of the signals (e.g. window length, 
post-processing techniques, noise contamination, nonstationarity, etc) 
to produce reliable results using different methods to quantify movement 
variability.




