\title{Corrections for thesis from draft v2.0 to draft v2.5}
\author{Miguel Xochicale}
\date{ \today }

\documentclass[10pt]{article}
\usepackage[margin=0.99in]{geometry}
\usepackage{enumitem}

\begin{document}
\maketitle



\begin{abstract}
%This document presents a log book for the corrections 
%of the draf02 handed on 20 Sep 2018 to Chris Baber (CB).
%Hand written comments of thesis draft v2.0 of were made on 10 Oct 2018. 
%These comments of draft 02 are located in 
%\emph{.../revisions/draft0O2/comments/*.pdf}.
%\\
%\\
Chapter 5 has been added with the results of human-image interaction
and minor changes for chapter 6 and 7 by  Miguel Xochicale (MX), 
Draft v2.5 are located in 
\emph{~/phd-thesis/draftrevisions/draft02/draft/v2.5/*.pdf}.
%\\
%\\
%Thesis draft version 02 is located at \emph{.../phd-thesis/thesis-draft02.pdf}.
\end{abstract}

\tableofcontents
\newpage




\section{Major corrections (MX)}


\subsection{chapter 1}


\begin{enumerate}

\item slighly changes of research questions

\begin{verbatim}

\item What are the weaknesses and strengths of RQA metrics when quantifying MV?

\item How the smoothing of raw time series affects the nonlinear analyses
	when quantifying MV?
 
\end{verbatim}
\textit{
SORTED: 
Tue  9 Oct 10:20:41 BST 2018
}
\\




\item amended outline of the thesis

\begin{verbatim}


\section{Outline of the thesis}
This thesis is organised as follow. 
Chapter 1 presents a background for the quantification of 
Movement Variability(MV), state-of-the-art for modelling human MV,
MV in the context of human-humanoid interaction and research questions
are stated.
.
.
.
Finally, Chapter 7 presents conclusions, answers for the
research questions, the contribution to knowledge and 
future work for this thesis
(see Fig \ref{fig:thesis-outline} for the thesis outline).


\end{verbatim}
\textit{
SORTED:
Tue  9 Oct 11:28:33 BST 2018
}
\\



\end{enumerate}







\subsection{chapter 5}


\begin{enumerate}


\item data preparation

\begin{verbatim}

hii paths are added with code and data `~/github/phd-thesis-code-data`.

\end{verbatim}
\textit{
SORTED: 
22 sep 2018 to 28 sep 2018
}
\\

\item description of the results

\begin{verbatim}


\section{Introduction}
\section{Time series}
\section{Minimum Embedding Parameters}
\subsection{Minimum dimension embedding values}
\subsection{Minimum delay embedding values}
\subsection{Average minimum embedding parameters}
\section{Reconstructed state spaces with UTDE}
\section{Recurrences Plots}
\section{Recurrence Quantification Analysis}
\subsection{REC values}
\subsection{DET values}
\subsection{RATIO values}
\subsection{ENTR values}
\section{The weaknesses and strengths of RQA}
\subsection{Sensors and activities}
\subsection{Window size}
\subsection{Smoothness}
\subsection{Participants}
\subsection{Final remarks}

\end{verbatim}
\textit{
SORTED:
1st oct 2018 to Mon  8 Oct 16:19:18 BST 2018
}
\\



\end{enumerate}







\subsection{chapter 6}

\begin{enumerate}

\item Adding subsections to the weaknesses and strenths of RQA.

\begin{verbatim}

\section{The weaknesses and strengths of RQA}
\subsection{Sensors and activities}
\subsection{Window size}
\subsection{Smoothness}
\subsection{Participants}
\subsection{Final remarks}

\end{verbatim}
\textit{
SORTED: 
Mon  8 Oct 16:13:40 BST 2018
}
\\


\item  Polishing Final remarks of weaknesses and stregnths of RQA

\begin{verbatim}

Generally, it can be noted the changes for RQA metrics are evident
with both the increase of embedding dimension parameters and the 
recurrence threshold for different structures, window size, 
levels of smoothness of the time series. 
RATIO values present a plateau (blue colour surface) which is independently 
to the source of time series, however there are peaks that change 
differently based on the source of time series for embedding parameters 
increasing with recurrence threshold less than one.
For DET values, 3D surfaces present a fluctuated surface (red colour)
which slightly differs with the source of time series.
With regards to REC values, 3D surfaces are affected with the type of 
velocity, for instance, surfaces for normal velocity presents 
an increase of REC values as recurrence threshold increase and keeping 
slightly uniform surface changes for the increase of embedding parameters, 
however for faster velocity the 3D surface fluctuations decrease as 
the embedding dimension increase and recurrence thresholds increase.
Similar as the human-image activities, 3D surfaces for ENTR values in 
this experiment present changes for any source of time series,
making ENTR values a robust metric to quantify human-humanoid activities 
for analysis of movements variability from different time series.




\end{verbatim}
\textit{
SORTED: 
Tue  9 Oct 16:57:56 BST 2018
}
\\




\end{enumerate}


\subsection{chapter 7}
Minor changes with the use of English language.
\begin{verbatim}

\end{verbatim}
\textit{
SORTED: 
Wed 10 Oct 14:07:24 BST 2018
}
\\





\end{document}

