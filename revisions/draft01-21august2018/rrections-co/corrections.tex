\title{Corrections for thesis from draft v01 to draft v02}
\author{Miguel P Xochicale}
\date{ \today }

\documentclass[10pt]{article}
\usepackage[margin=0.99in]{geometry}
\usepackage{enumitem}

\begin{document}
\maketitle


\begin{abstract}
The following sections present the corrections for the hand written comments 
of Chris Baber (CB) made on 21 of August 2018 for the thesis draft version 01. 
The comments of draft 01 are located in \emph{.../revisions/draft01-21august2018/comments/*.pdf}.
Thesis draft version 02 is located at \emph{.../phd-thesis/thesis-draft02.pdf}.
Additionally, I added the 3D surfaces to show the sensitivity and robustness 
of RQA metrics for different embedding parameters and recurrence thresholds.
\end{abstract}

%{\scriptsize
% {\tiny

\section{Minor Corrections}

%%%%%%%%%%%%%%%%%%%%%%%%%%%%%%%%%%%%%%%%%%%%%%%%%%%%%%%%%%%%%%%%%%%%%%%%%%%%%%%%
%%%%%%%%%%%%%%%%%%%%%%%%%%%%%%%%%%%%%%%%%%%%%%%%%%%%%%%%%%%%%%%%%%%%%%%%%%%%%%%%
\subsection{toc}

\begin{itemize}[noitemsep,topsep=0pt]
\item Comments about the use of English language!
\end{itemize}

%%%%%%%%%%%%%%%%%%%%%%%%%%%%%%%%%%%%%%%%%%%%%%%%%%%%%%%%%%%%%%%%%%%%%%%%%%%%%%%%
%%%%%%%%%%%%%%%%%%%%%%%%%%%%%%%%%%%%%%%%%%%%%%%%%%%%%%%%%%%%%%%%%%%%%%%%%%%%%%%%
\subsection{chapter 1}

\begin{itemize}[noitemsep,topsep=0pt]
\item Comments were essentially about the improvement of use of English language. \\
\textit{concs.}


\item References need to be in Harvard style. 
\end{itemize}

\subsection{chapter 2}

\begin{itemize}[noitemsep,topsep=0pt]
\item All References need to be in Harvard style. 
\item 2.1 comments are essentially about improvement of use of English.
\item when quoting someone, use page number.\\
\textit{references with pages were added for quotes.}

\item modify title of 2.4 which reads like:
	Gaps in the study of movement variability in the context of 
	human-humanoid interaction (p. 14)
\end{itemize}

\subsection{chapter 4}

\begin{itemize}[noitemsep,topsep=0pt]
\item Corrections with the use of English language
\end{itemize}



%%%%%%%%%%%%%%%%%%%%%%%%%%%%%%%%%%%%%%%%%%%%%%%%%%%%%%%%%%%%%%%%%%%%%%%%%%%%%%%%
%%%%%%%%%%%%%%%%%%%%%%%%%%%%%%%%%%%%%%%%%%%%%%%%%%%%%%%%%%%%%%%%%%%%%%%%%%%%%%%%
\section{Major corrections}

\subsection{chapter 2}

\begin{enumerate}[noitemsep,topsep=0pt]
\item 2.2 "So - if you talk of 'error'  What is this with references to?
	What is the nonerror signal?
	How is $V_e$ actually specified/measured?
	What does $V_{nl}$ mean and how it is measured?" \\


\textit{To explore those questions, we conducted an experiment 
with twenty right-handed healthy participants in the context 
of human-humanoid imitation activities where participants were asked to 
imitate simple arm movements performed by a humanoid robot.}

\item (p. 1) Give a clear meaning to the use of 'intuitive' word in the whole 


\item $V_e= V_{eb} +  V_{ee}+ V_{em}  $, so this say that $V_{e}$
	is made up of different components but 
	how these are measured?
\item What I don't see from this is an explanation on how variability 
	is modelled or measured here. (p. 10)

\item It would be useful to give an example here to show 
	how the parameters are measured and how they interact with each other. (p. 10)

\item So how do the two approaches differ and (more importantly)
	What are they missing? (p. 10)

\item "secondary blooming of variability" What do you mean? (p. 11)
	
\item So, this discussion needs a definition (in maths) of 
	what is variability and what affects it. (p. 11)

\item good, and why would nonlinear dynamics be appropriate? (p. 11)

\item This is a bad sentence that has the end repeat the start (p. 11)

\item Explain the formula and why this is unacceptable. (p. 11)

\item So entropy sounds to me like a measure of the stability 
	of a signal?
	I guess Hatze says anything that changes with time can be shown 
	to be stable?
	But you don't say why he calls it 'transentropy'
	or what is and how it is defined. (p. 11)

\item You don't explain the difference between stability and variability
	or what entropy is intended to measure (p. 12)

\item Of course it is because these are jus variations of the measure 
	why is this useful to say? (p. 12)

\item So -- what do you here is name approaches without 
	(a) explain them or 
	(b) critiquing them.
	If you intent to intent to describe in more depth, say so.
	Also, explain why these approaches are necessary for 
	your thesis.
	(p. 12)

\item Why did they do this study? (p.13)

\item What were the results? (p. 13)

\item What actions? (p. 13)

\item So -- the studies done to data have used small samples 
	and had incomplete analysis.
	What do they tell us? (p. 14)

\item Explain in more detail (p. 14)

\item It would be good to know more about the data (p. 14)

\item So -- how will your thesis fill these gaps? (p. 14)

\end{enumerate}




\subsection{chapter 4}


\begin{enumerate}[noitemsep,topsep=0pt]
\item explain what is meant by an embedding parameter (p. 21)
\item you haven't explained 'attractor' or how this is folded
	(in a manifold) (p. 25)

\item not sure I see this at $m \geq 5$ don't these equal to one. (p.28)

 
\item report in the last chapter (p. 29) 

\item give numbers to define these (p. 31)

\item so -- how does this compare to page 28? 
	
\item so, which one do you use and why? (p. 32)

\item so -- do you need this section?  (p. 32)

\item You haven't defined this or explained it (p. 38)

\item I think that the chapter shows quite well that you understand the various
	methods and can explain them.
	Why you don't do is explain why you will use these particular methods.
	Why are they appropriate to the type of data you intent 
	to collect, the type of variability you expect to see, 
	or the type of analysis you intent to make?
	You should explain your choice of method to the reader. (p. 41)
	
\end{enumerate}









\subsection{chapter 5}


\begin{enumerate}[noitemsep,topsep=0pt]
\item This should be written as an Experiment Method Chapter: \\
	1. Aims \\
	2. Participants \\
	3. Equipment \\
	4. Procedures / ethics \\
	5. Data preparation \\
		each section should have detail
		to allow the experiment to be replicated 
		\\
	I would expect this chapter to be 10 pages long (at least)

\item 5.1, 5.2 and 5.3 (each of these) could be illustrated from the images
	from the images from your instructions (p. 45)

\item Ethics -- you should say that the design of the experiment 
	adhered to UoB regulations. 
	Data were anonymised and stored only on your computer.
	Participants provided informed consent and were free to 
	withdraw from the study	(p. 45).
	
\end{enumerate}





\subsection{chapter 7}

\begin{enumerate}[noitemsep,topsep=0pt]
\item move the underlined section to previous chapter!

\item "results from three participants". 
	I don't see that these are 'space' problems for a thesis
	-- perhaps choice of three to make comparison easier?
	Other data in Appendix X. (p. 55)



\item So -- rather than 'individual' you use 'sample' 
	$m+1$ 
	but this is a new step that you haven't explained previously.
	Be a good idea to anticipate this in your Metrics chapter
	or you move your points on page 59/60 earlier. (p. 57)


\item  Fig 7.6 Should be bigger and clearer	(p. 58)

\item  Figs. 7.7 and 7.8 Should be bigger and clearer	(p. 59)

\item This sentence doesn't make sense (p. 59)

\item You should point some examples out to the reader -- 
	what are the figures 7.9 and 7.10 showing?
	What are the relevant or important things to notice?
	I think there is at least another page of explanation 
	provide here. (p. 60)

\item Figs 7.9 should be bigger and clearer (p. 61)

\item Figs 7.10 should be bigger and clearer (p. 62)

\item Again -- explain this figures and point out the important 
	relevant features.
	Don't assume the reader will see everything that you do.
	(p. 62)



\item Figs 7.11 should be bigger and clearer (p. 63)

\item "HS01" Is this from one person? Explain what
	HS01 means, and why use these data.

\item Would you expect them to change? (p.65)

\item This belongs to earlier in the metrics review chapter (p. 68)

\item Every figure should have at least one paragraph of explanation
	to point out key features (p. 70)

\item I can see you have produced results 
	but it is not obvious how these are meant to be interpreted.
	You need more explanation of the important and relevant elements
	of each figure.
	You need to say whether the results are expected and
	whether different methods agree or contradict each other. \\
	I am also worried that not including any of  your other data
	means you risk the thesis looking like a single study 
	MSc by Research rather than a PhD.


\end{enumerate}



\subsection{chapter 8}

\begin{enumerate}[noitemsep,topsep=0pt]

\item What were the research questions? \\
	How were these answered?\\
	How does your work extend and advance the field?\\
	(p. 73)

\end{enumerate}







\end{document}





