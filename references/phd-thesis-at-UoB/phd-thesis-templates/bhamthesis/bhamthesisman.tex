%% This is `bhamthesisman.tex'.

%% Version: Tin Lok Wong 19/03/2008

%%

%% This is an intended manual for the LaTeX document class bhamthesis

%%  written to match the thesis requirements set by

%%  the University of Birmingham.

%%

%% ******************************************************************

%% ** This is NOT an official document produced by the University. **

%% ******************************************************************

%%

%% Modification history:

%%  26/12/2007 Tin Lok Wong - document created

%%  27/12/2007 Tin Lok Wong - enriched

%%  28/12/2007 Tin Lok Wong - should be fine for now

%%                          - fill-in mode chapter written

%%  31/12/2007 Tin Lok Wong - finished

%%  17/03/2008 Tin Lok Wong - added the blank line problem

%%                          - added the pdftex problem

%%  19/03/2008 Tin Lok Wong - added display and displaypar

%%                          - added html.sty problem

%%

\documentclass{bhamthesis}



\usepackage{amsmath}

\usepackage{amstext}

\usepackage{layout}

%% \usepackage{layouts}  %% Xtra: needs layouts.sty

%% \usepackage{url}      %% Xtra: needs url.sty - to break URLs

\usepackage{bhamthesisfill}



\author{Tin~Lok Wong}

\title{The Unofficial \clsname\ Package\\ for \LaTeXe}

\submissionstatement{For theses to be submitted to}  %% Non-standard use

\degree{The University of Birmingham}                %% Non-standard use



\makeatletter

\renewcommand{\cite}{\@ifstar{\@ifstar{\HAR@acite}{\HAR@fcite}}{\HAR@acite}}

\makeatother



\newcommand{\mar}[1]{\marginpar{\raggedright#1}}



\newcommand{\cm}{\mathrm{cm}}

\newcommand{\mm}{\mathrm{mm}}

\newcommand{\inch}{\mathrm{inch}}

\newcommand{\pt}{\mathtt{pt}}

\providecommand{\BibTeX}{\textsc{Bib}\TeX}

\newcommand{\CTAN}{\textsc{ctan}}

\newcommand{\solnmark}{\textasteriskcentered}

\newcommand{\req}{\begin{singlespace}\small\item}

\newcommand{\soln}{\end{singlespace}\normalsize\item[\solnmark]}

\newcommand{\clsname}{\pkg{bhamthesis}}

\newcommand{\bksl}{\char`\\}

\newcommand{\cmd}[1]{\texttt{\bksl{}#1}}

\newcommand{\pkg}[1]{\textsf{#1}}

\newcommand{\file}[1]{\texttt{#1}}

\newcommand{\env}[1]{\texttt{#1}}

\newcommand{\ext}[1]{\texttt{#1}}

\newcommand{\opt}[1]{\texttt{#1}}

\newcommand{\bkname}[1]{\textit{#1}}

\newcommand{\bfield}[1]{\textbf{#1}}

\newcommand{\linesep}{\begin{center}\makebox[.5\textwidth][s]{\hrulefill}\end{center}}

\newcommand{\textarg}{\char`\{\char`\}}

\newcommand{\cmdargt}[1]

 {\mdseries\texttt{\char`\{}\argt{#1}\texttt{\char`\}}}

\newcommand{\argt}[1]

 {$\langle\mbox{\itshape #1}\rangle$}

\newenvironment{ttlist}

 {\begin{center}\ttfamily\begin{tabular}{l}}

 {\end{tabular}\end{center}}

\newcommand{\desbox}[2]{\begin{minipage}[t]{#1\textwidth}\raggedright#2\end{minipage}}

\newcommand{\cmddessep}{\renewcommand{\labelsep}{2em}}

\providecommand{\url}[1]{\(\mathtt{#1}\)}



\begin{document}

\frontmatter

\maketitle





%% \begin{abstract}

%% This document contains details about how \clsname\ try to satisfy

%% the thesis requirements of the University of Birmingham, and how

%% it can be used.  Limitations and known problems are also

%% discussed.

%% \end{abstract}





%% \begin{dedication}

%% To students in the University of Birmingham, especially those in

%% the School of Mathematics

%% \end{dedication}





\begin{acknowledgements}

Firstly, I want to thank \file{report.cls}, \file{size12.clo} and

\file{latex.ltx} which form the basis of the \clsname\ class, and

the \pkg{custom-bib} package which form the basis of the \clsname\

\BibTeX\ style.  These files are all from the standard \LaTeX\

package.



In addition, I am grateful to all the manuals and books I

consulted.  These include the

\bkname{\TeX{}book}~\cite{man:TeXbook}, the \LaTeX\

manual~\cite{book:LaTeX_manual}, the \bkname{\LaTeX\

Companion}~\cite{book:LaTeX_companion}, the \BibTeX\

manual~\cite{man:design-bst}, and `Tame the BeaST'~\cite{man:ttb}.





The class file \file{adfathesis.cls} from

\url{http://tug.ctan.org/tex-archive/macros/latex/contrib/adfathesis/}

showed me a way to break lines in the title.  An answer in the UK

List of \TeX\ Frequently Asked Questions on the Web at

\url{http://www.tex.ac.uk/cgi-bin/texfaq2html?label=oarglikesect}

also lent me a trick to deal with optional arguments.



Finally, I would like to thank Eligio Cerval-Pe\~na for pointing

out a clash with the \pkg{html} package and various comments.  The

clash is hopefully fixed now.

\end{acknowledgements}





\tableofcontents %

%% \listoffigures %

%% \listoftables





\mainmatter

\chapter{Disclaimer and Introduction}

Disclaimer: \clsname\ is \emph{not} an official document class

produced by the University of Birmingham.  There is \emph{no}

guarantee that a document that uses \clsname\ will be accepted by

the University.



\linesep



It seems nice if there exists a \ext{cls} file for students at the

University of Birmingham who write their theses in \LaTeX. For

some reason, it does not exist when I tried to find it myself. I

hope that the package \clsname\ will be helpful to those who don't

bother to write it, those who don't have enough time to write it,

and those who don't feel like struggling with it.



By no means I am implying that I like the style provided by

\clsname.  It is \emph{not} the one that I usually use, but is the

one required by the University (at the time of

writing~\cite{booklet:prequote,booklet:prethesis}). I followed the

requirements rigidly, \emph{ignoring} all personal preferences on

aesthetics, common sense, and logic.



Chapter~\ref{ch:usage} of this manual tells one how to load

\clsname. Chapter~\ref{ch:cmd+param} contains a description of the

macros provided.  You should be able to use the full strength of

\clsname\ by reading up to this point.  Chapter~\ref{ch:require}

shows the things \clsname\ do to match the thesis requirements.

Chapter~\ref{ch:fill-in} has some information about the

supplementary mode of \clsname\ called the \emph{fill-in mode},

and Chapter~\ref{ch:beware} is a list of things look out at when

using \clsname.  These chapters in the main text should be

sufficient for most users.  The appendices contain the technical

stuff.  The appendix will be useful if you are trying to modify

\clsname.



I hope you all have a good time writing up.  I endeavour to make

\clsname\ as good as possible, and so your comments and

suggestions are most welcomed.  Please send them to me via e-mail

(\texttt{wongtl}).



Good luck, and enjoy!





\chapter{Usage}\label{ch:usage}

The current release of \clsname is of version~1.41 and dated

19~March 2008. It contains the following files.

\begin{center}

\begin{tabular}{ll}

 \file{bhamthesis.cls}

   &\desbox{.5}{The document class} \\

 \file{bhamthesis.bst}

   &\desbox{.5}{The \BibTeX\ style} \\

 \file{bhamthesisfill.sty}

   &\desbox{.5}{The supplement package} \\

 \file{bhamthesisman.tex}

   &\desbox{.5}{\LaTeX\ source for this manual} \\

 \file{bhamthesisman.pdf}

   &\desbox{.5}{This manual} \\

 \file{man.bib}

   &\desbox{.5}{\BibTeX\ database for the manual}

\end{tabular}

\end{center}

There are three \LaTeX\ packages required by \clsname.  They can

all be obtained from \CTAN.

\begin{center}

\begin{tabular}{ll}

 \pkg{setspace}

  &\desbox{.7}{\url{http://www.ctan.org/tex-archive/macros/latex/contrib/setspace/setspace.sty}}

   \\

 \pkg{perpage}

  &\desbox{.7}{\url{http://www.ctan.org/get/macros/latex/contrib/ednotes/perpage.sty}}

   \\

 \pkg{harvard}

  &\desbox{.7}{\url{http://www.ctan.org/get/macros/latex/contrib/harvard/harvard.sty}}

\end{tabular}

\end{center}

Put these file (possibly except the manual files) to a place where

\LaTeX\ can find, e.g., the same folder as your \ext{tex} files.



To write a document using \clsname, put

\begin{ttlist}

\verb|\documentclass{bhamthesis}|

\end{ttlist}

on the first line of your \ext{tex} file (and remove other

commands who attempt to set the document class to something else).

Only two options are supported: \opt{harvard} and \opt{noharvard}.

The option \opt{harvard} loads the \pkg{harvard} package, and

option \opt{noharvard} doesn't.  Obviously, they can't be chosen

at the same time.  By default, \clsname\ chooses

\opt{harvard}.\footnote{Since \clsname\ is modified from the

standard \LaTeX\ \pkg{report} class, the options available there

are also `available' here, although they are not adequately

tested. If not requested otherwise, \clsname\ executes the

\opt{a4paper}, \opt{12pt}, \opt{oneside}, \opt{onecolumn},

\opt{final}, \opt{openany} options internally.}



To use the \BibTeX\ style \file{\clsname.bst}, just do the usual

thing: include

\begin{ttlist}

 \verb|\bibliographystyle{bhamthesis}|

\end{ttlist}

somewhere before your bibliography.



If you decide to switch back temporarily to the standard \LaTeX\

\pkg{book}, \pkg{report} or \pkg{article} class after using

\clsname\ for a while, then you can include

\begin{ttlist}

\verb|\usepackage{bhamthesisfill}|

\end{ttlist}

in the preamble before any \clsname\ stuff is used, so that the

missing commands are defined for you.\footnote{Therefore,

\file{bhamthesisfill.sty} is actually not needed in the normal

usage. However, there is no harm in including it even if you are

using \clsname.}  If you decide to load \file{bhamthesisfill.sty}

in conjunction with \pkg{harvard} or \pkg{perpage}, you

\emph{must} load \file{bhamthesisfill.sty} before loading the

external packages because the definitions will clash.  More about

\file{bhamthesisfill.sty} will be said in

Chapter~\ref{ch:fill-in}.



The file \file{bhamthesisman.pdf} is the manual you are reading,

while \file{bhamthesisman.tex} is its \LaTeX\ source.  The manual

is written in the \clsname\ class itself and hopefully it can

serve as a demonstration.\footnote{You may not get the same thing

as the \ext{pdf} file when you \LaTeX\ the \ext{tex} file, because

some lines are commented out to minimize the extra packages

needed.  Actually you may get a lot of \texttt{overfull boxes} and

\texttt{underfull boxes} because of the URLs.  To see everything

in a proper way, download the \pkg{layouts} package from

\url{http://www.ctan.org/tex-archive/macros/latex/contrib/layouts/}

and the \pkg{url} package from

\url{http://www.ctan.org/get/macros/latex/contrib/misc/url.sty},

and `comment in' the lines marked with `\;\texttt{\%\% Xtra}\;' at

the end.}





\chapter{Commands and Parameters}\label{ch:cmd+param}

The usual commands like \cmd{author\textarg}, \cmd{date\textarg},

\cmd{maketitle}, \cmd{footnote\textarg}, \cmd{figurename} and the

usual environments like \env{center}, \env{itemize} should

stay the same.  The commands \cmd{chapter\textarg},

\cmd{section\textarg}, etc.\ and the environments \env{quote},

\env{table}, etc.\ should work as expected.  I didn't even try

to touch \cmd{label\textarg}, \cmd{emph\textarg},

\cmd{cite\textarg}, \ldots



\section{Additional commands and parameters}\label{sec:add-cmds}

There are very few new commands you need to know.  There are three

high level sectioning commands.

\begin{description}\cmddessep

\item[\cmd{frontmatter}]

 Put this \emph{immediately} after \verb|\begin{document}|.  This

 command does virtually nothing, but is a nice thing to have.

\item[\cmd{mainmatter}]

 Put this to a place where you want the page numbering to start.

\item[\cmd{backmatter}]

 This command does absolutely nothing at the moment.  Put it in a

 sensible place.  It may be useful in the future.

\end{description}



Put the following in the preamble if you need them.  I don't know

what will happen if you put them in the main document.

\begin{description}\cmddessep

\item[\cmd{submissionstatement}\cmdargt{substate}]

 This takes one argument \argt{substate}, which will be the few

 lines of text that appears before the name of your degree in the

 title page.  The official

 statement~\cite[Section~3.2.1]{booklet:prethesis} is

 \begin{quote}

  A thesis submitted to\\

  The University of Birmingham\\

  for the degree of

 \end{quote}

 and so it is set as the default if you don't change it.

\item[\cmd{degree}\cmdargt{degname}]

  This sets the name of your degree as \argt{degname}, which

  appears on the title page. Since presumably every one wants to

  get a PhD, the default degree name is `Doctor of Philosophy'.

\item[\cmd{school}\cmdargt{schoolname}]

 This takes one argument \argt{schoolname}, which will be set as

 the name of school, and will appear in the title page.  If you

 don't tell me which school you are from, I will assume you are

 from the `School of Mathematics'.

\item[\cmd{university}\cmdargt{uniname}]

 The argument \argt{uniname} is the university name that appears

 on the title page.  For obvious reasons, the default university

 name is `The University of Birmingham'.

\end{description}



There are a few environments that help you create some special

parts of your thesis.  Put them in a place you want it to appear.

\begin{description}\cmddessep

\item[\env{dedication}]

 Typeset your dedication inside this environment.

\item[\env{acknowledgements}]

 Put your acknowledgements inside this environment.

\end{description}



If you want to customize the appearance, you can try to alter the

following parameters.\footnote{There are a few other new internal

commands and/or parameters.  I don't think they will be useful to

you, but you can always look them up in the messy \ext{cls} file.}

\begin{description}\cmddessep

\item[\cmd{addsswidth}]

 This helps decide the width of the box in the bottom right hand

 corner of the title page. If it is set to a positive value, then

 this value will be the width of the aforementioned box. If not,

 this width will be calculated so that the box is stuck to the

 right margin.

\item[\cmd{acknowname}]

 This is the title in the acknowledgement part. Initially this is

 set to `Acknowledgements'.

\item[\cmd{dedwidth}]

 This is the width of the dedication text. Its initial value is

 $\text{\cmd{textwidth}}/2$.

\item[\cmd{dedabove}]

 This is the proportion of white space above the dedication text,

 which is chosen to be~$3$ initially.

\item[\cmd{dedbelow}]

 This is the proportion of white space below the dedication text,

 which I chose to be~$8$ initially.

\end{description}



I am aware that some people use \env{quote} and \env{quotation} to

typeset important sentences, mottos, etc., because I am one of

them.  \clsname\ makes these environments single spaced because

they are meant to be used for quotes.  To display something double

spaced, we provide the following.

\begin{description}\cmddessep

\item[\env{display}]

 This is just \env{quote} double spaced.

\item[\env{displaypar}]

 This is just \env{quotation} double spaced.

\end{description}



\section{The environments \env{figurepage} and

         \env{tablepage}}\label{sec:fig+table-page}

Sometimes some special arrangements are needed when page-size

figures or tables are included in theses.  The environments

\env{figurepage} and \env{tablepage} are designed for this

purpose.  They do two things:

\begin{enumerate}

\item try to put the figure or table on a separate page as a

  float; and

\item increase the inner margin by \cmd{plateextramar} when it

  does that.\footnote{Actually, the margins are \emph{not}

  changed.  Only a white space is inserted on the inner side.  So

  the page number is actually not centred with respect to a

  float page, but to a usual page.  This seems to be a nicer and

  easier approach.}

\end{enumerate}

\cmd{plateextramar} is set to be $1\,\cm$ by default according to

the thesis requirements~\cite{booklet:prethesis}.  (See also

Chapter~\ref{ch:require}.)  Both environments do not take

arguments.



There are two things to note here.

\begin{itemize}

\item No two elements of

  $\{\mathtt{figurepage},\mathtt{tablepage}\}$ are put on

  the same page.



\item The command \cmd{caption\textarg} also works in these

  environments, but it works slightly differently.  In standard

  \LaTeX, \cmd{caption\textarg} is used to assign a figure or

  table number to the object, and to print out the caption.  In

  \env{figurepage} and \env{tablepage}, a number is

  \emph{always} assigned to the object, and this is \emph{always

  shown at the top}.  If you specify the caption, then it will be

  shown at the bottom, \emph{no matter where you put

  \cmd{caption\textarg}} (in the environment).  In particular,

  it doesn't matter where you put your \cmd{label\textarg}.

\end{itemize}



\section{Additional \mbox{\normalfont\BibTeX} fields}

There are a few additional \BibTeX\ fields provided by the

\clsname\ style.



\begin{description}\cmddessep

\item[\bfield{url}] Clearly, this is intended for the URL of the

  item.

\item[\bfield{urldate}] This stores the date on which the item is

  obtained from the URL stated.

\item[\bfield{abbrvname}] In case you don't like the default, this

  is the name you choose to use when referring to this item in the

  text (except for the first time).

\end{description}



\section{External packages}

The \pkg{setspace} package is automatically loaded by \clsname. It

is helpful in altering inter-line spacing.  It does not come with

a proper manual, but the beginning of the file \file{setspace.sty}

contains a description of all its commands.



The \pkg{harvard} package is also automatically loaded by

\clsname.  It helps changing the format of the

bibliography.\footnote{There are just too many ways to cite an

item that I didn't even attempt to provide such a package.

Although \pkg{harvard} contains some defects, it should be good

enough for most purposes, and unlike \clsname, it will hopefully

be continuously improved by various people.} There are also a

number of fancy citation commands available. Please see its

manual, which is available at

\url{http://www.ctan.org/get/macros/latex/contrib/harvard/harvard.ps},

for the variety of commands it provides.\footnote{If you are using

an overly helpful text editor (e.g., \pkg{WinEdt}), you may be

bombarded with messages telling you to download \pkg{latex2html}.

This is actually not necessary for \clsname.  To avoid this

annoyance (if you are using \pkg{harvard} version~2.0.5), replace

\begin{ttlist}

 \cmd{IfFileExists}\char`\{html.sty\char`\}

  \char`\{\cmd{RequirePackage}\char`\{html\char`\}

\end{ttlist}

on line~7 in \file{harvard.sty} by

\begin{ttlist}

 \cmd{RequirePackage}\char`\{html\char`\}\char`\{\%

\end{ttlist}

and the problem will be solved.  Please do this in secret because

according to \LaTeX\ convention, the packages are not supposed to

be altered without changing the file names.}





\chapter[Thesis Requirements]

  {Thesis Requirements and what is done in view of them}%

\label{ch:require}%

I referred to the present thesis

requirements~\cite{booklet:prequote,booklet:prethesis} available

from the Internet when I wrote the \clsname\ package. Let's hope

they won't change before we graduate.



\section{The preliminaries and the main text}

Here I quote the relevant things the

requirements~\cite{booklet:prethesis} say about a typical thesis

in mathematics.  Hopefully these quotations are not a mere

redundancy, and will be useful when the thesis requirements

change.  On the other hand, \emph{there may be something else in

the official document~\cite{booklet:prethesis} that you need to

know} because all of you are better than typical.  So take some

time reading it.  In what follows in this section,

\labelitemi~stands for the official

requirements~\cite{booklet:prethesis} and \solnmark~means me.



\begin{itemize}

\req `You should use

  A4~paper.'~\cite[Section~2.1]{booklet:prethesis}

  `You must type on one side of the paper

  only.'~\cite[Section~2.2]{booklet:prethesis} `[A] 12~point typeface

  is the recommended standard for general

  use.'~\cite[Section~2.3]{booklet:prethesis}

\soln These are the default options now.  Of course you need to

  print your final script on one side of the paper yourself.



\req `The paper itself should be good quality bond paper, weight

  to be at least 70~gram\-mes per square metre

  (gsm).'~\cite[Section~2.1]{booklet:prethesis}  `The final script

  should be printed using a letter-quality

  printer.'~\cite[Section~2.2]{booklet:prethesis}

\soln I can do nothing about these.



\req `The left-hand margin must be least $3\,\cm$.  [\ldots]  It

  is desirable to leave $3\,\cm$ at the top and bottom of the page

  and about $2\,\cm$ at the outer

  edge.'~\cite[Section~2.4]{booklet:prethesis}

\soln Presumably a piece A4~paper is of dimensions $297\,\mm

  \times 210\,\mm$.  So after some calculations, the left, right,

  top and bottom margins should be $3\,\cm$, $2\,\cm$, $3\,\cm$

  and $3\,\cm$ respectively.  Hence the actual box for text is of

  size $237\,\mm \times 160\,\mm$.



  As you may already know, one can't set these directly, and

  \LaTeX\ usually mess these up with its internal adjustments.

  This is the best I can do:

  \begin{itemize}

  \item \cmd{hoffset} and \cmd{voffset} are both zero by

    default.

  \item \cmd{paperwidth} and \cmd{paperheight} are set to

    $210\,\mm$ and $297\,\mm$ respectively.

  \item \cmd{textwidth} and \cmd{textheight} are set to $160\,\mm$

    and $237\,\mm$ respectively.

  \item The left margin is controlled by \cmd{oddsidemargin}

    because we print on one side of the paper only.  This

    parameter is given by

    \begin{align*}

     \text{\cmd{oddsidemargin}}

      &= \frac{6}{10}(\text{\cmd{paperwidth}}-\text{\cmd{textwidth}})

         - 1\,\inch \\

      &= \frac{6}{10}(210\,\mm-160\,\mm) - 1\,\inch \\

      &= 30\,\mm - 1\,\inch.

    \end{align*}

    The left margin should then be

    \(\text{\cmd{oddsidemargin}} - 1\,\inch - \text{\cmd{hoffset}}\),

    which is $3\,\cm$, and the right margin should be

    \begin{align*}

     &\text{\cmd{paperwidth}}

      -\text{\cmd{textwidth}}-\text{left margin} \\

     &= 210\,\mm - 160\,\mm - 30\,\mm \\

     &= 20\,\mm.

    \end{align*}

  \item Not surprisingly, the top margin is controlled by

    \cmd{topmargin}, but they are not the same.  Other parameters

    that affect the top margin are \cmd{headheight} and

    \cmd{headsep}.  We don't care what values they take because we

    set \cmd{topmargin} to be

    \begin{equation*}

     3\,\cm - \cmd{headheight} - \cmd{headsep} - 1\,\inch,

    \end{equation*}

    and the top margin is

    \begin{align*}

     \text{\cmd{voffset}}+1\,\inch+\text{\cmd{topmargin}}

     +\text{\cmd{headheight}}+\text{\cmd{headsep}},

    \end{align*}

    which comes out to be $3\,\cm$.  The bottom margin can then be

    calculated to be

    \begin{align*}

     &\text{\cmd{paperheight}}

     -\text{top margin}-\text{\cmd{textheight}}\\

     &= 297\,\mm - 30\,\mm - 237\,\mm \\

     &= 30\,\mm.

    \end{align*}

  \end{itemize}

  All lengths are rounded to the nearest \TeX\ $\pt$ in

  some parts of the calculations, and all units are \TeX\

  dimensions.  \emph{Always check when you print something out}

  because the printer may move the offsets, and there are rounding

  errors in calculations. Figure~\ref{fig:layout} shows what

  \LaTeX\ thinks the page looks like, where $1\,\inch$ is

  equivalent to $72.27$~\TeX\

  $\pt$s~\cite[Chapter~10]{man:TeXbook}. %

%%   Figure~\ref{fig:pgparam-meaning} shows us pictorially what the   %% Xtra: needs layouts.sty

%%   parameters mentioned above means.  Table~\ref{table:pgparam}     %% Xtra: needs layouts.sty

%%   shows the actual parameter values for the present document.      %% Xtra: needs layouts.sty

%%   Both Figure~\ref{fig:pgparam-meaning} and                        %% Xtra: needs layouts.sty

%%   Table~\ref{table:pgparam} are drawn using the \pkg{layouts}      %% Xtra: needs layouts.sty

%%   package.                                                         %% Xtra: needs layouts.sty

  \begin{figurepage}

   \centering\vskip\abovecaptionskip

   \layout

   \vskip .08\textheight

   \caption{Current page layout drawn by the \pkg{layout} package}%

   \label{fig:layout}%

  \end{figurepage}

%%   \begin{figurepage}                                             %% Xtra: needs layouts.sty

%%    \centering\reversemarginpartrue%                              %% Xtra: needs layouts.sty

%%    \pagediagram%                                                 %% Xtra: needs layouts.sty

%%    \caption{Meanings of page layout parameters (not to scale)}%  %% Xtra: needs layouts.sty

%%    \label{fig:pgparam-meaning}                                   %% Xtra: needs layouts.sty

%%   \end{figurepage}                                               %% Xtra: needs layouts.sty

%%   \begin{table}[bt]                                              %% Xtra: needs layouts.sty

%%   \centering\reversemarginpartrue\printinunitsof{mm}\pagevalues %% Xtra: needs layouts.sty

%%    \caption{Page layout parameters                               %% Xtra: needs layouts.sty

%%      (correct to the nearest $0.00001\,\mm$)}%                   %% Xtra: needs layouts.sty

%%    \label{table:pgparam}                                         %% Xtra: needs layouts.sty

%%   \end{table}                                                    %% Xtra: needs layouts.sty



\req `A clear standard typeface should be used. Recommended

  typefaces include Times Roman and Courier; a number of other

  standard faces are equally acceptable but consult your

  supervisor if in doubt.'~\cite[Section~2.3]{booklet:prethesis}

\soln The default typeface is that used by \LaTeX.  I think that

  is called Computer Modern.



\req `Use double line-spacing throughout, except for quotations,

  footnotes, captions, etc.\ which may be single spaced.

  Single-spacing within items in the bibliography is also

  acceptable.'~\cite[Section~2.2]{booklet:prethesis}

\soln The package \pkg{setspace} is loaded to enable double

  line-spacing.\footnote{I did not do this myself not because it

  is hard to double the inter-line spacing (using

  \cmd{linespread\char`\{2\char`\}}), but because there

  are too many things to take care of here: switching back to

  single spacing keeping an eye on the spacing during the

  transitions, getting into the inner workings of

  \cmd{footnote\textarg} to make it single spaced, etc. In

  addition, there are definitely more people improving

  \pkg{setspace} than \clsname.  So I left \pkg{setspace} as an

  add-on package.}  If not specified, \clsname\ makes the whole

  document doubly spaced.  \pkg{setspace} leaves the footnotes,

  figures and tables single spaced, and according to the

  requirements, the environments \env{quote},

  \env{quotation}, and the bibliography are made single spaced.

  Lines are single spaced in marginal notes as well.

  You may want more places to be single spaced, e.g., the table of

  contents, the abstract, etc.  Do it yourself using the commands

  provided by the \pkg{setspace} package.



\req `Typing should be reasonably uniform in length of line and

  the number of lines per page.'~\cite[Section~2.2]{booklet:prethesis}

\soln I think these are all done in \TeX\ automatically unless you

  spoil it.



\req `Page numbers may either be placed at top-centre, $1\,\cm$

  below the edge or at the foot of the page, $2\,\cm$ above the

  edge.  Be consistent in whichever style you

  choose.'~\cite[Section~2.4]{booklet:prethesis}

\soln I chose to place the page numbers at the bottom.\footnote{If

  you want to place them at the top, then you'll have to find your

  own way to do it.}  The bottom of the page number is made

  $1\,\cm$ below the bottom of the main text box by setting

  \cmd{footskip} to $1\,\cm$.  This means that the bottom of the

  page number should be $2\,\cm$ from the bottom edge because the

  bottom margin is meant to be $3\,\cm$ wide.



\req `Preliminary pages [i.e., the title page, abstract,

  dedication, acknowledgements, table of contents, list of

  illustrations, list of tables, list of definitions and/or

  abbreviations] are unnumbered, pagination beginning with the

  first page of the text

  proper.'~\cite[Section~2.4]{booklet:prethesis}

\soln There is no page numbering given by \clsname\ unless you

  specify it explicitly using \cmd{mainmatter}.  Putting this

  command just before the first \cmd{chapter\textarg} would do the

  job for most of you.



\req `New chapters should always commence on a fresh page. Titles

  should be in capitals and centered. Sub-headings within chapters

  should be left justified.'~\cite[Section~2.5]{booklet:prethesis}

\soln The original \cmd{chapter\textarg} is modified to make the

  chapter titles in capitals and centered.  Other sectioning

  commands like \cmd{section\textarg} should work fine without

  modification.



\req `Plates can be mounted upright or sideways.  The general

  rule in either case is: caption below plate, and plate number

  immediately above the plate.  If the plate is to be mounted

  sideways then its head should be towards the binding edge.



  `If the plate is to be inserted facing the text then a binding

  margin of $4\,\cm$s must be available: if the plate is to face

  the blank verso of the previous page then the binding margin

  must be at the left.



  [\ldots]



  `Page-size graphs [and diagrams] should be treated in the same

  way as plates [\ldots] with regard to numbering, captions and

  margins.'~\cite[Sections~2.8 and~2.9]{booklet:prethesis}

\soln I ignored plates, and concentrated on page-size graphs

  and diagrams that are to be mounted upright facing the blank

  verso of the previous page.  In these cases, you can use the

  environments \env{figurepage} and \env{tablepage}

  described in Section~\ref{sec:fig+table-page}.



\req According to the

  requirements~\cite[Section~3]{booklet:prethesis}, the thesis has to

  follow the order depicted in Figure~\ref{fig:thesis-order}.

  `Unlike a book, a thesis has no

  index.'~\cite[Section~3.2.5.1]{booklet:prethesis}

  %

  \begin{figure}[tbh]

  \begin{align*}

   &\text{\namepart{1}}\rightarrow\text{\namepart{2}}\rightarrow

    \text{\namepart{3} (optional)} \\

   &\rightarrow\text{\namepart{4}}\rightarrow\text{\namepart{5}} \\

   &\rightarrow\text{\namepart{6} (if needed)}\rightarrow

    \text{\namepart{7} (if needed)} \\

   &\rightarrow\text{\namepart{8} (if needed)} \\

   &\rightarrow\text{\namepart{9}}\rightarrow\text{\namepart{10}}

    \rightarrow\text{\namepart{11}}

  \end{align*}

  \label{fig:thesis-order}%

  \caption{Order of thesis elements}

  \end{figure}

\soln A warning will be issued whenever the order of things is not

  as expected.  Similarly, when using \clsname, you will receive a

  warning if you have an index, or if you lack a \namepart{1}, an

  \namepart{2}, a \namepart{5}, or a

  \namepart{9}.\footnote{I hope this doesn't intrude into your

  private life too much.}  However, you will still get your document

  processed.  The names in Figure~\ref{fig:thesis-order} are used

  verbatim instead of their standard \LaTeX\

  counterparts.\footnote{In case you are wondering, \clsname\ does

  not provide anything for a \namepart{8}. You have to figure out

  how to do it yourself if you want one.}



\req `The title of the thesis should be given between

  $5$~and~$7\,\cm$ from the top of the page, followed by the names

  of the author and, after about a $5\,\cm$ space, a statement of

  the the degree for which the thesis is submitted:\bigskip



  {\sffamily

  AN INVESTIGATION INTO THE EFFECTS\\

  OF SELECTION WITHIN AN INBREEDING\\

  PROGRAMME IN SUNFLOWER\medskip



  by\medskip



  MARTIN TREVOR SMITH\bigskip\bigskip



  A thesis submitted to\\

  The University of Birmingham\\

  for the degree of\\

  DOCTOR OF PHILOSPHY\bigskip}



  `The bottom right-hand corner should state school or department,

  university and year of submission, with each element being on a

  separate line:



  [\ldots]\bigskip



  {\sffamily

  \begin{tabular}{l}

   School of Biosciences\\

   The University of Birmingham\\

   March~2000\bigskip

  \end{tabular}}



  `Give your full name on the title page, as it will appear on the

  degree congregation list.'\footnote{The \textsf{sans serif}s are

  mine.}~\cite[Section~3.2.1]{booklet:prethesis}

\soln The command \cmd{maketitle} produces a title page similar to

  the one described above.  The top margin is meant to be

  $6\,\cm$, but it turns out to be a little wider for some unknown

  reason.  The contents of the title page can be changed using the

  usual commands \cmd{author\textarg}, \cmd{title\textarg} and

  \cmd{date{\textarg}} in \LaTeX, as well as those mentioned in

  Section~\ref{sec:add-cmds}.



\req The abstract, dedication (if any) and acknowledgements

  should not be recorded in the table of contents.

\soln It will actually take me extra effort to list them in the

  table of contents.  So of course I didn't do it.  The

  requirements only require the acknowledgements to be on a

  separate page, but the three environments \env{abstract},

  \env{dedication} and \env{acknowledgements} each start a

  new page in \clsname.



\req `A table of contents, and if needed a list of illustrations

  and a list of tables, should always be included[. \ldots]  The

  table of contents should show chapter and section titles (if

  any), demonstrating the relationship of the parts to each other

  by (if appropriate) indentation and numbering. Chapters and

  sections should be referenced to their page numbers.



  `[\ldots]  A page number should follow the title of the

  illustration.  [\ldots]'~\cite[Sections~3.2.5.1

  and~3.2.5.2]{booklet:prethesis}

\soln I hope that \LaTeX's default configurations in

  \cmd{tableofcontents}, \cmd{listoffigures} and

  \cmd{listoftables} satisfy their demands.



\req `Footnotes are generally indicated by small superscript

  numbers placed at the end of a sentence.  Numbering begins afresh

  on each new page.'~\cite[Section~3.3.2.1]{booklet:prethesis}

\soln Footnotes are naturally indicated by small superscripts in

  \LaTeX.  Clearly, \clsname\ can't force you to place footnotes

  at the end of a sentence.  Starting footnote numbering on each

  new page turns out to be hard for \TeX\ and \LaTeX.  The package

  \pkg{perpage} is loaded to do this job.\footnote{It turns out

  that \TeX\ doesn't know on which page a footnote will be placed

  when it is read.  It seems that the only way to get round this

  is to record the information in an external file.  (Can you do

  better?)  This is what is done by \pkg{perpage}.  So there is no

  point for me to write this same thing into \clsname\ again.}



\req `Appendices should be listed on the contents page. Where

  more than one appendix is included, assign each one a number and

  list them like chapters.'~\cite[Section~3.4.1]{booklet:prethesis}

\soln As in usual \LaTeX, \cmd{appendix} signals the start of

  appendices.  Use \cmd{chapter\textarg} for a numbered appendix,

  and \cmd{chapter*\textarg} for an unnumbered

  appendix.  Unlike in usual \LaTeX, both of these are listed in

  the table of contents.  This does not affect

  \cmd{chapter*\textarg} outside the appendix.



\req `\,``List of References'' is often the preferred heading for

  introducing [a bibliography] for a thesis in the

  Sciences[\ldots]'~\cite[Section~3.4.2]{booklet:prethesis}

\soln The name of the bibliography is changed to `\bibname'.

\end{itemize}



May I reiterate that the above is only a subjective selection of

materials from the official document~\cite{booklet:prethesis}.  I

paid special attention to the mathematics and typesetting related

parts.



\section{The list of references}

The official document~\cite{booklet:prequote} only described the

requirements for the bibliography using examples.  The

following are all I can spot in the official document, which is

slightly inconsistent itself.  Since I am not English, probably

some parts do not make sense, and your help is very appreciated in

identifying them.

\begin{itemize}

\item The Harvard style is used, i.e., a reference label contains

  the names of the authors or editors, and the publication year

  (unless otherwise specified).

\item The names are all abbreviated, with the last names first.

  Items with more than three authors or editors are truncated

  using `et al'.  There is no space between the initials.

\item Titles of books, journals, proceedings volumes, and theses

  are in boldface, while titles of proceedings articles and book

  chapters are put in double quotation marks.  All other titles

  are left normal.

\item The publisher information is written in the form `Address:

  Publisher'.

\item The word `In', wherever it is used, is underlined, and there

  is no punctuation added after it.

\item The edition stays close to the book title, and the series

  name and publisher information (in this order) follow.

\item A period is inserted before chapter numbers or page numbers,

  and the letter following this period is not capitalized.

\item A period is added to the end of every item, \emph{except}

  those ending with a page number.

\item Access dates for URLs are mandatory.  (You will receive a

  warning from \BibTeX\ if this is missing.)

\end{itemize}



Since the number of situations that a reference can be in is

bigger than the largest number I can comprehend at any one time,

you may find \clsname\texttt.\ext{bst} doing something weird to

your references.  If this is so, please report it to me so that I

can try to improve.





\chapter{The Fill-in Mode}\label{ch:fill-in}

Probably this isn't a good name, but I will stick to it before

some of you suggest a better name to me.  We say that you are in

\emph{fill-in mode} if and only if you are using

\file{bhamthesisfill.sty} but not \file{bhamthesis.cls}.  This

most likely occurs when you decide to pause using \clsname\ for a

short while.  The fill-in mode provides a \emph{minimal} meaning

of most new commands used in \clsname.  It can't do anything

serious for you, but probably it will be useful when you want to

print the first drafts of your thesis and try to be nice to the

environment at the same time.



As mentioned in Chapter~\ref{ch:usage}, the fill-in mode is

chiefly designed for the standard \LaTeX\ document classes

\pkg{book}, \pkg{report} and \pkg{article}.  This may seem a

little strange because \pkg{article} is quite different from the

other two: it does not have chapters.  Actually, when used with

the \pkg{article} class, the fill-in mode reinterprets your

sectioning commands: a chapter is viewed as a section, a section

is viewed as subsection, and a subsection is viewed as a

subsubsection.  This has to stop somewhere and so a subsubsection

is viewed as a mutated subsubsection.  However, beware that they

\emph{use the same counter} and \emph{appear the same in the table

of contents}.  I told you it can't do anything serious, but it

should be acceptable if you don't have subsubsections originally.

In fact, the fill-in mode may also be useful when you want to

switch from \pkg{book} or \pkg{report} to \pkg{article}

temporarily.



Certain parameters and internal commands defined in \clsname\ are

missing in the fill-in mode.  I am afraid you have to live with

it.  On the other hand, there is no harm in loading

\file{bhamthesisfill.sty} when using \clsname: it will simply be

ignored.  So probably it is safe to have

\begin{ttlist}

 \verb|\usepackage{bhamthesisfill}|

\end{ttlist}

in your preamble whenever you use \clsname.



There is a final, not so important, use of the fill-in mode.  When

you think that something is wrong in \clsname, run fill-in mode

with the standard \pkg{report} class before reporting it to me. If

it turns out to be the same, then it's probably because \LaTeX\

chose to do it that way, not me.



The fill-in mode can actually be very much improved, but it is not

the main thing I want to make and so I reduced the amount of time

in writing it to the minimum.





\chapter{Beware}\label{ch:beware}

It is almost certain that there are limitations to what \clsname\

and I can do.  Beside the ones mentioned previously in the text,

Section~\ref{sec:pts2note} discusses a few side effects of using

\clsname.  Some tricks are suggested to get round these wherever

available.  Section~\ref{sec:issues} contains problems that I

can't tackle.  If you have solutions, please do let me know.



\section{Points to note}\label{sec:pts2note}

\begin{enumerate}

\item Do \emph{not} load \pkg{perpage} and \pkg{setspace} in your

  document when you use \clsname.  The same is true for

  \pkg{harvard} if you are using the \clsname\ class \emph{without}

  the \opt{noharvard} option.



\item If for any reason you need lowercase letters in the title

  and/or the name, then you can do it by putting them into a

  command.  For example, the title of this document is set by

  putting something like

  \begin{ttlist}

   \verb|\title{The Unofficial \clsname\ Package\\ for \LaTeXe}| \\

   \verb|\newcommand{\clsname}{\textsf{bhamthesis}}|

  \end{ttlist}

  in the preamble.  Chapter titles need a little more effort.  You

  may need to issue an explicit \cmd{lowercase} command at places

  where you don't want letters to be capitalized.

\item It is possible to put \cmd{thanks\textarg} and

  \cmd{footnote\textarg} anywhere in the title page, but if you

  put them in \cmd{title} or \cmd{author}, then the

  footnote text will be capitalized. If you don't want that to

  happen, put footnote text in a command to protect them.

\item The school name cannot contain a line break at the moment.

  If you need more than one line, try stacking them up in a single

  box first.  I haven't tried it yet and don't know if it works.

\item Today's month and year is shown in the title page if

  \cmd{date\textarg} is not called.  If you need to show the

  day in the month, try using \cmd{number}\cmd{day} in your

  \cmd{date} definition.

\item Marginal notes are put in the (wider) inner margins instead

  of the more common practice of putting them in the outer

  margins.  As in standard \LaTeX, you can force marginal notes to

  appear on the other side by using

  \cmd{reversemarginpar}.

\item Although the environments \env{figurepage} and

  \env{tablepage} \emph{try} to put your stuff on a separate

  page, they do \emph{not always} succeed.  Since the in-line

  floats are dealt with differently on the page, you will have to

  check each \env{figurepage} and \env{tablepage} to see if

  you are floating on a page on their own to ensure

  consistency.\footnote{It seems that there is no automated way to

  do this.  It is because whether a float is put on a separate

  page affects how the contents of the float are arranged (since

  the margins are changed), and how the arrangement of the

  contents of a float affects whether the float is put in a

  separate page (since the height may be changed).  So there is

  a possibility of going into an infinite loop.  Am I true?  If I

  am, what are we actually doing when we arrange them manually?

  \par  In reality, these environments are made as tall as

  \cmd{textheight}\ldots}

\item On the other hand, \env{figure}s and \env{table}s may

  sneak onto a separate page without your permission.  So you need

  to check them as well if you don't want that the happen.  You

  can also increase \cmd{floatpagefraction} so that only floats

  occupying at least that proportion of height on the page go on a

  separate page.  Note that more than one small floats may make up

  enough height to go onto a new page on their own.

  \cmd{floatpagefraction} is set to $0.5$ initially by \LaTeX.  You

  can increase it up to $0.99999237$, correct to $8$~decimal places,

  before it affects \env{figurepage} and \env{tablepage}.

\item You need to \LaTeX\ your \ext{tex} file at least twice to

  get the footnote numbering right in most cases, because

  \pkg{perpage} needs to write to and read from the \ext{aux}

  file.

\item Be careful when choosing \bfield{abbrvname} for your

  \BibTeX\ items.  They are not used to sort the list of

  references, and it may be hard for the reader to find it in the

  list.  There is also chance that you may end up with two items

  with the same citation label!

\item As mentioned in Chapter~\ref{ch:cmd+param},

  \bfield{abbrvname} is not shown the first time the item is

  cited.  This is the built-in behaviour of \pkg{harvard} and has

  nothing to do with me.  If you don't want this to happen (and if

  you are using version~2.0.5 of \pkg{harvard}), then you can

  change this behaviour by adding the lines

  \begingroup

   \footnotesize

   \begin{ttlist}

    \cmd{makeatletter}\\

    \verb| \renewcommand{\cite}{\@ifstar{\@ifstar{\HAR@acite}{\HAR@fcite}}{\HAR@acite}}|\\

    \cmd{makeatother}

   \end{ttlist}

  \endgroup

  in the preamble (after loading \pkg{harvard} if you are

  including it separately).

\end{enumerate}



\section{Known issues}\label{sec:issues}

\begin{enumerate}

\item The proportion given by \cmd{dedabove} and \cmd{dedbelow}

  does not seem to be the correct ratio appearing in the

  dedication page.  The top part constantly get more space,

  possibly because of margin problems.  So if you want to place

  your text according to the golden ratio, say, then you will have

  to do it by trial-and-error.

\item There is too much space around environments like

  \env{quote}, \env{itemize}, \env{center}, etc.  I will

  leave this for the authors of \pkg{setspace} to sort out.  You

  can also fix them manually.

\item \LaTeX\ does not know how to break URLs into separate lines.

  As usual, the \pkg{url} package helps solve this problem.

  However, this introduces another problem when the an URL appears

  in the bibliography.  \BibTeX\ breaks a line automatically when

  it becomes too long.  In the process, it adds an \texttt\% at

  the end of the first line (or few lines) so that the line breaks

  are ignored by \LaTeX.  Unfortunately, when this happens at an

  URL typeset using the \pkg{url} package, this \texttt\% is not

  ignored because it is seen as a part of the URL, causing an extra

  \texttt\% to appear.  I see no way to get round this problem.

  If you don't mind getting your hands dirty, you can fix this

  manually in the \ext{bbl} file \emph{every time} you run

  \BibTeX.

\item If you downloaded the \clsname\ package from the

  Internet, say from the \textsf{Postgraduate Society Website},

  then there may be plenty of blank lines added into your files

  during the upload/download process.  These extra blank lines

  cause a huge load of problems.  If you wish to check whether

  your copy is affected, then simply check whether the file

  \file{bhamthesis.cls} contains any blank lines --- it should

  \emph{not} contain a single blank line.  If this happens to you,

  try using \texttt{lynx} to open the webpage instead.  This seems

  to work fine on \texttt{babbage}.

\item As reported by Eligio Cerval-Pe\~na, there seems to be a

 warning appearing when using \clsname\ with

 \begin{ttlist}

  pdfTeX, Version 3.141592-1.40.4 (MiKTeX 2.6)

 \end{ttlist}

 dated \texttt{2007.11.28}.  The message looks something like

 \begin{center}\begin{minipage}{.86\textwidth}\small

 \begin{verbatim}

pdfTeX warning (ext4): destination with the same identifier (name

{page.i}) has been already used, duplicate ignored

<to be read again>

                   \relax

l.28 \end{abstract}\end{verbatim}

 \end{minipage}\end{center}

 I have no idea what caused this problem, let alone what to do.

 If you know what the problem is, please do let me know.

 Fortunately, this doesn't stop pdf\TeX\ from running, and the

 output looks fine.  Moreover, there isn't such a problem with

 \LaTeX\ as far as I am aware of.  Alternatively, use

 \texttt{dvi2pdf}.

\item If you happen to have the \pkg{html} package on your

 installation of \LaTeX, then it is automatically loaded by

 \pkg{harvard}.  This causes some unnecessary hyperlinks

 potentially, but these should not appear on the pages when

 printed.  There should be no such problem if you don't have

 \pkg{html} put in a place where \LaTeX\ can find.

\end{enumerate}





\appendix

\chapter*{Technical Comments}

This chapter contains some technical notes that may be useful to

those who want to modify \clsname.

\begin{enumerate}

\item \clsname\ is not supposed to work under the \LaTeXe\

  Compatibility Mode emulating \LaTeX~2.09.

\item No effort has been made to conform to the standard \LaTeX\

  convention.  In particular, \cmd{textheight} is \emph{not} an

  integral multiple of \cmd{baselineskip}.

\item If you use other packages to tweak floats, then you may

  experience compatibility problems with the \env{figurepage}

  and \env{tablepage} environments.

\item Is it possible to provide the definitions of, say,

  \cmd{harvardcite}, after the preamble, but before the document

  loads the \ext{aux} file, by the package

  \file{bhamthesisfill.sty} automatically?  The

  \cmd{AtBeginDocument} command does not work.

\item Is it possible to check if the bibliography exist?  The

  method that works for other parts, i.e., by using

  \cmd{AtEndDocument}, does not work here.

\item There is a Boolean variable that checks whether an abstract

  exists.  Why is \cmd{global} needed there but not in anywhere

  else, like in the table of contents?

\end{enumerate}



\backmatter

\bibliographystyle{bhamthesis}

\bibliography{man}

\end{document}

%%

%% End of `bhamthesisman.tex'.

