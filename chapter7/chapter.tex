%*******************************************************************************
%******************************* Seventh Chapter *******************************
%*******************************************************************************
\chapter{Conclusions and future work}
%*******************************************************************************

\section{Conclusions}
In this thesis, we went from the use of RRSs with UTDE, RPs to 
RQA metrics in order to realise
that one should compute and then select the appropriate embedding parameters 
before using any of the tools for nonlinear analyses, however the process of 
selection for appropriate embedding parameters can be laborious 
(see Chapter \ref{chapter3}) and in addition to that it is still an 
open problem \citep{gomezgarcia2014}.
We then found the work of \citep{iwanski1998} in which is stated that 
patterns in RPs and metrics for RQA are independent of embedding 
dimension parameters. However, that is not the case when using different 
recurrence thresholds. 
Hence, embedded parameters and recurrence thresholds were considered 
to create 3D surfaces of RQA which, I hypothesise, might be a better approach
to provide understanding on the dynamics of different characteristic of 
time series 
such as window size length, participants, sensors and levels of smoothness
(see weaknesses and strengths of RQA in Chapters \ref{chapter5} and 
\ref{chapter6}).

To my knowledge, I can conclude that, no scientific work has been reported 
regarding nonlinear analyses (e.g. RSSs with UTDE, RPs and RQAs) to 
quantify movement variability in the context of human-humanoid interaction.
Additionally, this thesis has explored the weaknesses and strengths of RQA
using 3D surfaces of the variation of embedding parameters and recurrence
thresholds such approach requires little parametrization.
We then found that the 3D surfaces for RQA ENTR metric can be used 
to model any of the effects of movement variability of the activities 
as well as the post processing of real-world time series data 
which have different window size length,
smoothness and structures (shape, amplitude and phase) of time series
that correspond to different activities or different participants
(see Sections of weaknesses and strengths of RQA in Chapters 
\ref{chapter5} and \ref{chapter6}).
Hence, in the following sections, we will point out the positives
and negatives of this thesis by answering the raised research questions 
in Chapter \ref{chapter1}.





\subsection*{What are the effects 
	of different parameters for nonlinear analyses 
	with different characteristics of time series?}

It is evident that time series from different sources 
of time series (e.g. participants, movements, axis type, window size lengths 
or levels of smoothness) will present differences for not only
embedding parameters but also for the patterns in RSSs, RP, RQAs 
and 3D surfaces of RQA metrics.
With that in mind, we conclude that the selection of the appropriate
embedding parameters or recurrence threshold is crucial to get 
meaningful results from nonlinear analyses tools. However, in this 
thesis we find that the creation of 3D surfaces of RQA metrics is 
a new approach that is independent of either the type of time series
or the selection of parameters.


\subsection*{What are the weaknesses and strengths of RQA metrics 
	when quantifying MV?}


After getting results from our experiments, we can state that the 
weaknesses of RQA are (i) the requirement of an expert(s) for 
interpretation and computation of embedding parameters and 
recurrence thresholds,  
(ii) the implementation and computation of methods is 
laborious and algorithms and the computation of those parameters 
is still an open problem, and 
(iii) the selection of particular parameters to apply nonlinear analyses 
does not necessary give the best representation of the dynamics of 
the time series.

With regards to the strengths of RQA metrics, we realised that varying 
embedded parameters and recurrence thresholds to create 3D surfaces 
of RQA might be: 
(i) a better approach to understand the dynamics of different 
characteristic of time series and
(ii) require little set up of parametrisation.


\subsection*{How the smoothing of raw time series affects the 
	nonlinear analyses when quantifying MV?}

The answer depends on what to measure, for instance, to 
avoid erratic changes in the metrics, smoothing raw signals can help
to obtain well defined and constant metrics.
However, we observed that the increase of level of smoothness 
of time series created more complex trajectories in the RSSs and 
added more black dots in Recurrence Plots 
(see RSSs and RPs sections in Chapters \ref{chapter5} and \ref{chapter6}).
It is important to mention that some RQA's metrics (e.g. DET and  ENTR) 
are more robust to the effect of smoothness of time series.

Also, using raw data, one might create a closer representation of the 
nature of movement variability.
However, we state that further investigation is required to be done
as we think that finding the right balance between 
smoothness and the raw data to capture movement variability is 
a still a problem.


%\newpage
\section{Future work}

\subsection*{Inertial sensors}
To have fundamental understating of the nature of signals collected 
through inertial sensor in the context of human-robot interaction,
future experiments can be conducted considering the application of 
derivates to the acceleration data. 
With that in mind, it can then be explored 
(i) the jerkiness of movements and therefore the nature of arm movements 
which typically have minimum jerk \citep{flash1985},
(ii) the relationship with different body parts, for instance, 
how rapid or slowly one perform arm and legs movements as we grow up 
\citep{devries1982, mori2012} or (iii) 
the application, to real-world time series data, of higher derivatives 
of displacement with respect time such as jounce, snap, 
crackle and pop \citep{eager2016}.


\subsection*{Nonlinear analyses}

\subsubsection*{Optimal embedding parameters}
When using the method of False Nearest Neighbour \citep{Cao1997}, where 
the values for $E_1(m)$ stop changing to find the minimum embedding 
dimension, a threshold should be defined in order to obtain the minimum 
embedding dimension $m_0$. Hence, a further investigation is required 
for the selection of the threshold in the $E_1(m)$, as the selection 
of the threshold in this thesis is only based on no particular method 
but visual inspection of the $E_1(m)$ curves
(see Section \ref{ch3:fnn} in Chapter \ref{chapter3}).
Similarly, further research is required to be done with regards to the 
selection of the minimum delay embedding because it is not clear:
(i) why the choose of the first minimum of the AMI is the minimum delay 
embedding parameter \citep{kantz2003} or 
(ii) why the probability distribution of the AMI function 
is computed with the use of histograms which depends on a heuristic 
choice of number of bins for the AMI partitioning \citep{garcia2005e71}. 
Additionally, "the AMI method is proposed for two dimensional 
reconstructions and extended to be used in a multidimensional case 
which is not necessarily hold in higher dimensions" 
\citep[p. 156]{gomezgarcia2014}.


\subsubsection*{Other methodologies for state space reconstruction.}
In addition to the method of Uniform Time-Delay Embedding to reconstruct
state spaces, other methods have been investigated stating a better 
dynamic representations of time series in the reconstructed state spaces:
(i) the nonuniform time-delay embedding methodology  
where the consecutive delayed copies of $\{ \boldsymbol{x}_n  \} $ are not
equidistant
\citep{pecora2007, uzal2011, 
Quintana-Duque2012, Quintana-Duque2013, Quintana-Duque2016}, and
(ii) uniform 2 time-delay embedding method which takes advantage 
of finding an embedding window instead of the traditional method 
of finding the embedding parameters separately \citep{gomezgarcia2014}.


\subsubsection*{RQA parameters}
Having presented our results with RQA metrics, we consider that further 
investigation is required to be done in order to have a better 
understanding of the RQA metrics and ensure its robustness in many of 
its applications. 
For example, \cite{marwan2007} and 
\cite{marwan2015} reviewed different aspects to 
compute RPs using different criteria for (i) neighbours, 
(ii) different norms ( $L_{1-norm}$, $L_{2-norm}$, or $L_{\infty-norm}$ ) or 
(iii) different methods to select the recurrence thresholds such as: 
using only certain percentage of the signal
($\sqrt{m_0} \times$ 10\% of the fluctuations of the time series)
\citep{letellier2006}, selecting a determined amount of noise, and 
using a factor based on the standard deviation of the 
observational noise \cite{marwan2007}.



\subsubsection*{Robustness of Entropy measures with RQA}
Further investigation is required to be done with regards to
the use of Shannon entropy with recurrence plots.
For example, \cite{letellier2006} investigated the robustest of 
the Shannon entropy based on line segments distributions of 
recurrence plots $S_{RP}$ 
against the Shannon entropy based on system dynamics $S_{SD}$.
With that, \cite{letellier2006} pointed out that Shannon entropy based on 
recurrence plots has strong dependency with the choice of observable,
variable of the dynamical system, while Shannon entropy based on system dynamics 
is more robust to noise-contaminated signals.
Recently, \cite{corso2017} tackled the quite know problem of ENTR with RQA
in a logistic map \citep{marwan2007} where ENTR values decrease despite the 
increase of nonlineariry by introduction the use of microstates. 
Additionally, \cite{corso2017} presented the robustness of his method 
with changes to recurrence thresholds.


\subsubsection*{Advanced RQA quantifications}
In addition to the applied RQA metrics (REC, RATIO, DET and ENTR) for 
recurrence quantification, I can see different lines of research for this 
thesis which is the investigation of methods of the RP based on 
complex networks statics, calculation of dynamic invariants, 
study of the intermittency in the systems, 
applying different windowing techniques or 
the study of bivariate recurrence analysis for correlations, 
couplings, coupling directions or synchronisation between dynamical systems
\citep{marwan2007, marwan2015}.


\subsection*{Variability in perception of velocity}
While conducting the experiments with different arm movements 
velocities (e.g. normal and faster), we realised that participants 
perceive velocity differently, shedding light for a have better 
understanding on why each participant perceive body movement velocity 
differently and how to quantify such variability of perception of movement.
For instance, from our experiments we realised that some participants 
considered a normal velocity movement as a slow velocity movement and 
some others considered a slow velocity movement as being performed 
in normal velocity. 
With that in mind, we hypothesise that the differences in perception of 
velocities are related to the background of each person,
personality traits or even to some kind of movement experience 
(in music or sports) that make them more aware of their body 
movements. 

\subsection*{A richer dataset of real-world time series}
It should also be highlighted that the experiments for this thesis are 
limited to twenty three healthy right-handed participants of a 
range age of mean 19.8 and SD=1.39, for which participants of 
different ages, state of health and anthropomorphic features 
would create a richer dataset of real-world time series.


\subsection*{Applications}
I can foresee many of the published work regarding modeling human movement 
variability applied to the context of human-humanoid interaction.
For instance, implementing nonlinear analyses algorithms in humanoid robots 
to evaluate the improvement of movement performances \citep{muller2004}, 
to quantify and provide feedback of level skillfulness as a function 
of movement variability \citep{seifert2011} or to quantify movement 
adaptations, pathologies and skill learning 
\citep{preatoni2007, preatoni2010, preatoni2013}.
More specifically in the context of human-humanoid rehabilitation 
where little work has been done \citep{gorer2013, guneysu2015}
with regards to the use of nonlinear analyses and therefore 
provide adequate metrics to quantify and provide feedback 
for movement variability. 



