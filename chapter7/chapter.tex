%*******************************************************************************
%****************************** Sixth Chapter *********************************
%*******************************************************************************

\chapter{Conclusion}

%% **************************** Define Graphics Path **************************
%\ifpdf
%    \graphicspath{{chapter7/figs/raster/}{chapter7/figs/PDF/}{chapter7/figs/}}
%\else
%    \graphicspath{{chapter7/figs/vector/}{chapter7/figs/}}
%\fi
%


%%**************************** %Broad Purpose  **********************************
%\section*{Summary and broad purpose of the chapter}
%* How long (number of words)?
%* Deadline
%* What have you got?


\section{Discussion}


\subsection{ami}


Further investigation is required for the selection of the threshold 
in the $E_1(m)$, as the selection of the threshold in this work is base on 
no particular method but visual inspection.

Additionally, we observe two additional negative aspects of the use of the 
False Nearest Neighbour method \cite{Cao1997}: 
(i) %the main requirements for the use of Cao's method are 
from the input time series, a value to set up the maximum embedding dimension 
and delay embedding, Cao's algorithm computes $E_1(m)$ and $E_2(m)$, however
when the values for $E_1(m)$ stop changing, a threshold should be 
defined in order to obtain the minimum embedding dimension $m_0$, and
(ii) Cao's method computes different embedding dimensions 
when using different window lengths for the time series as shown in chapter 
\ref{chapter5} and \ref{chapter6}.



\subsection{other}

It is evidently that time series from different sources (participants, movements, axis type, 
window length or levels of smoothness) presents visual differences for 
embedding parameters and therefore for RRSs. For which, the selection of embedding parameters 
was our first challenge where we computed embedding parameters for each time series and 
then computed a sample mean over all time series 
in order to get two embedding parameters to compute all RSSs with its corresponded type of movement.
Then we found that the quantification of variability with regard to the shape 
of the trajectories in RSSs requires more investigation since our original proposed method 
base on euclidean metric failed to quantify those trajectories. Specially, for trajectories 
which were not well unfolded. With that in mind, we proceed to take advantage of 
four RQA metrics (REC, DET, RATIO and ENTR) in order to avoid any subjective interpretations 
or personal bias with regard to the evolution of the trajectories in RSSs.


\subsection{RQA metrics with fixed parameters}
Considering that RQA metrics were computed with fixed embedding parameters 
($m=6$ and $\tau=8$) and recurrence thresholds ($\epsilon=1$), we found the following.
REC values, which represents the \% of black points in the RPs, were more affected with 
and increase in normal speed movements (HN and VN) than faster movements (HF and VF)  
for the sensor attached to the participants (HS01). Such decrease of REC values 
from normal speed to faster speed movements is also presented in data from sensor 
attached to the robot (RS01), and little can be said with regard to the dynamics
of the time series coming from RS01. 
Similarly, DET values, representing predictability and organisation in the RPs, 
present little variation in the any of the time series where little can be said.
In contrast, RATIO values, which represent dynamic transitions, were more 
variable for faster movements (HF and VF) than normal speed movements (HN and VN) 
with sensors attached to the participants (HS01).
For data coming from sensors attached to the robot (RS01), RATIO values
from horizontal movements (HN, HF) appear to vary more than 
values coming from vertical movmentes (VN, VF).
With that, it can be said that RATIO values can represent a bit better
than REC or DET metrics for the variability of imitation activities in each of the conditions
for time series.
In the same way, ENTR values for HN were higher than values for HF
and ENTR values varied more for sensor attached to participants 
than ENTR values for sensors of the robot. It is evidently that 
the higher the entropy the more complex the dynamics are, 
however, ENTR values for HN appear a bit higher than HF values, 
for which we believe this happens because of the structure the time series
which appear more complex for HN than  HF movements which presented a 
more consistence repetition.

We observed that some RQA metrics are affected by the smoothness of data.
For which, we also explored the effect of smoothness of raw-normalised data 
where, for example, REC and DET values were not affected by the smoothness 
of data since these seemed to be constants. However, for RATIO values, 
the effect of smoothness can be noticed with a slight decrease of amplitude 
in any of the time series conditions which is also presented with ENTR values.


\subsection{RQA metrics with different parameters}
Patterns in RPs and metrics for RQA are independent of embedding dimension parameters \cite{iwanski1998},
however, that is not the case when using different recurrence thresholds. Such changes of 
recurrence threshold values can modify the patterns in RPs and therefore the values of RQA metrics. 
We therefore computed 3D surfaces to explore the sensibility and robustness of 
embedding parameters and recurrence threshold in RQA  metrics. 
Following the same methodology of computing 3D surfaces, we also considered variation 
of window length size to present RQA metrics dependencies with embedding parameters, 
recurrence thresholds and window length size.





\section{Conclusions}
We conclude that using a different level of smoothness for time series help us 
to visualise and to quantify the variation of movements between participants 
using RSSs, RPs and RQA. Also, it is important to mention that some RQA's metrics 
(e.g. DET and  ENTR) are more robust to the effect of smoothness of time series.

Althought our our experiment is limited to twenty healthy right-handed participants 
of a range age of mean 19.8 SD=1.39, RQA metrics can quantify human movement 
variability. With that in mind, we conclude that qunatification of human-humanoid
imitation activities is possible for participants of different ages, state of health 
and anthropomorphic features.


%
%Such understanding and measurement of movement variability using
%cheap wearable inertial sensors lead us to have a more intuitive selection of parameters
%to reconstruct the state spaces and to create meaningful interpretations
%of the recurrence plots and the results of the metrics with recurrence quantification 
%analysis. 
%

In general, activity type, window length and structure of the time series 
affects the values of the metrics of RQA for which certain RQA metrics 
are better to describe determined type of movement.
Using determined RQA metrics depends on what one want to quantify, for instance, 
one can find predicability, organisation of the RPs, dynamics transitions, 
or complexity and determinism.


%
%However, we believe that further investigation is required to find the 
%right balance between the level of smoothness of the signal
%and defining what is the aim of RQA where 
%its representations using RSS, RP and RQA.
%Specially, where the level of smoothness does not affect 
%the variation of each of the movements quantification. 
%%We can also conclude that finding the right balance between 
%%smoothness and the raw data to capture movement variability is 
%%a still a problem that has many avenues for exploration.
%

Similarly, such differences in time series created differences in each of the 
RQA metrics, for instance, RATIO and ENTR are helpful to distinguish 
differences in any of the categories of the time series (sensor, activity, 
level of smoothness and number of participant), however for certain time series 
(data from the sensor attached to the robot) seemed to have little variations 
between each of the participants. The latter phenomena was in a way evidently
as robot degrees of freedom did not allow it to move with a wide range of variability. 





\section{Future Work}

\subsection{Inertial Sensors}
To have more fundamental understating of nature of signals collected through 
inertial sensors in the context of human-robot interaction, we are considering to apply 
derivates to the acceleration data. We can then explore the jerkiness of movements 
and therefore the nature of arm movements which typically have minimum jerk \cite{flash1985},
its relationship with different body parts \cite{devries1982, mori2012} or
the application of higher derivatives of displacement with respect time 
such as snap, crackle and pop \cite{eager2016}.



\subsection{RQA}
Having presented our results with RQA metrics, we believe that further investigation 
is required to have more robust metrics. For example, Marwan et al. \cite{marwan2007, marwan2015} 
reviewed different aspects to compute RPs using different criteria for neighbours, 
different norms ( $L_{1-norm}$, $L_{2-norm}$, or $L_{\infty-norm}$ ) or 
different methods to select the recurrence threshold $\epsilon$, such as using certain percentage 
of the signal \cite{letellier2006}, the amount of noise or using a factor based 
on the standard deviation of the observational noise among many others \cite{marwan2007}.











%%**************************** %Summary of the Argument  **********************************
%\section[Conclusion]{Summary of Argument}
%
%\section[Future Work]{Implications}
%

In this work, an experiment is performed in the context of human-humanoid imitation activity
to test nonlinear dynamics methods to quantify human movement variability. 
The presented results that illustrate the potential of nonlinear dynamics tools 
by providing a balanced review of positive and negatives aspects of each 
technique to quantitatively and qualitatively measure movement variability.

With regards to the visual inspection and understanding of the patters
for reconstructed state spaces and recurrence plots,
it can also be concluded that the performance of such tools is subjective 
since biased personal data interpretation might be provided.
Hence, without any bias, RQA metrics (REC, DET, RATIO and ENTR) help us to show 
such differences of movement variability for difference categories 
of the time series (participants, movements, axis type, window length or levels of smoothness).
Furthermore, it was noticed that each of the metrics of RQA show the differences 
but particularly the metrics of RATIO and ENTR are helpful to distinguish 
the differences in each of the categories of the time series.
%We therefore collecte data series using inertial sensors and we smooth
%the signals to see its graphical effects and also in the metrics.
%With that in mind, embedding values to create the reconstructe state spaces
%and theferfore its recurrence plots. By doing that, we saw that visually
%each of the particpants in each of the conditions for movemetns and 
%levels of smootheness of the signlas are evicently differently, therefore
%the challenge for us were to have a quantitavely method for each of the 
%difference in the patterns for RP for which we took the adnavce of the use 
%of Recurrene Qantification Analysis which show important results 
%that help us to quantify the variality between particpants and between 
%movments. We esentially noticed that each of the metrics of RQA
%(REC, DET, RATIO and ENTR) can shown differences but specially RATIO
%and ENTR are helpful to distinguis the differences in each of the movments.
%%Tue 19 Jun 12:21:56 BST 2018

Although RPs and metrics for RQA are independent of embedding dimension \cite{iwanski1998},
it was found that recurrence threshold values can modify the results for both RPs and RQA.
This work has carried out experiments on different activities in which 
the sensor's axis was representative of difference of structures in the time series.
Hence, further investigation is required.
In a similar experiment carried out by \cite{letellier2006}, an equation was defined following 
several trials for the selection of recurrence threshold. 
This equation was used to determine the recurrence plot 
($\sqrt{m_0} \times$ 10\% of the fluctuations of the time series).



%With that in mind we did the same for the selection 
%of the right threshold for our particular problem where we first select 
%a threhold of 1 for all the signals however, these value is not
%appropriate for both activites, as for example the HN and HF 
%show sometimes white RP which should not be the case if 
%we select a right emedding threshodl.
%However, we have found the following problems
%% ENT
%For RQA the window effect is crucial for each of the metrics
%so for example, ENT values for HN are higher and HF are lower
%meaming that these are less comples than HN values, we believe
%that happens because of the lnght windows. Althought,
%these values apper to be more complex for HF, 
%the values in HN have less repetutions of tmovments 
%which is the reason to have those differences in ENT values.
%%#added Mon 18 Jun 12:32:55 BST 2018
%For, RATIO values apper to have more variations across particpants
%for which we believe that RATIO values represent a bit better
%the variatbiltuy of imitation activitivies and also the 
%movment variaiblaituy that is created in this experiemnt.
%%added: Mon 18 Jun 14:06:19 BST 2018
%and for ENTR values show little variation across partipants and these
%are generally higher for HN than HF movmentes in each of the 
%smothed signals. We believe that ENTR values are affected
%by the windown lenght so as to say that HN values 
%represent more complex structures than those coming from 
%HF values.
%%added: Mon 18 Jun 14:17:27 BST 2018

It was also found that using a different levels of smoothness
for time series helps to visualise the variations of movements 
between participants using RSSs, RPs and RQA. Also, it is important to mention 
that some RQA's metrics (e.g. DET and  ENTR) are more robust 
to the effect of smoothness of time series.
However, we believe that further investigation is required to find the 
right balance between the level of smoothness of the signal
and its representations using RSS, RP and RQA.
Particularly, where the level of smoothness does not affect 
the variation of each of the movements' quantification.
%which is similar to the raw normalised data. Also the raw normalised
%data can show more details information of for the movement, however
%these are not lost as the time series is smoothed.
%We can also conclude that finding the right balance between 
%smoothness and the raw data to capture movement variability is 
%a still a problem that has many avenues for exploration.
%
%other variables affect the data from the sensors such as  
%temperature of the sensor processing, variation of the sampling
%rate which esentially are involved in the quqlity of the data.
% 
%We only create two to four levels of smoothness using the 
%savtiktzy-golay filter with same degree for different filter lenght
%for which we were able to see the differences in each of the 
%nonlinear dyamics tools. 
%We also propose four window size lenght  to see the effects
%in the nonlinear dynamics tools.
%%add more conclusions about the reuslts of the changes in rss, rp and rqa
%%as these parameters chagne.
%\subsection{Perception of speed}

It is important to mention that while performing the experiments 
with different arm movements speeds (e.g. normal and faster), it was 
realised that participants perceive speed in different ways.
For instance, some participants considered a normal speed movement 
as slow speed movement and some others considered a 
slow speed movement as being performed in normal speed. 
That sheds light of the need for future work to understand 
how each participant perceive body movement speed differently.
%which we hipothese that the differences in perception of speed
%are related to the background of each person, for exmaple,
%person who have receive musical training in their infancy
%are more aware of their body movemnets [add reference]
%It would also be intersting to ask participants
%to move in three different speeds withouth any aconstration in order to 
%capture the natural movements for slow, normal and faster speed 
%arm  movements.
It should also be highlighted that the experiment is limited to twenty healthy 
right-handed participants of an age range of mean 19.8 SD=1.39, 
for which participants of different ages, state of health and anthropomorphic features
would create more richness in the dataset of time series.
%to understand and quantify movement variability using nonlinear dynamics tools.
%With regard to the humanoid movements, I realise that
%it is imporant to program the robot with using the same speed
%and also considering movements that the robot can perform, 
%so for example when the speed is a bit higher the 
%robot's movements tend to be jerky.
%Also create uniform speeds of each of the movments, 
%I realise the the horizontal and vertical movements in 
%normal and faster speed were different which might 
%affect the perception of people movements.
%

%FUTURE WORK

%INERTIAL SENSORS
Additionally, a more meaningful understating
of the nature of the signals collected with inertial sensors
is required in cases where by derivate the acceleration data,
jerk movements and its relationship with different 
body parts can be explored \cite{devries1982, mori2012} and its nature of 
arm movements considered to have typically minimum jerk \cite{flash1985}.
%THIS IS MEANINGLESS AND DOES NOT FIT IN THIS PARAGRAPH AT ALL
% which can interpreted as the change in acceleration
%or jounce which is the derivatives of jerk and it is know to measure 
%the change of jerk.
%%%%Tue  8 May 17:44:36 BST 2018
%
%More interesting problems can be explored as the relationshop 
%for how rapid or slowly for arm and legs we perform as we grow up  
%\cite{devries1982, mori2012} and its nature o
%and in different activitues wher for example for human 
%reaching movmetnes, theser present typically minimum jerk 
%%\cite{Flash Hogan (1985).}
%and fundamentally how it can be qunatified using jerk and jounce.
%%added: Tue 19 Jun 13:16:02 BST 2018
%




%
%
%%%%%%%%%%%%%%%%%%%%%%%%%%%%%%%%
%%% Mon 14 May 13:15:33 BST 2018
%
%Having better understanding of RP, 
%requires to read the marvan2008: 
%%@ARTICLE{2008EPJST.164....3M,
%%   author = {{Marwan}, N.},
%%    title = "{A historical review of recurrence plots}",
%%  journal = {European Physical Journal Special Topics},
%%archivePrefix = "arXiv",
%%   eprint = {1709.09971},
%%     year = 2008,
%%    month = oct,
%%   volume = 164,
%%    pages = {3-12},
%%      doi = {10.1140/epjst/e2008-00829-1},
%%   adsurl = {http://adsabs.harvard.edu/abs/2008EPJST.164....3M},
%%  adsnote = {Provided by the SAO/NASA Astrophysics Data System}
%%}
%%
%
%
%


%%%%%%%%%%%%%%%%%%%%%%%%%%%%%%%
%%%Thu 10 May 13:10:45 BST 2018
%Considering the work of \cite{shoaib2016},
%futher experimets can be permoed with the combination of linear acceleartion
%n, which is obtained by removing acceleration due to 
%gravity from the accelerometer, with accelerometer and gyroscope.
%"the acceleration due to gravity is useful for differentiating static postures
%such as sitting and standing" but it is sensitive to changes in sensor orientation
%and body position.
%
%[Florentino-Liano, B.; O’Mahony, N.; Artés-Rodríguez, A. Human activity recognition using inertial sensors
%with invariance to sensor orientation. In Proceedings of the 2012 3rd IEEE International Workshop on
%Cognitive Information Processing (CIP), Baiona, Spain, 28–30 May 2012; pp. 1–6.]
%%



%\subsection*{Perception of speed}
%While performing the experiments in the human-humanoid imitation activies with 
%different arm movements speeds (e.g. normal and faster), we realised that 
%participants perceive speed in different way, for instance, some participants 
%might consider a normal speed movement as slow speed movement and some other
%the viceversa case. That led us to our future work, where further experimentation is required 
%to understand how each participant perceive body movement speed differently.
%%which we hipothese that the differences in perception of speed
%%are related to the background of each person, for exmaple,
%%person who have receive musical training in their infancy
%%are more aware of their body movemnets [add reference]
%%It would also be intersting to ask participants
%%to move in three different speeds withouth any aconstration in order to 
%%capture the natural movements for slow, normal and faster speed 
%%arm  movements. 
%% added: Tue 17 Jul 19:26:05 BST 2018



%It is also important to note that we considered the use of normalised raw 
%time series from the inertial sensors, however, performing the reconstructed state space 
%require a dimensionality recudtion using PCA where another noramlisation of data 
%is performed for such dimensioanlity reduction.
%In constrast, computing RPs and RQAs require the creating of an UTDE matrix, 
%however, there is no extra normalisation than the normalised raw data of the input 
%of RPs and RQAs.
%added: Fri 20 Jul 00:48:14 BST 2018

%
%FUTURE WORK FOR RQA
%Similarly, Iwansky et al. \cite{iwanski1998} pointed out that two dissimilar RPs: 
%one from the R\"{o}ssler system and the other one from a sine-wave signal of varying 
%period have got equal values of REC (2.1\%) and near-equal values of 
%DET (42.9\%, 45.8\%, respectively). Where we believe other RQA's can be more 
%realiable for certain source of the time series.
%added: Fri 20 Jul 01:12:50 BST 2018
%




%HUMAN_HUMANOID IMTAITON APPLICAITONS
%The work of \cite{guneysu2014} raised an important point of not considering 
%latency of motions for velocity or symmetry of motion which can be used as 
%indicators of attention deficit, boredom, or lack of perception.
%Tue 31 Jul 19:22:29 BST 2018



% Investigate \section{Group Activity in Human-Humanoid Imitation}
%added Thu  2 Aug 21:56:20 BST 2018





%%**************************** %Zero Section  ***********************************
%\chapter{Automatic Classification}
%\section{Convolutional Neural Networks (CNN)}
%\lipsum[0-4]
%
%\section{CNN Using time-series}
%

%added Thu  2 Aug 23:40:14 BST 2018



FUTURE WORK WITH
\subsection{Other methodologies for state space reconstruction.}
In addition to the Uniform Time-Delay Embedding method, other methods have 
recently been investigated to perform state space reconstruction.
For instance, (i) the nonuniform time-delay embedding methodology  
where the consecutive delayed copies of $\{ \boldsymbol{x}_n  \} $ are not
equidistant. Such method has been probed to create better representations 
of the dynamics of the state space to analyse, for instance, 
quasiperiodic and multiple time-scale time series over the conventional 
uniform time-delay embedding algorithm \citep{pecora2007, uzal2011, 
Quintana-Duque2012, Quintana-Duque2013, Quintana-Duque2016}, and
(ii) Uniform 2 time-delay embedding method which takes advantage 
of finding an embedding window instead of the traditional method 
of finding the embedding parameters separately \citep{gomezgarcia2014}.
In general, Uniform 2 time-delay embedding method computes $m$ with 
False Nearest Neighbour (FNN) algorithm and $\tau$ is computed as 
$\tau= d_w / (m-1)$, where $d_w$ is given by the minimisation of the 
Minimum Description Length \citep{small2004}.
However, these methods are out of the scope of the thesis but it is 
important to refer readers to those references for further investigations  
in this regard.


Future work with 
\subsection{Advanced quantifications}
In addition to the previous variables for recurrence quantification, 
\cite{marwan2007, marwan2015} investigated further quantification methodologies 
of the RP based on complex networks statics, calculation of dynamic invariants, 
study of the intermittency in the systems, applying different windowing techniques or the study of bivariate recurrence analysis for correlations, 
couplings, coupling directions or synchronisation between dynamical systems.



Future work with AMI
%Similarly as Cao's algorithm negatives, AMI's algorithm is not an
%exception for negatives, which are worthwhile to mention for further 
%investigations.
It is not clear why the choose of the first minimum of the AMI is the 
minimum delay embedding parameter \citep{kantz2003} or why the probability 
distribution of the AMI function is computed with the use of histograms 
which depends on a heuristic choice of number of bins for AMI partitioning 
\citep{garcia2005e71}. Further, "the method is proposed for two dimensional 
reconstructions and then extended to be used in a multidimensional case 
which is not necessarily hold in higher dimensions" 
\citep[p. 156]{gomezgarcia2014}.



