%*******************************************************************************
%******************************* Seventh Chapter *******************************
%*******************************************************************************
\chapter{Conclusions and future work} \label{chapter7}
%*******************************************************************************

\section{Conclusions}
In this thesis, nonlinear analysis methods have been explored from 
reconstructed state spaces (RSS) with uniform time-delay embedding (UTDE)
to recurrence quantification analysis (RQA) with recurrence plots (RP).
However, it is necessary to compute and then select appropriate 
embedding parameters before using any of the tools for 
nonlinear analysis (see Chapter \ref{chapter3}).
%and it is still an open problem \citep{gomezgarcia2014}.
\cite{iwanski1998} stated that 
patterns in recurrence plots and metrics for recurrence quantification
analysis are independent of embedding dimension parameters. 
However, that is not the case for different recurrence thresholds. 
Hence, embedded parameters and recurrence thresholds were considered 
to create three dimensional surface plots of recurrence quantification analysis 
which was hypothesised to be a better approach to understand the impact 
of different characteristic of real-world time series data 
such as window size length, participants, sensors and levels of smoothness.

No scientific work has been reported 
regarding the use of nonlinear analysis (e.g. RSS with UTDE, RP and RQA) to 
quantify movement variability in the context of human-humanoid interaction.
This thesis has explored the weaknesses and strengths of RQA
using 3D surfaces of the variation of embedding parameters and recurrence
thresholds which lead me to conclude that this approach requires 
less parametrization than others used in this thesis 
(e.g. RSS with UTDE, RP and RQA). 
Additionally, it was found that the 3D surface plots for RQA ENTR metric 
can be used to model any of the effects of movement variability for
different activities or different participants
as well as the post processing of real-world time series data 
with different window size length,
smoothness and structures (shape, amplitude and phase) of time series
(see Sections of weaknesses and strengths of RQA in Chapters 
\ref{chapter5} and \ref{chapter6}).

In the following sections, positives and negatives of this thesis 
are pointed out by answering the raised research questions 
posed in Chapter \ref{chapter1}.

\subsection*{
	What are the effects on RSSs, RPs, and RQA metrics
	of different embedding parameters, different recurrence thresholds 
	and different characteristics of time series 
	(structure, smoothness and window length size)?
}
It is evident that time series from different sources 
of time series (e.g. participants, movements, axis type, window size lengths 
or levels of smoothness) present differences for not only
embedding parameters but also for the patterns in RSS, RP, RQA 
and 3D surfaces of RQA metrics.
With that in mind, it can be concluded that the selection 
of appropriate embedding parameters and recurrence threshold 
is crucial to get meaningful results from nonlinear analysis tools. 
However, in this thesis it has been found that the creation of 
3D surface plots of RQA metrics is a new approach that is 
independent of the type of time series
and the selection of embedding parameters. 
Specifically, it was found that 3D RQA ENTR is robust against 
different sources of time series data, which can led to insight 
into the quantification of movement variability. 

\subsection*{What are the weaknesses and strengths of RQA metrics 
	when quantifying movement variability?
}
From the reported results in chapters 
\ref{chapter5} and \ref{chapter6}, it can be stated that the 
weaknesses of RQA, investigated in this thesis, are three:  
(i) the requirement of an expert(s) 
to interpret and compute embedding parameters and 
recurrence thresholds,  
(ii) the implementation and computation of methods of 
nonlinear analysis is laborious and computation of the parameters 
for such methods is still an open problem, and 
(iii) the selection of particular parameters to apply 
methods of nonlinear analysis does not necessarily give 
the best representation of the dynamics of the time series.

Hence, by proposing a variation of embedded parameters and recurrence 
thresholds to create 3D surfaces of RQA, it can be stated 
two strengths of RQA metrics:  
(i) little set up of parametrisation for 3D RQA metrics is required and 
(ii) 3D RQA ENTR might be a suitable approach to give insight 
to the understanding of the dynamics of different 
characteristic of time series.

\subsection*{How does the smoothing of raw time series affect 
	methods of nonlinear analysis when 
	quantifying movement variability?
}
The answer to this question depends on 
(i) what to quantify in movement variability and also 
(ii) which hardware is involved in the collection of 
time-series data. 
For instance, to avoid erratic changes in the metrics
of nonlinear analysis, smoothing raw signals can both help
to obtain well defined trajectories in RSS and patterns in RP
as well as constant values in RQA's metrics.
However, on one hand, it has been observed that the 
increase of smoothness of time-series data created 
more complex trajectories (i.e. not well defined) 
in the Reconstructed State Spaces and also added 
more black dots in Recurrence Plots 
(see RSSs and RPs sections in Chapters \ref{chapter5} and \ref{chapter6}).
On the other hand, two metrics of RQA 
(e.g. DET and ENTR) are more robust against the 
effect of smoothness of time series.

Additionally, smoothing time-series data can preserve 
the structure of the dynamics of NAO's arm movements 
when applying nonlinear analysis, as sometimes NAO 
produces jerky arm movements due to 
(i) its 14 degrees of freedom (DOF) for arms and head,
(ii) the range of joint movement, 
(iii) joint torques and velocities, 
(iv) control of dynamic response (fast acceleration/deceleration), 
(v) stiffness of gear mechanics, or
(vi) the number of degrees of freedom
(see \cite{gouaillier2009} for more references
on NAO's mechanical and dynamic capabilities).

\section{Future work}

\subsection*{Inertial sensors}
To have fundamental understating of the nature of signals collected 
through inertial sensor in the context of human-robot interaction,
future experiments can be conducted considering the application of 
derivates of the accelerometer data. 
With that in mind, the following points can be explored 
(i) both the jerkiness of movements and the 
nature of arm movements which typically have minimum 
jerk \citep{flash1985},
(ii) the relationship of movement between different body parts, 
for instance, how rapidly or slowly a person performs arm and leg 
movements \citep{devries1982, mori2012} or (iii) 
the application, to real-world time series data, of higher derivatives 
of displacement with respect time such as jounce, snap, 
crackle and pop \citep{eager2016}.

\subsection*{Smoothing time-series data}
It has been hypothesised that one might create a closer 
representation of the nature of movement variability 
when using raw data from sensors.
However, the quality of raw time-series data from inertial
sensors can be affected by changes in sample rate, 
drift effect of long time-series data 
or changes of external variables such as 
temperature and magnetic fields to inertial sensors. 
Additionally, humanoid robots can sometimes produce
jerky movements due its mechanical and dynamic capabilities.
That said, further investigation is required to be done
regarding the search of the appropriate balance between 
and the raw data and the degree of smoothness 
that can get closer to the quantification of the nature of 
movement variability in the context of 
human-humanoid interaction. 

\subsection*{Surrogate data analysis}
Non-stationarity and non-linearity of experimental time-series data 
were assumed in this thesis 
(see Chapter \ref{chapter1}).
Such assumption was made based on the ambiguity of 
nonlinear analysis methods to quantify movement variability
and the not yet fully explored area of application of nonlinear analysis 
methods in human-humanoid interaction
(see Chapters \ref{chapter1} and \ref{chapter2}). 
From the examiners of the PhD viva, 
one recommendation to avoid such prejudice of the type of data  
is to test the non-linearity and non-stationarity  
of the experimental time series data before nonlinear analysis 
methods are applied.
Hence, a possible avenue to tackle such caveat 
is to apply surrogate data analysis to test that 
data have not been generated by "a stationary Gaussian linear
stochastic process that is observed through an invertible,
static, but possible linear stochastic function" 
\citep[p. 2]{schreiber2000}.
However, applying surrogate data analysis to time series data 
that show strong periodicity or quasi-periodicity 
might create misleading results and perhaps provide unfair 
conclusion 
(see Figures in \codelink{
https://github.com/mxochicale-phd/thesis/tree/master/0_code_data/1_code/x_surrogate/00_timeseries/doc/surrogatedata.pdf} that illustrate
how different realisations of the same periodic sinusoidal signal 
show to be sometimes stationarity and others non-stationarity).
That said, further research require to be done,
perhaps consider the works of 
\cite{stam1998} and  \cite{small2002}
to test weak non-stationarity 
of periodic and quasi-periodic time series data.
Also, for future work, 
it can be considered other time series data from 
activities that involve more than one joint 
in order to test the robustness of 
not only nonlinear analysis methods but 
also surrogate data analysis.

%Therefore, it can be said that further research
%requires to be done 
%to avoid doing any assumption to the type of data
%when quantifying the phenomena of movement 
%variability in the context
%of human-humanoid interaction.
%and the reliablity of these tools with 
%periodic and quatisiperidoc experimental times 
%with weak non-stationarity  \citep{stam1998, schreiber2000}.

%non-stationarity and non-linearity of experimental time-series data 
%get closer to the use of the appropriate 
%type of movements for humans and the robot.
%Perhaps, apply the work of \cite{stam1998} 
%
%* show the fairness of the methods
%* arsenal of linear methods
%* linear stochastic signal

%after the ambiguity found in 
%different methodologies of nonlinear dynamics that 
%quantify movement variability 
%(see Chapters \ref{chapter1} and \ref{chapter2}).
%To do that, surrogate data analysis has been performed 
%for time-series data from the experiments and 
%it can be said that the null hypothesis has only been rejected 
%18 out 20 of the performance of vertical faster arm movements.
%Whereas it has been rejected 11 times for vertical faster arm movements
%from data of the participants.

%Hence, more research require to be done as 
%the null hypothesis is rejected and non-rejected for similar 
%type of movements for humans and the robot.
%Perhaps, apply the work of \cite{stam1998} 
%for non-linearity test in experimental time series with 
%strong periodic components using the method of 
%nonlinear cross-prediction.

%See inkscape $HOME/phd/phd-thesis/0_code_data/1_code/x_surrogate/01_hri/vector/preliminary_results_07_05_2019.svg
%that require to use more joints biomechanical degrees of freedom (i.e. faster movements),
%hence data show properties of being non-linear and non-stationary.


\subsection*{Nonlinear analysis}

\subsubsection*{Optimal embedding parameters}
The method of False Nearest Neighbour \citep{Cao1997} states that 
values of $E_1(m)$ become insensitive to the increase of dimension, 
for which, in this thesis, a threshold has been defined in 
order to obtain the minimum embedding dimension $m_0$. 
However, a further investigation is required to be done for the 
selection of the threshold in the $E_1(m)$ plots, 
as there were no particular method 
but visual inspection of the $E_1(m)$ curves to set such a threshold 
(see Section \ref{ch3:fnn} in Chapter \ref{chapter3}).
Similarly, further research is required to be done with regards to the 
selection of the minimum delay embedding because it is not clear:
(i) why the choice of the first minimum of the AMI is the minimum delay 
embedding parameter \citep{kantz2003} or 
(ii) why the probability distribution of the AMI function 
is computed with the use of histograms which depend on a heuristic 
selection of number of bins for the AMI partitioning \citep{garcia2005e71}.
Additionally, "the AMI method is proposed for two dimensional 
reconstructions and extended to be used in a multidimensional case 
which is not necessarily held in higher dimensions" 
\citep[p. 156]{gomezgarcia2014}.

\subsubsection*{Other methodologies for state space reconstruction.}
In addition to the method of Uniform Time-Delay Embedding to reconstruct
state spaces, other methods have been stated a better dynamic 
representations of time series in the reconstructed state spaces such as: 
(i) the nonuniform time-delay embedding methodology  
where the consecutive delayed copies of $\{ \boldsymbol{x}_n  \} $ are not
equidistant
\citep{pecora2007, uzal2011, 
Quintana-Duque2012, Quintana-Duque2013, Quintana-Duque2016}, or 
(ii) the uniform 2 time-delay embedding method which takes advantage 
of finding an embedding window instead of the traditional method 
of finding the embedding parameters separately \citep{gomezgarcia2014}.
As a future work, it might be worthwhile to apply (i) and (ii) 
methods to the current problem. 

\subsubsection*{RP and RQA parameters}
There are different avenues that can be investigated 
with regard to the computation of RP and RQA parameters.
However from this thesis, it is suggested that the work of 
\cite{marwan2007} and \cite{marwan2015} can be the starting point 
for further research with regards to different criteria for 
(i) neighbours, 
(ii) different norms ( $L_{1-norm}$, $L_{2-norm}$, or $L_{\infty-norm}$ ) or 
(iii) different methods to select the recurrence thresholds such as: 
using only certain percentage of the signal
($\sqrt{m_0} \times$ 10\% of the fluctuations of the time series)
\citep{letellier2006}, and selecting a determined amount of noise, and 
using a factor based on the standard deviation of the 
observational noise \citep{marwan2007}.

\subsubsection*{Robustness of Entropy measures with RQA}
Further investigation is required 
to be done with regards to the application of Shannon entropy with 
recurrence plots.
\cite{letellier2006}, for example, investigated the robustest of 
the Shannon entropy based on line segments distributions of 
recurrence plots $S_{RP}$ 
against the Shannon entropy based on system dynamics $S_{SD}$.
With that, \cite{letellier2006} pointed out that Shannon entropy based on 
recurrence plots has strong dependency with the choice of observable 
(i.e. variable of the dynamical system) while Shannon entropy based on 
system dynamics is more robust to noise-contaminated signals.
Recently, with the introduction of the use of microstates, 
\cite{corso2017} tackled the problem of Shannon entropy with RQA
where ENTR values decrease despite the increase of non-linearity 
in a logistic map \citep{marwan2007}. 
Additionally, \cite{corso2017} presented the robustness of their method 
with changes to recurrence thresholds.

\subsubsection*{Advanced RQA quantifications}
In addition to the application of RQA metrics (REC, RATIO, DET and ENTR) for 
recurrence quantification, advanced RQA metrics can be applied to 
the context of human-humanoid interaction. 
For example, RP based on complex networks statics, 
calculation of dynamic invariants, study of the intermittency in the systems, 
application of different windowing techniques, or 
the study of bivariate recurrence analysis for correlations, 
coupling directions or synchronisation between dynamical systems
\citep{marwan2007, marwan2015}.

\subsection*{Variability in perception of velocity}
While conducting the experiments where participants performed 
arm movements with different velocities (e.g. normal and faster),
it has been noted that participants perceive velocity differently.
Particularly, some participants considered a normal velocity movement 
as being performed in slow velocity and others participants considered 
a slow velocity movement as being performed in normal velocity. 
With that in mind, it has been hypothesised that the differences 
in perception of velocities are related to different factors 
of a person such as 
(i) the background, 
(ii) personality traits or
(iii) even their movement experience 
(in music or sports) that make them more aware of their body movements. 
That said, further research require to be done to have better 
understanding on why each participant perceive the velocity of 
body movement differently, how such variability of perception 
of movement can be quantified, and 
what impact such differences might have for the control of 
movement or for the ability to recognise decrease in control
ability. 

\subsection*{A richer dataset of real-world time series}
It should be highlighted that the experiments for this thesis are 
limited to twenty three healthy right-handed participants of a 
range age of mean 19.8 and SD=1.39.
Hence, participants of different ages, state of health and 
anthropomorphic features would create a richer dataset of 
real-world time series data to apply nonlinear analysis tools 
in the context of human-humanoid interaction. 

\subsection*{Applications}
The application of the literature in human movement variability 
in the context of human-humanoid interaction can present different
avenues.
For instance, implement nonlinear analysis algorithms in humanoid robots 
in order to 
(i) evaluate the improvement of movement performances \citep{muller2004}, 
(ii) quantify and provide feedback of level skillfulness as a function 
of movement variability \citep{seifert2011} or 
(iii) quantify movement adaptations, pathologies and skill learning 
\citep{preatoni2007, preatoni2010, preatoni2013}.
Also applications in human-humanoid rehabilitation 
\citep{gorer2013, guneysu2015}, 
where the use of nonlinear analysis can provide adequate 
metrics to quantify and provide feedback for movement variability. 

