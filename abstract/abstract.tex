% ************************** Thesis Abstract *****************************
%% This abstract were written folllwing the nature's template for abstracts 
%% (https://twitter.com/trevorabranch/status/620699527486373888?lang=en)


%%%% 200 words %%%%
\begin{abstract}
Nonlinear analyses investigate the dynamics of observed time-ordered data.
Such dynamics, for this thesis, are complex systems of sensorimotor 
variables of movement variability (MV) in the context of 
human-humanoid interaction.
Hence, this dissertation not only explores questions such as what to quantify in MV?,
or which nonlinear tools are appropriate to quantify MV?, but also how 
nonlinear analyses are affected with real-world time series data
(e.g. nonstationary, data length limitations, sampling rate changes or noisiness).
Particularly, I review nonlinear tools such as methods 
to determine embedding parameters, Reconstructed State Spaces (RSSs), 
Recurrence Plots (RPs) and Recurrence Quantification Analyses (RQA).
To my knowledge, I can conclude that, no scientific work has been reported 
regarding nonlinear analyses (e.g. RSSs with UTDE, RPs and RQAs) to 
quantify movement variability in the context of human-humanoid interaction.
Also, we created 3D surfaces of RQA
values considering the variation of embedded parameters and 
recurrence thresholds to show that 3D surfaces of RQA ENTR might be a 
better approach to provide understanding on the dynamics of different 
characteristic of real-world time series data. 
I can foresee many areas of applications where humanoids robots
can be pre-programmed with nonlinear analyses algorithms 
to evaluate the improvement of movement performances,
to quantify and provide feedback of skill learning
or to quantify movement adaptations and pathologies.

\end{abstract}
