% ************************** Thesis Abstract *****************************
%% This abstract were written folllwing the nature's template for abstracts 
%% (https://twitter.com/trevorabranch/status/620699527486373888?lang=en)


%%%% 200 words %%%%
\begin{abstract}
%%%%%%%%%%%%%%%%%%%%%%%%%%%%%%%%%%%%%%%%%%%%%%%%%%%%%%%%%%%%%%%%%%%%%%%%%%%%%%%%
%%%%% One or two sentences proving a basic introduction to the field,
%%%%% comprehensible to a scientist in any discipline.
Nonlinear analyses investigate the dynamics of observed time-ordered data.
Such dynamics, for this thesis, are complex systems of sensorimotor 
variables of movement variability (MV) in the context of 
human-humanoid interaction.
%31 words
%movement variability (MV) and also in the mechanical 
%limitations of a humanoid robot (robot with the general structure of a human).
%%%%%%%%%%%%%%%%%%%%%%%%%%%%%%%%%%%%%%%%%%%%%%%%%%%%%%%%%%%%%%%%%%%%%%%%%%%%%%%%
%%%%%%%%%%%%%%%%%%%%%%%%%%%%%%%%%%%%%%%%%%%%%%%%%%%%%%%%%%%%%%%%%%%%%%%%%%%%%%%%
%%%%% Two to three sentences of more detailed background,
%%%%% comprehensible to scientist in related disciplines.
Hence, this dissertation not only explores questions such as what to quantify in MV?,
or which nonlinear tools are appropriate to quantify MV?, but also how 
nonlinear analyses are affected with real-world time series data
(e.g. nonstationary, data length limitations, sampling rate changes or noisiness).
%46 words
%We conducted two experiment in human-humanoid interaction where participants 
%imitate simple horizontal and vertical arm movements of a humanoid robot.
%to test the 
%Given that traditional methods in time-domain and frequency domain fail to
%detect tiny modulations in frequency or phase of time series,
%we consider a methodology from nonlinear dynamics called uniform reconstructed state space 
%to quantify movement variability which essentially the dynamics of
%an unknown system can be reconstructed using one dimensional time series.
%As pointed out by Bradley et al. uniform reconstructed state space,
%if done right, can guarantee to be topologically identical to the true dynamics
%and determine dynamics invariants such as fractal dimension, Kolmogorov-Sinai
%entropy or Lyaponov exponents.
%These algorithms, however, require time series measured with costly sensors
%that provide well sampled data with little noise.
%Such requirement is generally a common problem when doing precise
%characterisation of time series using dynamic invariants,
%to which Bradley et al. proposed additional tools of
%nonlinear time series analysis for practitioners such as surrogate data,
%permutation entropy, recurrence plots and 
%network characteristics for time series
%%%%%%%%%%%%%%%%%%%%%%%%%%%%%%%%%%%%%%%%%%%%%%%%%%%%%%%%%%%%%%%%%%%%%%%%%%%%%%%%%
%%%%%%%%%%%%%%%%%%%%%%%%%%%%%%%%%%%%%%%%%%%%%%%%%%%%%%%%%%%%%%%%%%%%%%%%%%%%%%%%%
%%%%%% One sentence clearly stating the general problem being addressed by this
%%%%%% particular study.
Particularly, I review nonlinear tools such as methods 
to determine embedding parameters, Reconstructed State Spaces (RSSs), 
Recurrence Plots (RPs) and Recurrence Quantification Analyses (RQA).
%25 words
%(window size length) and 
%preprocessing techniques (e.g. smoothing and normalisation) 
%%%%%%%%%%%%%%%%%%%%%%%%%%%%%%%%%%%%%%%%%%%%%%%%%%%%%%%%%%%%%%%%%%%%%%%%%%%%%%%%
%%%%%%%%%%%%%%%%%%%%%%%%%%%%%%%%%%%%%%%%%%%%%%%%%%%%%%%%%%%%%%%%%%%%%%%%%%%%%%%%
%%%%%% One sentence summarising the main result (with the words "here we show"
%%%%%% or their equivalent)
%such as window size length, participants, sensors and levels of smoothness
%(see weaknesses and strengths of RQA in Chapters \ref{chapter5} and 
%\ref{chapter6}).
%So, here we show the characterisation for time series to understand 
%human movement variability in the
%context of human-humanoid imitation activities and demonstrate the potential 
%of nonlinear techniques to quantify human movement variability. 
%%%%%%%%%%%%%%%%%%%%%%%%%%%%%%%%%%%%%%%%%%%%%%%%%%%%%%%%%%%%%%%%%%%%%%%%%%%%%%%%%
%%%%%%%%%%%%%%%%%%%%%%%%%%%%%%%%%%%%%%%%%%%%%%%%%%%%%%%%%%%%%%%%%%%%%%%%%%%%%%%%%
%%%%%% Two or three sentences explaining what the main results reveals in direct
%%%%%% comparison to what was thought to be the case previously, or how the main
%%%%%% results adds to previous knowledge.
To my knowledge, I can conclude that, no scientific work has been reported 
regarding nonlinear analyses (e.g. RSSs with UTDE, RPs and RQAs) to 
quantify movement variability in the context of human-humanoid interaction.
%34 words
%thorough experimentation and exploration to test 
%the weaknesses and robustness of such tools, 
Also, we created 3D surfaces of RQA
values considering the variation of embedded parameters and 
recurrence thresholds to show that 3D surfaces of RQA ENTR might be a 
better approach to provide understanding on the dynamics of different 
characteristic of real-world time series data. 
%44 words
%Specifically, we explore the reconstruction of state spaces, its recurrence plots
%and metrics of recurrence quantification analysis for
%20 participants performing repetitions of simple vertical and horizontal arm
%movements in normal and faster speed.
%We also explore the differences between wearable inertial sensors attached
%to the person and to the humanoid robot and between different axes of inertial sensors.
%With that in mind, our contribution to knowledge is in regard to the
%reliability of data from cheap wearable inertial sensors
%to analyse human movement variability in the context of human-humanoid imitation activities
%using methodologies of nonlinear dynamics.
%%%%%%%%%%%%%%%%%%%%%%%%%%%%%%%%%%%%%%%%%%%%%%%%%%%%%%%%%%%%%%%%%%%%%%%%%%%%%%%%%
%%%%%% One or two sentences to put the results into a more general context.
%Such understanding and measurement of movement variability using
%cheap wearable inertial sensors lead us to have a more intuitive selection of parameters
%to reconstruct the state spaces and to create meaningful interpretations
%of the recurrence plots and the results of the metrics with recurrence quantification 
%analysis. Additionally, having a better understanding of
%nonlinear dynamics tools with the use of cheap inertial sensors
%can enhance the development of better diagnostic tools for various pathologies 
%which can be applied in areas of rehabilitation, entertainment or sport science.
%%%%%%%%%%%%%%%%%%%%%%%%%%%%%%%%%%%%%%%%%%%%%%%%%%%%%%%%%%%%%%%%%%%%%%%%%%%%%%%%%
%%%%%% Two or three sentences to provide a broader perspective, 
%%%%%% readily comprehensible to a scientist in any discipline, 
%%%%%% may be included in the first paragraph
%%%%%% if the editor considers that the accessibility of the paper is 
%%%%%% significantly enhanced by their inclusion. Under this circumstances, 
%%%%%% the length of the paragraph can be up to 300 words
I can foresee many areas of applications where humanoids robots
can be pre-programmed with nonlinear analyses algorithms 
to evaluate the improvement of movement performances,
to quantify and provide feedback of skill learning
or to quantify movement adaptations and pathologies.
%38 words




\end{abstract}
