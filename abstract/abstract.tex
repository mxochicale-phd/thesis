% ************************** Thesis Abstract *****************************

\begin{abstract}
Nonlinear analysis can be applied to investigate the dynamics of time-ordered data.
Such dynamics relate to sensorimotor 
variability in the context of human-humanoid interaction.
Hence, this dissertation not only explores questions such as 
how to quantify movement variability 
or which methods of nonlinear analysis are appropriate 
to quantify movement variability 
but also how methods of nonlinear analysis are affected 
by real-world time series data (e.g. non-stationary, data length size, 
sampling rate changes or noise).
Methods are explored to determine embedding parameters, 
reconstructed state spaces, recurrence plots and 
recurrence quantification analysis. 
Additionally, this thesis presents three dimensional surface plots of 
recurrence quantification analysis with which to consider 
the variation of embedded parameters and recurrence thresholds.
These show that three dimensional 
surface plots of Shannon entropy might be a suitable approach 
to understand the dynamics of real-world time series data. 
This thesis opens new avenues of applications in human-humanoid interaction
where humanoid robots can be pre-programmed with nonlinear analysis algorithms 
to evaluate, for instance, the improvement of movement performances,
to quantify and provide feedback of skill learning
or to quantify movement adaptations and pathologies.
\end{abstract}
% 194 words

