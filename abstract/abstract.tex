% ************************** Thesis Abstract *****************************
% Use `abstract' as an option in the document class to print only the titlepage and the abstract.
\begin{abstract}


% I follow the guidelines of the nature's template for abstracts (https://twitter.com/trevorabranch/status/620699527486373888?lang=en)

%%%%%%%%%%%%%%%%%%%%%%%%%%%%%%%%%%%%%%%%%%%%%%%%%%%%%%%%%%%%%%%%%%%%%%%%%%%%%%%%
%%%%% One or two sentences proving a basic introduction to the field,
%%%%% comprehensible to a scientist in any discipline.
Movement variability is defined as the variations that occur in motor
performance across multiple repetitions of a task and such behaviour is an
inherent feature within and between each persons' movement.
In the previous four decades, research on measurement and understanding of
movement variability with methodologies of nonlinear dynamics has been well established
in areas such as biomechanics, sport science, psychology, cognitive science,
neuroscience and robotics.
%%%%%%%%%%%%%%%%%%%%%%%%%%%%%%%%%%%%%%%%%%%%%%%%%%%%%%%%%%%%%%%%%%%%%%%%%%%%%%%%
%%%%% Two to three sentences of more detailed background,
%%%%% comprehensible to scientist in related disciplines.
Given that traditional methods in time-domain and frequency domain fail to
detect tiny modulations in frequency or phase of time series,
we consider a methodology from nonlinear dynamics called uniform reconstructed state space 
to quantify movement variability which essentially the dynamics of
an unknown system can be reconstructed using one dimensional time series.
As pointed out by Bradley et al. uniform reconstructed state space,
if done right, can guarantee to be topologically identical to the true dynamics
and determine dynamics invariants such as fractal dimension, Kolmogorov-Sinai
entropy or Lyaponov exponents.
These algorithms, however, require time series measured with costly sensors
that provide well sampled data with little noise.
Such requirement is generally a common problem when doing precise
characterisation of time series using dynamic invariants,
to which Bradley et al. proposed additional tools of
nonlinear time series analysis for practitioners such as surrogate data,
permutation entropy, recurrence plots and network characteristics for time series.
%%%%%%%%%%%%%%%%%%%%%%%%%%%%%%%%%%%%%%%%%%%%%%%%%%%%%%%%%%%%%%%%%%%%%%%%%%%%%%%%
%%%%% One sentence clearly stating the general problem being addressed by this
%%%%% particular study.
For this study, we are thefore interested in the use of uniform reconstructed
state space and the analysis of recurrence plots and metrics of recurence
quantification analysis so as to understand the quantification of movement variability.
Particularly, we are interested in the analysis of data collected through
cheap wearable inertial sensors and its effects on the
reconstructed state space, recurrence plots and metrics for recurrence 
quantification analysis for different window lengths and preprocessing techniques
(like smoothing and normalisation) of the time series.
%%%%%%%%%%%%%%%%%%%%%%%%%%%%%%%%%%%%%%%%%%%%%%%%%%%%%%%%%%%%%%%%%%%%%%%%%%%%%%%%
%%%%% One sentence summarising the main result (with the words "here we show"
%%%%% or their equivalent)
So, here we show the characterisation for time series to understand human movement variability in the
context of human-humanoid imitation activities and demostrate the potential 
of nonlinear techniques to quantify human movement varialibity. 
%%%%%%%%%%%%%%%%%%%%%%%%%%%%%%%%%%%%%%%%%%%%%%%%%%%%%%%%%%%%%%%%%%%%%%%%%%%%%%%%
%%%%% Two or three sentences explaining what the main results reveals in direct
%%%%% comparison to what was thought to be the case previously, or how the main
%%%%% results adds to previous knowledge.
Specifically, we explore the reconstruction of state spaces, its recurrence plots
and metrics of recurrence quantification analysis for
20 participants performing repetitions of simple vertical and horizontal arm
movements in normal and faster speed.
We also explore the differences between wearable inertial sensors attached
to the person and to the humanoid robot and between different axes of inertial sensors.
With that in mind, our contribution to knowledge is in regard to the
reliability of data from cheap wearable inertial sensors
to analyse human movement variability in the context of human-humanoid imitation activities
using methodologies of nonlinear dynamics.
%%%%%%%%%%%%%%%%%%%%%%%%%%%%%%%%%%%%%%%%%%%%%%%%%%%%%%%%%%%%%%%%%%%%%%%%%%%%%%%%
%%%%% One or two sentences to put the results into a more general context.
Such understanding and measurement of movement variability using
cheap wearable inertial sensors lead us to have a more intuitive selection of parameters
to reconstruct the state spaces and to create meaningful interpretations
of the recurrence plots and the results of the metrics with recurrence quantification 
analysis. Additionally, having a better understanding of
nonlinear dynamics tools with the use of cheap inertial sensors
can enhance the development of better diagnostic tools for various pathologies 
which can be applied in areas of rehabilitation, entertainment or sport science.
%%%%%%%%%%%%%%%%%%%%%%%%%%%%%%%%%%%%%%%%%%%%%%%%%%%%%%%%%%%%%%%%%%%%%%%%%%%%%%%%
%%%%% Two or three sentences to provide a broader perspective, readily comprehensible
%%%%% to a scientist in any discipline, may be included in the first paragraph
%%%%% if the editor considers that the accessibility of the paper is significantly
%%%%% enhanced by their inclusion. Under this circumstances, the length of the
%%%%% paragraph can be up to 300 words




\end{abstract}
